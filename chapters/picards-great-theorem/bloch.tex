\section{Bloch's Theorem}
\label{sec:blochs-theorem}

If $G \subseteq \C$ is a domain and $f \in H(G)$ is non-constant, then $f(G)$ is a domain as well. In particular, $f(G)$ contains open disks of some, potentially very small, radius. Bloch's Theorem asserts that for any $f \in H(\D) \cap C(\cl{\D}; \C)$ satisfying $f'(0) = 1$, the set $f(\D)$ always contains a disk of fixed radius.

Note that $f(0)$ may not always be the center of such a disk; consider the sequence
\begin{equation*}
    f_n(z) \coloneqq \frac{1 - e^{- n z}}{n} \in H(\D) \cap C(\cl{\D}; \C), \quad n \in \N,
\end{equation*}
which satisfies $f_n(0) = 0$ and $f_n'(0) = 1$, but omits the value $1 / n$.

If $f$ is as before, then $f$ is an open mapping on $G$, that is $f$ maps open sets to open sets. We thus first observe a general criterion, not relying on the holomorphic property, for the size of disks in the image domain of such functions:

\begin{lemma} \label{lem:bloch-lemma-1}
    Let $G \subset \C$ be a bounded domain and $f \in C(\cl{G}; \C)$ such that $\restr{f}{G}$ is an open mapping. Let $a \in G$ and set $s \coloneqq d(f(a), f(\partial G))$. If $s > 0$, then $B_{s}(f(a)) \subseteq f(G)$.
\end{lemma}

\begin{proof}
    Since $G$ is bounded, $\cl{G}$ is compact and, by continuity of $f$, so is $\cl{f(G)}$.
    The function $z \mapsto \vert z - f(a) \vert$ is continuous on the compact set $\partial f(G)$, hence it assumes its minimum $m$ at some $w_* \in \partial f(G)$. Choose a sequence $(z_n)_{n \in \N}$ in $G$ with $\lim_{n \to \infty} f(z_n) = w_*$, then, since $\cl{G}$ is compact, we can find a subsequence that converges to some $z_* \in \cl{G}$. By continuity of $f$, we have $f(z_*) = w_*$.

    If $z_* \in G$, since $\restr{f}{G}$ is open, the image of any open set in $G$ containing $z_*$ under $f$ is an open set in $f(G)$ containing $w_*$, which is impossible since $w_* \in \partial f(G)$.

    Therefore, $z_* \in \partial G$ and we have
    $$ d(f(a), \partial f(G)) = m = \vert w_* - f(a) \vert = \vert f(z_*) - f(a) \vert \geq s, $$
    which implies $B_{s}(f(a)) \subseteq f(G)$ if $s > 0$.
\end{proof}

\begin{lemma} \label{lem:bloch-lemma-2}
    Fix $a \in \C, r > 0$ and let $B \coloneqq B_{r}(a)$. Suppose further $G \subseteq \C$ is a domain such that $\cl{B} \subset G$ and $f \in H(G)$ such that $\Vert f' \Vert_B \leq 2 \vert f'(a) \vert$. Then $ B_{R}(f(a)) \subseteq f(B)$, where $ R \coloneqq (3 - 2 \sqrt{2}) r \vert f'(a) \vert $.
\end{lemma}

\begin{proof}
    We may assume $a = f(a) = 0$, otherwise we consider $f_1(z) \coloneqq f(z + a) - f(a)$. The function
    $$ \mapping{\alpha_f}{B & \to & \C,}{z & \mapsto & f(z) - f'(0) z,} $$
    satisfies, for all $z \in B$,
    \begin{equation} \label{eq:bloch-lemma-2:estimate-1}
        \vert \alpha_f(z) \vert = \left\vert \int_{[0, z]} f'(\zeta) - f'(0) \diff \zeta \right\vert \leq \int_0^1 \vert f'(t z) - f'(0) \vert \vert z \vert \diff t. \tag{\textasteriskcentered}
    \end{equation}
    We wish to further estimate the integrand. Let $w \in B$, then Cauchy's integral formula gives
    \begin{align*}
        \vert f'(w) - f'(0) \vert &= \frac{1}{2\pi} \left\vert \oint_{\partial B} \frac{f'(\zeta)}{\zeta - w} - \frac{f'(\zeta)}{\zeta} \diff \zeta \right\vert = \frac{1}{2\pi} \left\vert \oint_{\partial B} \frac{w f'(\zeta)}{\zeta (\zeta - w)} \diff \zeta \right\vert \leq \\
        &\leq \frac{1}{2\pi} \oint_{\partial B} \frac{\vert w \vert \Vert f' \Vert_B}{r(r - \vert w \vert)} \diff \zeta = \frac{\vert w \vert}{r - \vert w \vert} \Vert f' \Vert_B.
    \end{align*}
    Combining the above with \eqref{eq:bloch-lemma-2:estimate-1} and our estimate on $\Vert f' \Vert_B$ yields
    \begin{align*}
        \vert \alpha_f(z) \vert &\leq \int_0^1 \frac{\vert z t \vert \Vert f' \Vert_B}{r - \vert z t \vert} \vert z \vert \diff t \leq \frac{\vert z \vert^2}{r - \vert z \vert} \Vert f' \Vert_B \int_0^1 t \diff t \leq \frac{\vert z \vert^2}{r - \vert z \vert} \vert f'(0) \vert.
    \end{align*}
    Let $0 < \rho < r$, then for $\vert z \vert = \rho$ we have
    \begin{align*}
        \vert f'(0) \vert \rho - \vert f(z) \vert &\leq \vert \alpha_f(z) \vert \leq \frac{\rho^2}{r - \rho} \vert f'(0) \vert \\
        \Longleftrightarrow \quad \vert f(z) \vert &\geq \left( \rho - \frac{\rho^2}{r - \rho} \right) \vert f'(0) \vert.
    \end{align*}
    The function $\rho \mapsto \rho - \rho^2 / (r - \rho)$ assumes its maximum value at $\rho_* \coloneqq (1 - 1 / \sqrt{2})r \in (0, r)$, namely $(3 - 2 \sqrt{2}) r$. Therefore,
    \begin{equation*}
        \vert f(z) \vert \geq (3 - 2 \sqrt{2}) r \vert f'(0) \vert = R, \quad \textrm{for all} \quad \vert z \vert = \rho_*.
    \end{equation*}
    In particular, $\min_{z \in \partial B_{\rho_*}} \vert f(z) \vert \geq R > 0$, thus invoking \Cref{lem:bloch-lemma-1} with the domain $B_{\rho_*}(0)$ yields $B_{R}(0) \subseteq f(B_{\rho_*}(0)) \subseteq f(B)$.
\end{proof}

\begin{theorem} \label{thm:bloch-stronger}
    Let $f \in H(\D) \cap C(\cl{\D}; \C)$ be non-constant. Then there is a point $p \in \D$ and a constant $C_f > 0$ such that $B_{R}(f(p)) \subseteq f(\D)$, where $R \coloneqq (\frac{3}{2} - \sqrt{2}) C_f \geq (\frac{3}{2} - \sqrt{2}) \vert f'(0) \vert$.
\end{theorem}

\begin{proof}
    The function
    $$ \mapping{\alpha_f}{\cl{\D} & \to & \R}{z & \mapsto & \vert f'(z) \vert (1 - \vert z \vert)} $$
    is continuous and assumes its maximum $C_f > 0$ at some point $p \in \cl{\D}$. Note that $C_f \geq \vert f'(0) \vert$ and, since $f$ is non-constant and $\restr{\alpha_f}{\partial \D} = 0$, we even have $p \in \D$.

    Set $t \coloneqq \frac{1}{2}(1 - \vert p \vert) > 0$, then we have $B_{t}(p) \subseteq \D$. Furthermore, for $z \in B_t(p)$, we have
    $$ 1 - \vert z \vert \geq 1 - \vert z - p \vert - \vert p \vert \geq 1 - t - \vert p \vert = t. $$
    Since $\vert f'(z) \vert (1 - \vert z \vert) \leq C_f = 2 t \vert f'(p) \vert$, this implies $\vert f'(z) \vert \leq 2 \vert f'(p) \vert$ for all $z \in B_t(p)$. By \Cref{lem:bloch-lemma-2}, we get $ B_{R}(f(p)) \subseteq f(\D) $, where $R \coloneqq (3 - 2 \sqrt{2}) t \vert f'(p) \vert = ({\textstyle \frac{3}{2}} - \sqrt{2}) C_f$,
    which establishes the assertion.
\end{proof}

We now immediately obtain:    

\begin{theorem}[Bloch] \label{thm:bloch}
    Let $f \in H(\D) \cap C(\cl{\D}; \C)$ and assume $f'(0) = 1$. Then $f(\D)$ contains a disk of radius $\frac{3}{2} - \sqrt{2}$.
\end{theorem}

In the following we will denote by $\beta > 0$ any constant less than or equal to the radius in Bloch's Theorem, for example $\beta = \frac{1}{12} < \frac{3}{2} - \sqrt{2}$.

\begin{corollary} \label{cor:bloch-domain}
    Let $G \subseteq \C$ be a domain and $f \in H(G)$ with $f'(c) \neq 0$ for some $c \in G$. Then $f(G)$ contains a disk of every radius $\beta s \vert f'(c) \vert$, where $0 < s < d(c, \partial G)$.
\end{corollary}

\begin{proof}
    We may assume $c = 0$, otherwise we consider $f_1(z) \coloneqq f(z+c)$. Let $0 < s < d(c, \partial G)$, then $f$ is holomorphic in and continuous on $\cl{B_s(0)} \subseteq G$, thus we have
    $$ g(z) \coloneqq \frac{f(sz)}{sf'(0)} \in H(\D) \cap C(\cl{\D}; \C). $$
    Since $g'(0) = 1$, Bloch's Theorem yields a disk $B$ of radius $\beta$ with $B \subseteq g(\D)$. Then $D \coloneqq s \vert f'(0) \vert B$ is a disk of radius $\beta s \vert f'(0) \vert$ and we have
    \begin{equation*}
        D = s \vert f'(0) \vert B \subseteq s \vert f'(0) \vert g(\D) = f(B_s(0)) \subseteq f(G). \qedhere
    \end{equation*}
\end{proof}

\begin{corollary} \label{cor:bloch-entire}
    If $f \in H(\C)$ is non-constant, then $f(\C)$ contains a disk of every radius.
\end{corollary}