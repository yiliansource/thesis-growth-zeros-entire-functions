\section{Picard's Great Theorem}
\label{sec:picards-great-theorem}

\begin{definition} \label{def:normal-family}
    Let $G \subseteq \C$ be a domain and $\F \subseteq H(G)$. Then:
    \begin{itemize}
        \item $\F$ is called \emph{locally bounded} if for every $w \in G$ there is a neighborhood $U$ of $w$ and a constant $C > 0$ such that $\vert f(z) \vert \leq C$ for all $f \in \F$ and $z \in U$.
        \item $\F$ is called \emph{normal} in $G$ if every sequence in $\F$ has a subsequence which converges compactly in $G$ to some $f \in H(G)$ or to $\infty$.
    \end{itemize}
\end{definition}

The following well-known theorem establishes the equivalence of the two concepts:

\begin{theorem}[Montel] \label{thm:montel}
    Let $G \subseteq \C$ be a domain. Then a family $\F \subseteq H(G)$ is normal if and only if it is locally bounded.
\end{theorem}

\begin{theorem}[Hurwitz] \label{thm:hurwitz}
    Let $G \subseteq \C$ be a domain and $(f_n)_{n \in \N}$ a sequence in $H(G)$ that converges compactly to $f$. If for every $n \in \N$ the number of $a$-points of $f_n$ is bounded by some $m \in \N_0$, then either the number of $a$-points of $f$ are also bounded by $m$, or $f \equiv a$.
\end{theorem}

The following lemma can be interpreted as a sharpened version of Montel's Theorem (\ref{thm:montel}) and is sometimes referred to as the \emph{fundamental normality test}.

\begin{lemma} \label{lem:montel-sharpened}
    For any domain $G \subseteq \C$ the family $\F \coloneqq H(G, \C \setminus \{ 0, 1 \})$ is normal in $G$.
\end{lemma}

\begin{proof}
    We shall give the proof in three steps:
    \begin{enumerate}
        \item Let $w \in G$, $c > 0$ and $\F_*$ be a subfamily of $\F$ such that $\vert f(w) \vert \leq c$ for all $f \in \F_*$. We aim to show that there is a neighborhood of $w$ in which $\F_*$ is bounded. Select $t > 0$ such that $\cl{B_t(w)} \subset G$. Without loss of generality we may assume $w = 0$ and $t = 1$. By the maximum modulus principle and Schottky's Theorem (\ref{thm:schottky}) we obtain
        $$ \sup \{ \Vert f \Vert_{B_{1/2}(0)} : f \in \F_* \} \leq \psi(1/2, c) < \infty. $$

        \item Fix some $w_* \in G$ and set $\F_1 \coloneqq \{ f \in \F : \vert f(w_*) \vert \leq 1 \}$. We aim to show that $\F_1$ is locally bounded (and by Montel's Theorem (\ref{thm:montel}) therefore normal) in $G$. Consider the set
        $$ U \coloneqq \{ w \in G : \F_1 \textrm{ is bounded in a neighborhood of } w \}, $$
        which is open in $G$. By (1) we have that $w_* \in U$. For sake of contradiction, assume that $U \neq G$. Then there exists some $w \in \partial U \cap G$ and a sequence $f_n \in \F_1$ with $\lim_{n \to \infty} f_n(w) = \infty$. Set $g_n \coloneqq 1 / f_n \in \F$, then $\lim_{n \to \infty} g_n(w) = 0$ and by (1) the family $(g_n)_{n \in \N}$ must be bounded in a neighborhood of $w$. By Montel's Theorem (\ref{thm:montel}) it is therefore normal and there exists a subsequence $(g_{n_k})_{k \in \N}$ which converges compactly to a $g \in H(B)$, for some disk $B$ around $w$. The functions $g_n$ have no zeros, but $g(w) = 0$; by Hurwitz's Theorem (\ref{thm:hurwitz}) we therefore have $g \equiv 0$. Then for any $z \in B \cap U$ we have
        $$ \lim_{k \to \infty} f_{n_k}(z) = \lim_{k \to \infty} 1 / g_{n_k}(z) = \infty, $$
        contradicting the assumption that $\F_1$ is bounded in a neighborhood of $z$.

        \item We can now conclude the proof. Let $(f_n)_{n \in \N}$ be a sequence in $\F$. If it has a subsequence in $\F_1$, then the claim follows from (2).
        
        On the other hand, if there are only finitely many $f_n$ in $\F_1$, then $(1 / f_n)_{n \in \N}$ has some subsequence, say $g_n$, in $\F_1$, which converges compactly in $G$ to some $g \in H(G)$. If $g$ has no zeros, then the subsequence $1 / g_n$ of the sequence $f_n$ converges compactly in $G$ to $1 / g$. If $g$ has zeros, then by Hurwitz's Theorem (\ref{thm:hurwitz}) we have $g \equiv 0$, thus $1 / g_n$ converges compactly in $G$ to $\infty$.
    \end{enumerate}
\end{proof}

\begin{lemma} \label{lem:great-picard-bounded}
    Let $f \in H(\D^\times; \C \setminus \{ 0, 1 \})$. Then $f$ or $1/f$ is bounded in a punctured neighborhood of zero.
\end{lemma}

\begin{proof}
    Set $f_n(z) \coloneqq f(z / n) \in H(\D^\times; \C \setminus \{ 0, 1 \})$. By \Cref{lem:montel-sharpened} there exists a subsequence $(f_{n_k})_{k \in \N}$ such that $(f_{n_k})_{k \in \N}$ or $(1 / f_{n_k})_{k \in \N}$ is bounded on $\partial B_{1/2}(0)$.

    In the first case, we have $\vert f(z / n_k) \vert \leq M$ for some $M > 0$ and all $\vert z \vert = \frac{1}{2}$, $k \in \N$. Thus $f$ is bounded on every circle of radius $1 / (2 n_k)$. By the maximum modulus principle, $f$ must therefore be bounded on every annulus $1 / (2 n_{k + 1}) \leq \vert z \vert \leq 1 / (2 n_k)$. This yields that $f$ is bounded in a punctured neighborhood of zero.

    In the second case, the same argument shows that $1 / f$ is bounded in a punctured neighborhood of zero.
\end{proof}

\begin{theorem}[Picard's Great Theorem] \label{thm:picards-great-theorem}
    Let $G \subseteq \C$ be open, $w \in G$ and suppose $f \in H(G \setminus \{ w \})$ such that $f$ has an essential singularity at $w$. Then $f$ assumes all values in $\C$, with at most one exception, infinitely often in any punctured neighborhood of $w$.
\end{theorem}

\begin{proof}
    Aiming for contradiction, assume that $f$ only takes on $z_0, z_1 \in \C$ finitely often in some punctured neighborhood $W$ of $w$. Then $W$ contains a punctured disk of radius $\varepsilon > 0$ around $w$, on which $f$ does not assume $z_0, z_1$. The function
    $$ g(z) \coloneqq \frac{f(\varepsilon z + w) - z_0}{z_1 - z_0} \in H(\D^\times; \C \setminus \{ 0, 1 \}) $$
    has an essential singularity at zero. By \Cref{lem:great-picard-bounded}, we have that either $g$ or $1/g$ must be bounded in a neighborhood of zero. In the former case the singularity must therefore be removable, whereas in the latter case it must be a pole, yielding a contradiction.
\end{proof}

\begin{corollary} \label{cor:transcendental-every-value-inf}
    Every entire transcendental function assumes every value in $\C$ infinitely often, with at most one exception.
\end{corollary}

\begin{proof}
    Let $f(z) = \sum_{n=0}^\infty a_n z^n$ be transcendental and entire, and consider $g(z) \coloneqq f(1 / z)$. Then $g$ has an essential singularity at zero, thus by Picard's Great Theorem $g$ assumes all values in $\C$ on $B_1(0) \setminus \{ 0 \}$, except at most one. Thus $f$ does the same on $\C \setminus B_1(0)$.
\end{proof}

\begin{corollary}[Picard's Little Theorem] \label{thm:picards-little-theorem}
    Every non-constant entire function omits at most one value.
\end{corollary}

\begin{proof}
    A non-constant entire function $f$ is either a non-constant polynomial or transcendental. In the latter case, the claim follows from \Cref{cor:transcendental-every-value-inf}.
    
    Otherwise let $w \in \C$, then $f(z) - w$ has a complex root by the Fundamental Theorem of Algebra and hence $f$ assumes all values.
\end{proof}