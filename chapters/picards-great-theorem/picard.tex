\section{Picard's Great Theorem}
\label{sec:picards-great-theorem}

The following lemma is essential in showing that functions which omit two values on the punctured unit disk cannot have an essential singularity at $0$.

\begin{lemma} \label{lem:great-picard-bounded}
    Let $f \in H(\D^\times; \C \setminus \{ 0, 1 \})$. Then $f$ or $1/f$ is bounded in a punctured neighborhood of zero.
\end{lemma}

\begin{proof}
    For $n \in \N$ set $f_n(z) \coloneqq f(z / n) \in H(\D^\times; \C \setminus \{ 0, 1 \})$. By \Cref{lem:montel-sharpened} the sequence $(f_n)_{n \in \N}$ has a subsequence $(f_{n_k})_{k \in \N}$ that converges compactly to a $f \in H(\D^\times)$ or to $\infty$.

    Assume the former case. Then there is some $k_0 \in \N$ such that for all $k \geq k_0$ we have $ \Vert f_{n_k} - f \Vert_{\partial B_{1/2}(0)} < 1 $ and thus
    \begin{equation*}
        \Vert f \Vert_{\partial B_{1/2n_k}(0)} = \Vert f_{n_k} \Vert_{\partial B_{1/2}(0)} \leq \Vert f_{n_k} - f \Vert_{\partial B_{1/2}(0)} + \Vert f \Vert_{\partial B_{1/2}(0)} \leq 1 + \Vert f \Vert_{\partial B_{1/2}(0)},
    \end{equation*}
    whereby $f$ is bounded on $\partial B_{1/2n_k}(0)$.
    By the maximum modulus principle, $f$ must therefore be bounded on every annulus $1 / (2 n_{k + 1}) \leq \vert z \vert \leq 1 / (2 n_k)$, for $k \geq k_0$. Thus $f$ is bounded on
    $$ \bigcup_{k \geq k_0} \left\{ z \in \C : \frac{1}{2 n_{k+1}} \leq \vert z \vert \leq \frac{1}{2 n_k} \right\} = \cl{B_{1/2n_k}(0)} \setminus \{ 0 \}, $$
    which is a punctured neighborhood of zero.

    In the latter case, $(1 / f_{n_k})_{k \in \N}$ converges compactly to $0$ by \Cref{lem:compact-convergence-reciprocals}. Replacing $f_{n_k}$ with $1 / f_{n_k}$ and $f$ with $0$ in the above we likewise obtain that $1 / f$ is bounded in a punctured neighborhood of zero.
\end{proof}

\begin{theorem}[Picard's Great Theorem] \label{thm:picards-great-theorem}
    Let $G \subseteq \C$ be open, $w \in G$ and suppose $f \in H(G \setminus \{ w \})$ such that $f$ has an essential singularity at $w$. Then $f$ assumes all values in $\C$, with at most one exception\footnote{Such an exception is also referred to as a \emph{Picard Exceptional Value}.}, infinitely often in any punctured neighborhood of $w$.
\end{theorem}

\begin{proof}
    Assume towards a contradiction that $f$ takes on two distinct values $a, b \in \C$ only finitely often in some punctured neighborhood $W$ of $w$. Then $W$ contains a punctured disk of radius $t > 0$ around $w$, on which $f$ does not assume $a$ or $b$, and thus
    $$ g(z) \coloneqq \frac{f(t z + w) - a}{b - a} \in H(\D^\times; \C \setminus \{ 0, 1 \}), $$
    where $g$ has an essential singularity at zero. By \Cref{lem:great-picard-bounded}, we have that either $g$ or $1/g$ must be bounded in a punctured neighborhood of zero. By the classification of isolated singularities, in the former case the singularity must therefore be removable, whereas in the latter case it must be a pole, yielding a contradiction.
\end{proof}

The following corollary is also referred to as Picard's Great Theorem. Here an entire transcendental function is an entire function which is not a polynomial.

\begin{corollary} \label{cor:transcendental-every-value-inf}
    Every entire transcendental function assumes every value in $\C$ infinitely often, with at most one exception.
\end{corollary}

\begin{proof}
    Let $f(z) = \sum_{n=0}^\infty a_n z^n$ be transcendental and entire, and consider $g(z) \coloneqq f(1 / z) \in H(\C^\times)$. Then the Laurent series expansion of $g$ at $z = 0$ has infinite principal part, therefore $g$ has an essential singularity at zero. By Picard's Great Theorem $g$ assumes all values in $\C$ on $B_1(0) \setminus \{ 0 \}$ infinitely often, except at most one, and so $f$ does the same on $\C \setminus B_1(0)$.
\end{proof}

\begin{remark}
    Picard's Little Theorem is contained in Picard's Great Theorem:
    
    A non-constant $f \in H(\C)$ is either a non-constant polynomial or transcendental. In the latter case, $f$ omits at most one value by \Cref{cor:transcendental-every-value-inf}.
    
    In the former case let $w \in \C$, then $f(z) - w$ has a zero in $\C$ by the Fundamental Theorem of Algebra and hence $f$ assumes all values.
\end{remark}