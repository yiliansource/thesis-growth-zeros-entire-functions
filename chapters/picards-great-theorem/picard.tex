\section{Picard's Great Theorem}
\label{sec:picards-great-theorem}

\begin{theorem}[Montel] \label{thm:montel}
    Let $D \subset \C$ be open. Then a family $\mathscr{F} \subset H(D)$ is normal if and only if it is locally uniformly bounded.
\end{theorem}

\begin{theorem}[Hurwitz] \label{thm:hurwitz}
    \todo{TODO.}
\end{theorem}

\begin{lemma} \label{lem:montel-sharpened}
    Let $G \subseteq \C$ be a domain and set $\mathscr{F} \coloneqq H(G, \C \setminus \{ 0, 1 \})$. Furthermore, let $z_0 \in G$ and consider the subfamily $\mathscr{F}_1 \coloneqq \{ f \in \mathscr{F} : \vert f(z_0) \vert \leq 1 \}$. Then it holds that:
    \begin{enumerate}[i.]
        \item $\mathscr{F}_1$ is locally bounded in $G$.
        \item $\mathscr{F}$ is normal in $G$.
    \end{enumerate}
\end{lemma}

\begin{proof}
    \todo{TODO.}
\end{proof}

\begin{lemma} \label{lem:great-picard-bounded}
    Let $f \in H(\D^\times; \C \setminus \{ 0, 1 \})$. Then $f$ or $1/f$ is bounded in a punctured neighborhood of $0$.
\end{lemma}

\begin{proof}
    \todo{TODO.}
\end{proof}

\begin{theorem}[Picard's Great Theorem] \label{thm:picards-great-theorem}
    Let $G \subset \C$ be open, $w \in G$ and suppose $f \in H(G \setminus \{ w \})$ such that $f$ has an essential singularity at $w$. Then $f$ takes on all values in $\C$, with at most one exception, in any punctured neighborhood of $w$.
\end{theorem}

\begin{proof}
    Aiming for contradiction, assume that $f$ only takes on $z_0, z_1 \in \C$ finitely often in some punctured neighborhood $W$ of $w$. Then $W$ contains a punctured disk of radius $\varepsilon > 0$ around $w$, on which $f$ does not assume $z_0, z_1$. The function
    $$ g(z) \coloneqq \frac{f(\varepsilon z + w) - z_0}{z_1 - z_0} \in H(\D^\times; \C \setminus \{ 0, 1 \}) $$
    has an essential singularity at zero. By \Cref{lem:great-picard-bounded}, we have that either $g$ or $1/g$ must be bounded in a neighborhood of zero. In the former case the singularity must therefore be removable, whereas in the latter case it must be a pole, yielding a contradiction.
\end{proof}

\begin{corollary} \label{cor:transcendental-every-value-inf}
    Every entire transcendental function assumes every value in $\C$ infinitely often, with at most one exception.
\end{corollary}

\begin{proof}
    If $f(z) = \sum_{n \in \Z_{\geq 0}} a_n z^n$, then $f(z^{-1}) = \sum_{n \in \Z_{\leq 0}} a_{-n} z^{n}$, thus $f$ has an essential singularity at $\infty$ and Picard's Great Theorem (\ref{thm:picards-great-theorem}) concludes the claim.
\end{proof}

\begin{corollary}[Picard's Little Theorem] \label{thm:picards-little-theorem}
    Every nonconstant entire function omits at most one value.
\end{corollary}

\begin{proof}
    A non-constant entire function $f$ is either a non-constant polynomial or a transcendental function. In the latter case, the claim follows from \Cref{cor:transcendental-every-value-inf}.
    
    Otherwise let $w \in \C$, then $f(z) - w$ is a non-constant polynomial. By the Fundamental Theorem of Algebra it has a zero at some $z_0 \in \C$. Therefore $f(z_0) = w$ and $f$ attains all values.
\end{proof}