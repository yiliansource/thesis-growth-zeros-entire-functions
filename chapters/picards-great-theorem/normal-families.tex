\section{Normal families}
\label{sec:normal-families}

First, we recall a generalized form of locally uniform convergence:

\begin{definition}
    Let $G \subseteq \C$ be a domain, $f \in H(G)$ and $(f_n)_{n \in \N}$ a sequence in $H(G)$. We say that \emph{$f_n$ converges compactly in $G$ to $f$}, or \emph{$f_n$ converges compactly in $G$ to $\infty$} as $n \to \infty$, if for every compact set $K \subset G$
    \begin{equation}
        \lim_{n \to \infty} \sup_{z \in K} \vert f_n(z) - f(z) \vert = 0, \quad \textrm{or} \quad \lim_{n \to \infty} \inf_{z \in K} \vert f_n(z) \vert = \infty. \qedhere
    \end{equation}
\end{definition}

\begin{remark}
    One can define compact convergence generally for functions from a topological space $(X, \mathcal{T})$ into a metric space $(Y, d_Y)$. Compact convergence to $\infty$ can then be seen as compact convergence to a constant function with value $\infty$, where $d_Y(y, \infty)$ is appropriately defined for $y \in Y$. This is precisely the case when considering the chordal metric of the Riemann sphere, but we shall not elaborate further on this.
\end{remark}

\begin{definition}
    Let $f : \C \to \C$ and $a \in \C$, then the \emph{a-points of $f$} are defined as the zeros of $f(z) - a$, that is the set of all points $w \in \C$ with $f(w) = a$.
\end{definition}

A well-known theorem on compact convervence is:

\begin{theorem}[Hurwitz] \label{thm:hurwitz}
    Let $G \subseteq \C$ be a domain and $(f_n)_{n \in \N}$ a sequence in $H(G)$ that converges compactly to $f \in H(G)$. If for every $n \in \N$ the number of $a$-points of $f_n$ is bounded by some $m \in \N_0$, then either
    \begin{itemize}
        \item the number of $a$-points of $f$ are also bounded by $m$ or
        \item $f \equiv a$.
    \end{itemize}
\end{theorem}

As an immediate consequence we obtain that compact convergence is, in some ways, compatible with reciprocals:

\begin{lemma} \label{lem:compact-convergence-reciprocals}
Let $G \subseteq \C$ be a domain and $(f_n)_{n \in \N}$ a sequence in $H(G; \C \setminus \{ 0 \})$. If it converges compactly to some $f \in H(G)$, then it holds that either:
\begin{itemize}
    \item $0 \notin f(G)$ and $1 / f_n \to 1 / f$ compactly in $G$, or
    \item $f \equiv 0$ and $1 / f_n \to \infty$ compactly in $G$.
\end{itemize}
If it converges compactly to $\infty$, then $1 / f_n \to 0$ compactly in $G$.
\end{lemma}

\begin{proof}
    For the equivalence ``$f_n \to 0$ if and only if $1 / f_n \to \infty$'' it suffices to notice that for any compact $K \subset G$ we have
    \begin{equation*}
        \frac{1}{\sup_{z \in K} \vert f_n(z) \vert} = \inf_{z \in K} \left\vert \frac{1}{f_n(z)} \right\vert.
    \end{equation*}
    Since the $f_n$ vanish nowhere, by Hurwitz' Theorem we either have $0 \notin f(G)$ or $f \equiv 0$. In the latter case we have just shown that $1 / f_n \to \infty$ compactly.

    In the former case we have, again for any compact $K \subset G$, that $m \coloneqq \min_{z \in K} \vert f(z) \vert > 0$ and $\sup_{z \in K} \vert f_n(z) - f(z) \vert < \frac{m}{2}$ for sufficiently large $n \in \N$. Thus for all $z \in K$
    \begin{equation*}
        \frac{m}{2} > \vert f(z) - f_n(z) \vert \geq \vert f(z) \vert - \vert f_n(z) \vert \geq  m - \vert f_n(z) \vert
    \end{equation*}
    and $\vert f_n(z) \vert \geq \frac{m}{2}$. We obtain, for large $n \in \N$,
    \begin{equation*}
        \sup_{z \in K} \left\vert \frac{1}{f_n(z)} - \frac{1}{f(z)} \right\vert = \sup_{z \in K} \left\vert \frac{f(z) - f_n(z)}{f(z) f_n(z)} \right\vert \leq \sup_{z \in K} \left\vert f_n(z) - f(z) \right\vert \cdot \frac{1}{m} \cdot \frac{2}{m}
    \end{equation*}
    and therefore, after letting $n \to \infty$, that $1 / f_n \to 1 / f$ compactly in $G$.
\end{proof}

\begin{definition} \label{def:normal-family}
    Let $G \subseteq \C$ be a domain and $\F \subseteq H(G)$, then $\F$ is called:
    \begin{itemize}
        \item \emph{locally bounded}, if for every $w \in G$ there is a neighborhood $U$ of $w$ and a constant $C > 0$ such that $\vert f(z) \vert \leq C$ for all $f \in \F$ and $z \in U$.
        \item \emph{normal} in $G$, if every sequence in $\F$ has a subsequence which converges compactly in $G$ to some $f \in H(G)$. If the limit $\infty$ is also permitted it is instead called \emph{$\ast$-normal}.
    \end{itemize}
\end{definition}

The former two concepts are equivalent by the following well-known theorem:

\begin{theorem}[Montel] \label{thm:montel}
    Let $G \subseteq \C$ be a domain, then a family $\F \subseteq H(G)$ is normal if and only if it is locally bounded.
\end{theorem}

The following theorem can be interpreted as a sharpened version of Montel's Theorem and is sometimes referred to as the \emph{fundamental normality test}.

\begin{theorem} \label{lem:montel-sharpened}
    Let $G \subseteq \C$ be a domain, then any family $\F \subseteq H(G, \C \setminus \{ 0, 1 \})$ is $\ast$-normal in $G$.
\end{theorem}

\begin{proof}
    We present the proof in three steps:
    \begin{enumerate}
        \item Let $w \in G$, $c > 0$ and $\F_* \subseteq \F$ such that $\vert f(w) \vert \leq c$ for all $f \in \F_*$. We aim to show that there is an open disk at $w$ in which $\F_*$ is bounded. Select $t > 0$ such that $B_t(w) \subseteq G$. Let $f \in \F_*$, then $g(z) \coloneqq f(tz + w) \in H(\D)$. By the maximum modulus principle and Schottky's Theorem we obtain
        $$ \sup_{z \in B_{t/2}(w)} \vert f(z) \vert \leq \sup_{z \in B_{1/2}(0)} \vert g(z) \vert \leq \sup_{\vert z \vert = 1/2} \vert g(z) \vert \leq \psi(1/2, c) $$
        and $f$ is bounded on the disk $B_{t/2}(w)$. Since $f$ was arbitrary, $\F_*$ is bounded as well.

        \item Fix some $w_* \in G$ and set $\F_1 \coloneqq \{ f \in \F : \vert f(w_*) \vert \leq 1 \}$. We aim to show that $\F_1$ is locally bounded in $G$. Consider the set
        $$ U \coloneqq \{ w \in G : \F_1 \textrm{ is bounded in a neighborhood of } w \}, $$
        by (1) we have that $w_* \in U$. Note that $U$ is open in $G$, since if $\F_1$ is bounded in a disk $B_r(w)$, then for any $w' \in B_r(w)$ there is a disk $B_{r'}(w') \subseteq B_r(w)$, on which $\F_1$ is bounded as well.
        
        Assume towards a contradiction that $U \neq G$, then there exists some $w \in \partial U \cap G$ such that $\F_1$ is unbounded in every neighborhood of $w$.

        If there were some $c > 0$ such that $\vert f(w) \vert \leq c$ for all $f \in \F_1$, then by (1) there would exist an open disk centered at $w$ on which $\F_1$ would be bounded -- contradicting our assumption on $w$. Thus for every $n \in \N$, we can find some $f_n \in \F_1$ such that $\vert f_n(w) \vert \geq n$ and we obtain that $\lim_{n \to \infty} \vert f_n(w) \vert = \infty$.
        
        Set $g_n \coloneqq 1 / f_n \in \F$, then $\lim_{n \to \infty} \vert g_n(w) \vert = 0$. In particular, the family $(g_n)_{n \in \N}$ is bounded at $w$ by some constant, thus by (1) the family is bounded in some disk $B$ around $w$. By Montel's Theorem it is therefore normal in $B$, and there exists a subsequence $(g_{n_k})_{k \in \N}$ which converges compactly to a $g \in H(B)$. The functions $g_{n_k}$ have no zeros, but $g(w) = 0$; by Hurwitz's Theorem we therefore have $g \equiv 0$. Then for any $z \in B \cap U$ we have
        $$ \lim_{k \to \infty} \vert f_{n_k}(z) \vert = \lim_{k \to \infty} 1 / \vert g_{n_k}(z) \vert = \infty, $$
        contradicting the assumption that $\F_1$ is bounded in a neighborhood of such $z$. We thus have $U = G$, therefore $\F_1$ is locally bounded and by Montel's Theorem therefore normal.

        \item We can now conclude the proof. Let $(f_n)_{n \in \N}$ be a sequence in $\F$, we claim that it has some subsequence which converges compactly to some function in $H(G)$ or to $\infty$.
        
        If infinitely many $f_n$ lie in $\F_1$, then there is a subsequence $(f_{n_m})_{m \in \N}$ in $\F_1$, which by (2) has a subsequence $(f_{n_{m_k}})_{k \in \N}$ in $\F_1$ which convergences compactly in $G$ to some $f \in H(G)$. This sequence is also a subsequence of $(f_n)_{n \in \N}$, concluding the claim in this case.
        
        On the other hand, if there are only finitely many $f_n$ in $\F_1$, then infinitely many $1 / f_n$ lie in $\F_1$. As above, we thus obtain some subsequence in $\F_1$, say $(g_n)_{n \in \N}$, converging compactly in $G$ to some $g \in H(G)$. The sequence $(1 / g_n)_{n \in \N}$ is a subsequence of $(f_n)_{n \in \N}$, which -- by \Cref{lem:compact-convergence-reciprocals} -- converges compactly to $1 / g$ if $0 \notin g(G) $, and to $\infty$ otherwise. \qedhere
    \end{enumerate}
\end{proof}