\chapter{Picard's Great Theorem}
\label{ch:picards-great-theorem}

While the behaviour of holomorphic functions near their removable singularities and poles is fairly well-behaved, the same cannot be said for essential singularities.

Already an interesting result is given by the Casorati-Weierstrass Theorem: Let $G \subseteq \C$ be open, $w \in G$ and suppose $f \in H(G \setminus \{ w \})$ such that $f$ has an essential singularity at $w$. Then, for any punctured neighborhood $U \subseteq G$ of $w$, the set $f(U)$ is dense in $\C$.

Picard's Great Theorem will show that $f(U)$ is not only dense in $\C$, but that there is at most one value that is not taken on by $f$ infinitely often, on any such punctured neighborhood.

To obtain the proof we first study Bloch's Theorem, which estimatates the size of disks in the image of a holomorphic map. We then immediately obtain Picard's Little Theorem, which states that an entire function which omits two values is constant.

One can see this as a motivation to more closely study (non-entire) holomorphic functions that omit (at least) two values. For such functions, Schottky's Theorem will give an upper bound on their modulus. We use this to obtain a sharped version of Montel's Theorem, which asserts normality of any family of holomorphic functions, which omit two fixed values.

Restricting the domain in the above to be the punctured unit disk, we use the above to obtain that such functions cannot have an essential singularity at the origin, which precisely yields Picard's Great Theorem.

\section{Bloch's Theorem}
\label{sec:blochs-theorem}

If $G \subseteq \C$ is a domain and $f \in H(G)$ is non-constant, then $f(G)$ is a domain as well. In particular, $f(G)$ contains open disks of some, potentially very small, radius. Bloch's Theorem will show that for any $f \in H(D) \cap C(\cl{\D}; \C)$ satisfying $f'(0) = 1$, the set $f(\D)$ always contains a disk of fixed radius.

Note that $f(0)$ may not always be the center of such a disk; one must simply consider the sequence
\begin{equation*}
    f_n(z) \coloneqq \frac{1 - e^{- n z}}{n} \in H(\D) \cap C(\cl{\D}; \C), \quad n \in \N,
\end{equation*}
which satisfies $f_n(0) = 0$ and $f_n'(0) = 1$, but omits the value $1 / n$.

If $f$ is as before, then $f$ is an open mapping on $G$; we thus first observe a general criterion for the size of disks in image domains:

\begin{lemma} \label{lem:bloch-lemma-1}
    Let $G \subset \C$ be a bounded domain and $f \in C(\cl{G}; \C)$ such that $\restr{f}{G}$ is an open mapping. If there exists an $a \in G$ such that $s \coloneqq \min_{z \in \partial G} \vert f(z) - f(a) \vert > 0$, then $B_{s}(f(a)) \subseteq f(G)$.
\end{lemma}

\begin{proof}
    Since $G$ is bounded, $\cl{G}$ is compact and by continuity of $f$, so is $\cl{f(G)}$.
    The function $z \mapsto \vert z - f(a) \vert$ is continuous on the compact set $\partial f(G)$, hence it assumes its minimum $m$ at some $w_* \in \partial f(G)$. Choose a sequence $(z_n)_{n \in \N}$ in $G$ with $\lim_{n \to \infty} f(z_n) = w_*$ then, since $\cl{G}$ is compact, we can find a subsequence that converges to some $z_* \in \cl{G}$. By continuity of $f$ we have $f(z_*) = w_*$.

    If $z_* \in G$, since $\restr{f}{G}$ is open, the image of any open set in $G$ containing $z_*$ under $f$ is an open set in $f(G)$ containing $w_*$, which is impossible since $w_* \in \partial f(G)$.

    Therefore $z_* \in \partial G$ and we have
    $$ d(f(a), \partial f(G)) = m = \vert w_* - f(a) \vert = \vert f(z_*) - f(a) \vert \geq s, $$
    which implies $B_{s}(f(a)) \subseteq f(G)$.
\end{proof}

\begin{lemma} \label{lem:bloch-lemma-2}
    Fix $a \in \C, r > 0$ and let $B \coloneqq B_{r}(a)$. Suppose further that $f \in H(B) \cap C(\cl{B}; \C)$ such that $\Vert f' \Vert_B \leq 2 \vert f'(a) \vert$. Then $ B_{R}(f(a)) \subseteq f(B)$, where $ R \coloneqq (3 - 2 \sqrt{2}) r \vert f'(a) \vert $.
\end{lemma}

\begin{proof}
    We may assume $a = f(a) = 0$, otherwise we consider $f_1(z) \coloneqq f(z + a) - f(a)$. The function
    $$ \mapping{\alpha_f}{B & \to & \C,}{z & \mapsto & f(z) - f'(0) z,} $$
    satifies, for all $z \in B$,
    \begin{equation} \label{eq:bloch-lemma-2:estimate-1}
        \vert \alpha_f(z) \vert = \left\vert \int_{[0, z]} f'(\zeta) - f'(0) \diff \zeta \right\vert \leq \int_0^1 \vert f'(t z) - f'(0) \vert \vert z \vert \diff t. \tag{\textasteriskcentered}
    \end{equation}
    We wish to further estimate the integrand. Let $w \in B$, then Cauchy's integral formula gives
    \begin{align*}
        \vert f'(w) - f'(0) \vert &= \frac{1}{2\pi} \left\vert \int_{\partial B} \frac{f'(\zeta)}{\zeta - w} - \frac{f'(\zeta)}{\zeta} \diff \zeta \right\vert = \frac{1}{2\pi} \left\vert \int_{\partial B} \frac{w f'(\zeta)}{\zeta (\zeta - w)} \diff \zeta \right\vert \leq \\
        &\leq \frac{1}{2\pi} \int_{\partial B} \frac{\vert w \vert \Vert f' \Vert_B}{r(r - \vert w \vert)} \diff \zeta = \frac{\vert w \vert}{r - \vert w \vert} \Vert f' \Vert_B.
    \end{align*}
    Combining the above with \eqref{eq:bloch-lemma-2:estimate-1} and our estimate on $\Vert f' \Vert_B$ yields
    \begin{align*}
        \vert \alpha_f(z) \vert &\leq \int_0^1 \frac{\vert z t \vert \Vert f' \Vert_B}{r - \vert z t \vert} \vert z \vert \diff t \leq \frac{\vert z \vert^2}{r - \vert z \vert} \Vert f' \Vert_B \int_0^1 t \diff t \leq \frac{\vert z \vert^2}{r - \vert z \vert} \vert f'(0) \vert.
    \end{align*}
    Let $0 < \rho < r$, then for $\vert z \vert = \rho$ we have
    \begin{align*}
        \vert f'(0) \vert \rho - \vert f(z) \vert &\leq \vert \alpha_f(z) \vert \leq \frac{\rho^2}{r - \rho} \vert f'(0) \vert \\
        \Longleftrightarrow \quad \vert f(z) \vert &\geq \left( \rho - \frac{\rho^2}{r - \rho} \right) \vert f'(0) \vert.
    \end{align*}
    The function $\rho \mapsto \rho - \rho^2 / (r - \rho)$ assumes its maximum value at $\rho_* \coloneqq (1 - \sqrt{2} / 2)r \in (0, r)$, namely $(3 - 2 \sqrt{2}) r$. Therefore,
    \begin{equation*}
        \vert f(z) \vert \geq (3 - 2 \sqrt{2}) r \vert f'(0) \vert = R, \quad \textrm{for all} \quad \vert z \vert = \rho_*.
    \end{equation*}
    In particular, $\min_{z \in \partial B_{\rho_*}} \vert f(z) \vert \geq R > 0$, thus invoking \Cref{lem:bloch-lemma-1} with the domain $B_{\rho_*}(0)$ yields $B_{R}(0) \subseteq f(B_{\rho_*}(0)) \subseteq f(B)$.
\end{proof}

\begin{theorem} \label{thm:bloch-stronger}
    Let $f \in H(\D) \cap C(\cl{\D}; \C)$ be non-constant. Then there is a point $p \in \D$ and a constant $C_f > 0$ such that $B_{R}(f(p)) \subseteq f(\D)$, where $R \coloneqq (\frac{3}{2} - \sqrt{2}) C_f \geq (\frac{3}{2} - \sqrt{2}) \vert f'(0) \vert$.
\end{theorem}

\begin{proof}
    The function
    $$ \mapping{\alpha_f}{\cl{\D} & \to & \R}{z & \mapsto & \vert f'(z) \vert (1 - \vert z \vert)} $$
    is continuous and assumes its maximum $C_f > 0$ at some point $p \in \cl{\D}$. Note that $C_f \geq \vert f'(0) \vert$ and since $f$ is non-constant and $\restr{\alpha_f}{\partial \D} = 0$ we even have $p \in \D$.

    Set $t \coloneqq \frac{1}{2}(1 - \vert p \vert) > 0$, then we have $B_{t}(p) \subseteq \D$. Furthermore, for $z \in B_t(p)$, we have
    $$ 1 - \vert z \vert \geq 1 - \vert z - p \vert - \vert p \vert \geq 1 - t - \vert p \vert = t $$
    and since $\vert f'(z) \vert (1 - \vert z \vert) \leq C_f = 2 t \vert f'(p) \vert$, this implies $\vert f'(z) \vert \leq 2 \vert f'(p) \vert$ for all $z \in B_t(p)$. By \Cref{lem:bloch-lemma-2}, we get $ B_{R}(f(p)) \subseteq f(\D) $, where $R \coloneqq (3 - 2 \sqrt{2}) t \vert f'(p) \vert = ({\textstyle \frac{3}{2}} - \sqrt{2}) C_f$,
    which establishes the assertion.
\end{proof}

We now immediately obtain:    

\begin{theorem}[Bloch] \label{thm:bloch}
    Let $f \in H(\D) \cap C(\cl{\D}; \C)$ and assume $f'(0) = 1$. Then $f(\D)$ contains a disk of radius $\frac{3}{2} - \sqrt{2}$.
\end{theorem}

In the following we will denote by $\beta > 0$ any constant less than or equal to the radius in Bloch's Theorem, for example $\beta = \frac{1}{12} < \frac{3}{2} - \sqrt{2}$.

\begin{corollary} \label{cor:bloch-domain}
    Let $G \subseteq \C$ be a domain and $f \in H(G)$ with $f'(c) \neq 0$ for some $c \in G$. Then $f(G)$ contains a disk of every radius $\beta s \vert f'(c) \vert$, where $0 < s < d(c, \partial G)$.
\end{corollary}

\begin{proof}
    We may assume $c = 0$, otherwise we consider $f_1(z) \coloneqq f(z+c)$. Let $0 < s < d(c, \partial G)$, then $f$ is analytic on $\cl{B_s(0)} \subseteq G$, thus we have $g(z) \coloneqq f(sz) / sf'(0) \in H(\D) \cap C(\cl{\D}; \C)$. Since $g'(0) = 1$, Bloch's Theorem yields a disk $B$ of radius $\beta$ with $B \subseteq g(\D)$. Then $D \coloneqq s \vert f'(0) \vert B$ is a disk of radius $\beta s \vert f'(0) \vert$ and we have
    \begin{equation*}
        D = s \vert f'(0) \vert B \subseteq s \vert f'(0) \vert g(\D) = f(B_s(0)) \subseteq f(G).
    \end{equation*}
\end{proof}

\begin{corollary} \label{cor:bloch-entire}
    If $f \in H(\C)$ is non-constant, then $f(\C)$ contains a disk of every radius.
\end{corollary}
\section{Schottky's Theorem}
\label{sec:schottkys-theorem}

\begin{theorem}[Schottky] \label{thm:schottky}
    Let $f$ be analytic on $\cl{\D}$ such that it does not attain $0$ or $1$ as values in $\cl{\D}$. Then there is a function $\psi(\omega, \theta) : \C \times (0, 1) \to \R_{\geq 0}$ such that if $\vert f(0) \vert \leq \omega$, then
    \begin{equation}
        \vert f(z) \vert \leq \psi(\omega, \theta), \quad \vert z \vert \leq \theta.
    \end{equation}
\end{theorem}

\begin{proof}
    \todo{Ref: Classical Topics p237.}
\end{proof}
\section{Normal Families}
\label{sec:normal-families}

First, we recall a generalized form of locally uniform convergence:

\begin{definition}
    Let $G \subseteq \C$ be a domain, $f \in C(G, \C)$ and $(f_n)_{n \in \N}$ a sequence in $C(G, \C)$. We say that \emph{$f_n$ converges compactly in $G$ to $f$}, or \emph{$f_n$ converges compactly in $G$ to $\infty$} as $n \to \infty$, if for every compact set $K \subset G$
    \begin{equation}
        \lim_{n \to \infty} \sup_{z \in K} \vert f_n(z) - f(z) \vert = 0, \quad \textrm{or} \quad \lim_{n \to \infty} \inf_{z \in K} \vert f_n(z) \vert = \infty,
    \end{equation}
    where the supremum and infimum over empty sets are defined as $0$ and $\infty$, respectively.
\end{definition}

% \begin{remark}
%     One can define compact convergence generally for functions from a topological space $(X, \mathcal{T})$ into a metric space $(Y, d_Y)$. Compact convergence to $\infty$ can then be seen as compact convergence to a constant function with value $\infty$, where $d_Y(y, \infty)$ is appropriately defined for $y \in Y$. This is precisely the case when considering the chordal metric of the Riemann sphere, but we shall not elaborate further on this.
% \end{remark}

\begin{definition}
    Let $f : \C \to \C$ and $a \in \C$, then the \emph{a-points of $f$} are defined as the zeros of $f(z) - a$, that is the set of all points $w \in \C$ with $f(w) = a$.
\end{definition}

A well-known theorem on compact convergence is:

\begin{samepage}    
\begin{theorem}[Hurwitz] \label{thm:hurwitz}
    Let $G \subseteq \C$ be a domain, $a \in \C$ and $(f_n)_{n \in \N}$ a sequence in $H(G)$ that converges compactly to $f \in H(G)$. If for every $n \in \N$ the number of $a$-points of $f_n$ is bounded by some $m \in \N_0$, then either
    \begin{itemize}
        \item the number of $a$-points of $f$ are also bounded by $m$ or
        \item $f \equiv a$.
    \end{itemize}
\end{theorem}
\end{samepage}

As an immediate consequence we obtain that compact convergence is, in some ways, compatible with reciprocals:

\begin{lemma} \label{lem:compact-convergence-reciprocals}
Let $G \subseteq \C$ be a domain and $(f_n)_{n \in \N}$ a sequence in $H(G; \C \setminus \{ 0 \})$. If $(f_n)_{n \in \N}$ converges compactly to some $f \in H(G)$, then it holds that either:
\begin{itemize}
    \item $0 \notin f(G)$ and $1 / f_n \to 1 / f$ compactly in $G$, or
    \item $f \equiv 0$ and $1 / f_n \to \infty$ compactly in $G$.
\end{itemize}
If, on the other hand, $(f_n)_{n \in \N}$ converges compactly to $\infty$, then $1 / f_n \to 0$ compactly in $G$.
\end{lemma}

\begin{proof}
    For the equivalence ``$f_n \to 0$ if and only if $1 / f_n \to \infty$'' it suffices to notice that for any compact $K \subset G$ we have
    \begin{equation*}
        \frac{1}{\sup_{z \in K} \vert f_n(z) \vert} = \inf_{z \in K} \left\vert \frac{1}{f_n(z)} \right\vert.
    \end{equation*}
    Since the functions $(f_n)_{n \in \N}$ vanish nowhere, by Hurwitz' Theorem we either have $0 \notin f(G)$ or $f \equiv 0$. In the latter case we have just shown that $1 / f_n \to \infty$ compactly.

    In the former case, we have, again for any compact $K \subset G$, that $m \coloneqq \min_{z \in K} \vert f(z) \vert > 0$ and $\sup_{z \in K} \vert f_n(z) - f(z) \vert < \frac{m}{2}$ for all sufficiently large $n \in \N$. Thus, for all $z \in K$
    \begin{equation*}
        \frac{m}{2} > \vert f(z) - f_n(z) \vert \geq \vert f(z) \vert - \vert f_n(z) \vert \geq  m - \vert f_n(z) \vert
    \end{equation*}
    and $\vert f_n(z) \vert \geq \frac{m}{2}$. We obtain, for large $n \in \N$,
    \begin{equation*}
        \sup_{z \in K} \left\vert \frac{1}{f_n(z)} - \frac{1}{f(z)} \right\vert = \sup_{z \in K} \left\vert \frac{f(z) - f_n(z)}{f(z) f_n(z)} \right\vert \leq \sup_{z \in K} \left\vert f_n(z) - f(z) \right\vert \cdot \frac{1}{m} \cdot \frac{2}{m}
    \end{equation*}
    and therefore, after letting $n \to \infty$, that $1 / f_n \to 1 / f$ compactly in $G$.
\end{proof}

\begin{definition} \label{def:normal-family}
    Let $G \subseteq \C$ be a domain and $\F \subseteq H(G)$, then $\F$ is called:
    \begin{itemize}
        \item \emph{locally bounded}, if for every $w \in G$ there is a neighborhood $U$ of $w$ and a constant $C > 0$ such that $\vert f(z) \vert \leq C$ for all $f \in \F$ and $z \in U$.
        \item \emph{normal} in $G$, if every sequence in $\F$ has a subsequence which converges compactly in $G$ to some $f \in H(G)$. If the limit $\infty$ is also permitted, it is instead called \emph{$\ast$-normal}.
    \end{itemize}
\end{definition}

These two concepts are equivalent by the following well-known theorem:

\begin{theorem}[Montel] \label{thm:montel}
    Let $G \subseteq \C$ be a domain, then a family $\F \subseteq H(G)$ is normal if and only if it is locally bounded.
\end{theorem}

The following theorem can be interpreted as a sharpened version of the above:

\begin{theorem}[Fundamental Normality Test] \label{lem:montel-sharpened}
    Let $G \subseteq \C$ be a domain, then any family $\F \subseteq H(G, \C \setminus \{ 0, 1 \})$ is $\ast$-normal in $G$.
\end{theorem}

\begin{proof}
    We present the proof in three steps:
    \begin{enumerate}
        \item Let $w \in G$, $c > 0$ and $\F_* \subseteq \F$ such that $\vert f(w) \vert \leq c$ for all $f \in \F_*$. We aim to show that there is an open disk around $w$ in which $\F_*$ is bounded. Select $t > 0$ such that $B_t(w) \subseteq G$. Let $f \in \F_*$, then $g(z) \coloneqq f(tz + w) \in H(\D)$. By the maximum modulus principle and $\psi$ from Schottky's Theorem we obtain
        $$ \sup_{z \in B_{t/2}(w)} \vert f(z) \vert = \sup_{z \in B_{1/2}(0)} \vert g(z) \vert \leq \sup_{\vert z \vert = 1/2} \vert g(z) \vert \leq \psi(1/2, c) $$
        and $f$ is bounded on the disk $B_{t/2}(w)$. Since $f$ was arbitrary, $\F_*$ is bounded as well.

        \item Fix some $w_* \in G$ and set $\F_1 \coloneqq \{ f \in \F : \vert f(w_*) \vert \leq 1 \}$. We aim to show that $\F_1$ is locally bounded in $G$. Consider the set
        $$ U \coloneqq \{ w \in G : \F_1 \textrm{ is bounded in a neighborhood of } w \}, $$
        by 1. we have that $w_* \in U$. Note that $U$ is open in $G$, since if $\F_1$ is bounded in a disk $B_r(w)$, then for any $w' \in B_r(w)$ there is a disk $B_{r'}(w') \subseteq B_r(w)$, on which $\F_1$ is bounded as well.
        
        Assume towards a contradiction that $U \neq G$, then there exists some $w \in \partial U \cap G$ such that $\F_1$ is unbounded in every neighborhood of $w$.

        If there were some $c > 0$ such that $\vert f(w) \vert \leq c$ for all $f \in \F_1$, then by 1. there would exist an open disk centered at $w$ on which $\F_1$ would be bounded -- contradicting our assumption on $w$. Thus, for every $n \in \N$ we can find some $f_n \in \F_1$ such that $\vert f_n(w) \vert \geq n$ and we obtain that $\lim_{n \to \infty} \vert f_n(w) \vert = \infty$.
        
        Since the functions $(f_n)_{n \in \N}$ do not assume $0$ or $1$ as values, the functions $g_n \coloneqq 1 / f_n, n \in \N$ are well-defined, do not assume $0$ or $1$ as values, and satisfy
        $$ \lim_{n \to \infty} \vert g_n(w) \vert = 0. $$
        In particular, the family $(g_n)_{n \in \N}$ is bounded at $w$ by some constant, thus by 1. the family is bounded in some disk $B$ around $w$. By Montel's Theorem it is therefore normal in $B$, and there exists a subsequence $(g_{n_k})_{k \in \N}$ which converges compactly to some $g \in H(B)$. The functions $g_{n_k}$ have no zeros, but $g(w) = 0$; by Hurwitz's Theorem we therefore have $g \equiv 0$. Then for any $z \in B \cap U$ we have
        $$ \lim_{k \to \infty} \vert f_{n_k}(z) \vert = \lim_{k \to \infty} 1 / \vert g_{n_k}(z) \vert = \infty, $$
        contradicting the assumption that $\F_1$ is bounded in a neighborhood of such $z$. We thus have $U = G$, therefore $\F_1$ is locally bounded and by Montel's Theorem therefore normal.

        \item We can now conclude the proof. Let $(f_n)_{n \in \N}$ be a sequence in $\F$, we claim that it has some subsequence which converges compactly to some function in $H(G)$ or to $\infty$.
        
        If infinitely many $f_n$ lie in $\F_1$, then there is a subsequence $(f_{n_m})_{m \in \N}$ in $\F_1$, which by 2. has a subsequence $(f_{n_{m_k}})_{k \in \N}$ in $\F_1$ which converges compactly in $G$ to some $f \in H(G)$. This sequence is also a subsequence of $(f_n)_{n \in \N}$, concluding the claim in this case.
        
        On the other hand, if there are only finitely many $f_n$ in $\F_1$, then infinitely many $1 / f_n$ lie in $\F_1$. As above, we thus obtain some subsequence in $\F_1$, say $(g_n)_{n \in \N}$, converging compactly in $G$ to some $g \in H(G)$. The sequence $(1 / g_n)_{n \in \N}$ is a subsequence of $(f_n)_{n \in \N}$, which -- by \Cref{lem:compact-convergence-reciprocals} -- converges compactly to $1 / g$ if $0 \notin g(G) $, and to $\infty$ otherwise. \qedhere
    \end{enumerate}
\end{proof}
\section{Picard's Great Theorem}
\label{sec:picards-great-theorem}

\begin{theorem}[Picard's Great Theorem] \label{thm:picards-great-theorem}
    Let $c \in \C$ be an isolated singularity of $f$. Then, in every punctured neighbourhood of $c$, $f$ assumes every complex number as a value infinitely many times, with at most one exception.
\end{theorem}

\begin{proof}
    \todo{Ref: Classical Topics p240.}
\end{proof}

\begin{corollary} \label{cor:transcendental-every-value-inf}
    Let $f$ be a transcendental entire function. Then $f$ attains every value in $\C$ infinitely often, with at most one exception.
\end{corollary}

\begin{proof}
    If $f(z) = \sum_{n \in \Z_{\geq 0}} a_n z^n$, then $f(z^{-1}) = \sum_{n \in \Z_{\leq 0}} a_{-n} z^{n}$, thus $f$ has an essential singularity at $\infty$ and Picard's Great Theorem (\ref{thm:picards-great-theorem}) concludes the claim.
\end{proof}

\begin{corollary}[Picard's Little Theorem] \label{thm:picards-little-theorem}
    Every nonconstant entire function omits at most one value.
\end{corollary}

\begin{proof}
    A non-constant entire function $f$ is either a non-constant polynomial or a transcendental function. In the latter case, the claim follows from \Cref{cor:transcendental-every-value-inf}.
    
    Otherwise let $w \in \C$, then $f(z) - w$ is a non-constant polynomial. By the Fundamental Theorem of Algebra it has a zero at some $z_0 \in \C$. Therefore $f(z_0) = w$ and $f$ attains all values.
\end{proof}