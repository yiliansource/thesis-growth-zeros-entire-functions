\chapter{Picard's Great Theorem}
\label{ch:picards-great-theorem}

An isolated singularity of a holomorphic function which is not removable nor a pole, is called an \emph{essential singularity}. While the behaviour of holomorphic functions near their removable singularities and poles is fairly well-behaved, the same cannot be said for their essential singularities.

An interesting result is already given by the Casorati-Weierstrass Theorem: Let $G \subseteq \C$ be open, $w \in G$ and suppose $f \in H(G \setminus \{ w \})$ such that $f$ has an essential singularity at $w$. Then, for any punctured neighborhood $U \subseteq G$ of $w$, the set $f(U)$ is dense in $\C$.

Picard's Great Theorem will show that $f(U)$ is not only dense in $\C$, but there is at most one value which is not taken on by $f$ infinitely often, on any such punctured neighborhood.

We will reach the proof the same way as in Remmert \cite{remmert-function-theory}. We first study Bloch's Theorem, which estimates the size of disks in the image of a holomorphic map. As a corollary of the proof we obtain Picard's Little Theorem, which states that an entire function which omits two values is constant.

One can see this as a motivation to intently study (non-entire) holomorphic functions that omit (at least) two values. For such functions, Schottky's Theorem will give an upper bound on their modulus. We use this to obtain the Fundamental Normality Test, which asserts normality of any family of holomorphic functions omitting two fixed values.

Restricting the domain in the above to be the punctured unit disk, we obtain that such functions cannot have an essential singularity at the origin, which will imply Picard's Great Theorem.

\section{Bloch's Theorem}
\label{sec:blochs-theorem}

\begin{lemma} \label{lem:bloch-lemma-1}
    Let $G \subset \C$ be a bounded domain, $f : \cl{G} \to \C$ continuous and $\restr{f}{G} : G \to \C$ open. If there exists $a \in G$ such that $s \coloneqq \min_{z \in \partial G} \vert f(z) - f(a) \vert > 0$, then $B_{s}(f(a)) \subseteq f(G)$.
\end{lemma}

\begin{proof}
    The function $z \mapsto \vert z - f(a) \vert$ is continuous on the compact set $\partial f(G)$, hence it attains its minimum $m$ at some $w_* \in \partial f(G)$. Let $(z_n)_{n \in \N}$ be a sequence in $G$, such that $\lim_{n \to \infty} f(z_n) = w_*$. Since $\cl{G}$ is compact, we can find a subsequence that converges to some $z_* \in \cl{G}$ and by continuity of $f$ we have $f(z_*) = w_*$.

    Assuming $z_* \in G$, since $\restr{f}{G}$ is open, the image of any open set in $G$ containing $z_*$ under $f$ must be an open set in $f(G)$ containing $w_*$, which is impossible since $w_* \in \partial G$.

    Therefore $z_* \in \partial G$ and we have $m = \vert w_* - f(a) \vert = \vert f(z_*) - f(a) \vert \geq s$, from which it follows that $B_{s}(f(a)) \subseteq f(G)$.
\end{proof}

\begin{lemma} \label{lem:bloch-lemma-2}
    Let $a \in \C, r > 0$ and $B \coloneqq B_{r}(a)$. Suppose further that $f \in H(\cl{B})$ such that $\Vert f' \Vert_B \leq 2 \vert f'(a) \vert$. Then $ B_{R}(f(a)) \subseteq f(B)$, where $ R \coloneqq (3 - 2 \sqrt{2}) r \vert f'(a) \vert $.
\end{lemma}

\begin{proof}
    Without loss of generality we can assume $a = f(a) = 0$. Consider the function
    $$ \mapping{\alpha_f}{B & \to & \C,}{z & \mapsto & f(z) - f'(0) z,} $$
    which satifies, for the line $\gamma$ connecting $0$ to $z \in B$,
    \begin{equation} \label{eq:bloch-lemma-2:estimate-1}
        \vert \alpha_f(z) \vert = \left\vert \int_\gamma f'(\zeta) - f'(0) \diff \zeta \right\vert \leq \int_0^1 \vert f'(t z) - f'(0) \vert \vert z \vert \diff t. \tag{\textasteriskcentered}
    \end{equation}
    Let $w \in B$, then Cauchy's integral formula gives
    \begin{align*}
        \vert f'(w) - f'(0) \vert &= \frac{1}{2\pi} \left\vert \int_{\partial B} \frac{f'(\zeta)}{\zeta - w} - \frac{f'(\zeta)}{\zeta} \diff \zeta \right\vert = \frac{1}{2\pi} \left\vert \int_{\partial B} \frac{w f'(\zeta)}{\zeta (\zeta - w)} \diff \zeta \right\vert \leq \\
        &\leq \frac{1}{2\pi} \int_{\partial B} \frac{\vert w \vert \Vert f' \Vert_B}{r(r - \vert w \vert)} \diff \zeta = \frac{\vert w \vert}{r - \vert w \vert} \Vert f' \Vert_B.
    \end{align*}
    Combining the above with \eqref{eq:bloch-lemma-2:estimate-1} and our estimate on $\Vert f' \Vert_B$ yields
    \begin{align*}
        \vert \alpha_f(z) \vert &\leq \int_0^1 \frac{\vert z t \vert \Vert f' \Vert_B}{r - \vert z t \vert} \vert z \vert \diff t \leq \frac{\vert z \vert^2}{r - \vert z \vert} \Vert f' \Vert_B \int_0^1 t \diff t \leq \frac{\vert z \vert^2}{r - \vert z \vert} \vert f'(0) \vert.
    \end{align*}
    Let $0 < \rho < r$, then for $\vert z \vert = \rho$ we have
    \begin{align*}
        \vert f'(0) \vert \rho - \vert f(z) \vert &\leq \vert \alpha_f(z) \vert \leq \frac{\rho^2}{r - \rho} \vert f'(0) \vert \\
        \Longleftrightarrow \quad \vert f(z) \vert &\geq \left( \rho - \frac{\rho^2}{r - \rho} \right) \vert f'(0) \vert.
    \end{align*}
    Considering $\rho - \rho^2 / (r - \rho)$ as a function of $\rho$, it attains its maximum value, $(3 - 2 \sqrt{2}) r$, at $\rho_* \coloneqq (1 - \sqrt{2} / 2)r \in (0, r)$. Therefore,
    \begin{equation*}
        \vert f(z) \vert \geq (3 - 2 \sqrt{2}) r \vert f'(0) \vert = R, \quad \textrm{for all} \quad \vert z \vert = \rho_*.
    \end{equation*}
    Invoking \Cref{lem:bloch-lemma-1} with $G \coloneqq B_{\rho_*}(0)$ thus yields $B_{R}(0) \subseteq f(G) \subseteq f(B)$.
\end{proof}

\begin{theorem} \label{thm:bloch-stronger}
    Let $f \in H(\cl{\D})$ be non-constant. Then there is a point $p \in \D$ and a constant $C_f > 0$ such that $B_{R}(f(p)) \subseteq f(\D)$, where $R \coloneqq (\frac{3}{2} - \sqrt{2}) C_f \geq (\frac{3}{2} - \sqrt{2}) \vert f'(0) \vert$.
\end{theorem}

\begin{proof}
    The function
    $$ \mapping{\alpha_f}{\cl{\D} & \to & \R}{z & \mapsto & \vert f'(z) \vert (1 - \vert z \vert)} $$
    is continuous and attains its maximum $C_f > 0$ at some point $p \in \D$.
    
    Set $t \coloneqq \frac{1}{2}(1 - \vert p \vert) > 0$, then we have $B_{t}(p) \subseteq \D$ and $1 - \vert z \vert \geq t$ for $z \in B_{t}(p)$. Since $\vert f'(z) \vert (1 - \vert z \vert) \leq C_f = 2 t \vert f'(p) \vert$, this implies $\vert f'(z) \vert \leq 2 \vert f'(p) \vert$. By \Cref{lem:bloch-lemma-2}, we obtain
    $ B_{R}(f(p)) \subseteq f(\D) $, where 
    $$ R \coloneqq (3 - 2 \sqrt{2}) t \vert f'(p) \vert = ({\textstyle \frac{3}{2}} - \sqrt{2}) \vert f'(p) \vert (1 - \vert p \vert) > {\textstyle \frac{1}{12}} \vert f'(0) \vert, $$
    establishes the assertion.
\end{proof}

We immediately have:    

\begin{theorem}[Bloch] \label{thm:bloch}
    Let $f \in H(\cl{\D})$ and assume that $f'(0) = 1$. Then $f(\D)$ contains a disk of radius $\frac{3}{2} - \sqrt{2}$.
\end{theorem}

In the following we will denote by $\beta > 0$ any constant satisfying Bloch's Theorem, for example $\beta = \frac{1}{12} < \frac{3}{2} - \sqrt{2}$.

\begin{corollary} \label{cor:bloch-domain}
    Let $G \subset \C$ be a domain and $f \in H(G)$ with $f'(c) \neq 0$ for some $c \in G$. Then $f(G)$ contains a disk of every radius $\beta s \vert f'(c) \vert$, where $0 < s < d(c, \partial G)$.
\end{corollary}

\begin{proof}
    Without loss of generality we may assume $c = 0$. Since $f$ is analytic on $\cl{B_s(0)} \subseteq G$, we have $g(z) \coloneqq f(sz) / sf'(0) \in H(\cl{\D})$. Since $g'(0) = 1$, Bloch's Theorem (\ref{thm:bloch}) yields a disk $B$ of radius $\beta$ such that $B \subseteq g(\D)$. It follows that
    \begin{equation*}
        s \vert f'(0) \vert B \subseteq s \vert f'(0) \vert g(\D) = f(B_s(0)) \subseteq f(G).
    \end{equation*}
\end{proof}

\begin{corollary} \label{cor:bloch-entire}
    If $f \in H(\C)$ is non-constant, then $f(\C)$ contains a disk of every radius.
\end{corollary}
\section{Schottky's Theorem}
\label{sec:schottkys-theorem}

Holomorphic functions which omit the values $0$ and $1$ have a universal estimate on the growth of their modulus, which will be given by Schottky's Theorem.

For a domain $G \subseteq \C$ and a set $E \subseteq \C$ we define $H(G; E)$ as the set\footnote{Note that, unlike $H(G)$, the set $H(G; E)$ is usually not a linear space.} of all $f \in H(G)$ such that $f(G) \subseteq E$.

\begin{lemma} \label{lem:schottky-1}
    It holds that:
    \begin{enumerate}[i.]
        \item If $a, b \in \R$ with $\cos \pi a = \cos \pi b$, then $b = \pm a + 2n$ for some $n \in \Z$.
        \item For every $w \in \C$ there exists $v \in \C$ such that $\cos \pi v = w$ and $\vert v \vert \leq 1 + \vert w \vert$.
    \end{enumerate}
\end{lemma}

\begin{proof}
    For the first part it suffices to note that
    $$ 0 = \cos \pi a - \cos \pi b = \textstyle -2 \sin \frac{\pi}{2} ( a + b ) \sin \frac{\pi}{2} ( a - b ). $$
    Since the complex cosine function is surjective and $2 \pi$-periodic, we can choose $v = a + i b$ with $w = \cos \pi v$ and $\vert a \vert \leq 1$. Now we have
    \begin{align*}
        \vert w \vert^2 &= \vert \cos (\pi a + i \pi b) \vert^2 = \vert \cos \pi a \cos i \pi b + \sin \pi a \sin i \pi b \vert^2 = \\
        &= \vert \cos \pi a \cosh \pi b - i \sin \pi a \sinh \pi b \vert^2 = \\
        &= \cos^2 \pi a \cosh^2 \pi b + \sin^2 \pi a \sinh^2 \pi b = \\
        &= \cos^2 \pi a + \cos^2 \pi a \sinh^2 \pi b + \sin^2 \pi a \sinh^2 \pi b = \\
        &= \cos^2 \pi a + \sinh^2 \pi b \geq \sinh^2 \pi b \geq \pi^2 b^2,
    \end{align*}
    where the last inequality holds since $\sinh^2 x \geq x^2$ for $x \in \R$. We conclude
    \begin{equation*}
        \vert v \vert = \sqrt{a^2 + b^2} \leq \sqrt{1 + \vert w \vert^2 / \pi^2} \leq 1 + \vert w \vert.
    \end{equation*}
\end{proof}

We recall the following result: Let $G \subseteq \C$ be a simply connected domain and $f \in H(G)$, such that $f$ vanishes nowhere on $G$. There is a function\footnote{$g$ is also sometimes called a \emph{logarithm} of $f$.} $g \in H(G)$ such that $f = e^g$, which, for $n \in \N$, can also be used to obtain $n$-th roots of such functions by defining $\sqrt[n]{f} \coloneqq e^{g/n}$.

\begin{lemma} \label{lem:schottky-2}
    Let $G \subseteq \C$ be a simply connected domain and $f \in H(G; \C \setminus \{ -1, 1 \})$, then there exists $F \in H(G)$ such that $f = \cos F$.
\end{lemma}

\begin{proof}
    Since $1 - f^2$ vanishes nowhere in $G$, it has a square root $g \in H(G)$, therefore
    \begin{equation*}
        1 = f^2 + g^2 = (f + ig)(f - ig).
    \end{equation*}
    Thus $f + ig$ vanishes nowhere and there exists an $F \in H(G)$ with $f + ig = e^{iF}$. By the above we also have $f - ig = e^{-iF}$ and therefore
    \begin{equation*}
        f = {\textstyle \frac{1}{2}} (e^{iF} + e^{-iF}) = \cos F. \qedhere
    \end{equation*}
\end{proof}

\begin{lemma} \label{lem:schottky-3}
    Let $G \subseteq \C$ be a simply connected domain and $f \in H(G; \C \setminus \{ 0, 1 \})$. Then there exists $g \in H(G)$ such that:
    \begin{enumerate}[i.]
        \item $f = \frac{1}{2} ( 1 + \cos \pi (\cos \pi g))$.
        \item $\vert g(0) \vert \leq 3 + 2 \vert f(0) \vert$.
        \item $g(G)$ contains no disk of radius $1$.
        \item If $\D \subseteq G$, then $\vert g(z) \vert \leq \vert g(0) \vert + \frac{\theta}{\beta (1 - \theta)}$ for all $\vert z \vert \leq \theta$, where $0 < \theta < 1$.
    \end{enumerate}
\end{lemma}

\begin{proof}
    By \Cref{lem:schottky-2}, there exists $\widetilde{F} \in H(G)$ such that $2f - 1 = \cos \pi \widetilde{F}$ and, by \Cref{lem:schottky-1}, there is $b \in \C$ with $\cos \pi b = 2f(0) - 1$ and $\vert b \vert \leq 1 + \vert 2 f(0) - 1 \vert \leq 2 + 2 \vert f(0) \vert$. Furthermore, since $\cos \pi b = \cos \pi \widetilde{F}(0)$, we have $b = \pm \widetilde{F}(0) + 2k$ for some $k \in \Z$. Then $F \coloneqq \pm \widetilde{F} + 2k \in H(G)$ satisfies $F(0) = b$ and $2f - 1 = \cos \pi F$.
    
    Since $F$ must omit all integers, there exists $\widetilde{g} \in H(G)$ such that $F = \cos \pi \widetilde{g}$. Similarly, there is $a \in \C$ such that $\cos \pi a = b$ and $\vert a \vert \leq 1 + \vert b \vert \leq 3 + 2 \vert f(0) \vert$. Like in the above, since $\cos \pi a = \cos \pi \widetilde{g}(0)$, we have $a = \pm \widetilde{g}(0) + 2\ell$ for some $\ell \in \Z$, thus $g \coloneqq \pm \widetilde{g} + 2\ell \in H(G)$ satisfies $g(0) = a$ and $F = \cos \pi g$. Together we obtain
    $$ \textstyle f = \frac{1}{2} (1 + \cos \pi (\cos \pi g)), \quad \textrm{and} \quad \vert g(0) \vert = \vert a \vert \leq 3 + 2 \vert f(0) \vert $$
    and thus have shown i. and ii.

    To show iii. we consider the set
    $$ A \coloneqq \{ m \pm i \pi^{-1} \log (n + \sqrt{n^2 - 1}) : m \in \Z, n \in \N \setminus \{ 0 \} \}, $$
    the points of which can be considered the vertices of a rectangular grid in $\C$, the cells of which are of fixed width and variable height. The width of such a rectangular cell is $1$, and since
    \begin{align*}
        \log ((n+1) &+ \sqrt{(n+1)^2 - 1}) - \log (n + \sqrt{n^2 - 1}) = \\
        &= \log \frac{1 + \frac{1}{n} + \sqrt{1 + \frac{2}{n}}}{1 + \sqrt{1 - \frac{1}{n^2}}} \leq \log (2 + \sqrt{3}) < \pi,
    \end{align*}
    their height is bounded above by some $C < 1$. Therefore, for all $z \in \C$ there is $w_z \in A$ such that $\vert \Re z - \Re w_z \vert \leq \frac{1}{2}$ and $\vert \Im z - \Im w_z \vert \leq \frac{C}{2}$. Thus, we have
    $$ \vert z - w_z \vert \leq \vert \Re z - \Re w_z \vert + \vert \Im z - \Im w_z \vert \leq \frac{1}{2} + \frac{C}{2} < 1. $$
    If we can show that $g(G) \cap A = \emptyset$, then $g(G)$ cannot contain a disk of radius $1$. Let $a = p + i \pi^{-1} \log(q + \sqrt{q^2 - 1}) \in A$, then
    \begin{align*}
        \cos \pi a &= {\textstyle\frac{1}{2}}( e^{i \pi a} + e^{-i \pi a}) = {\textstyle\frac{1}{2}} (-1)^p ((q + \sqrt{q^2 - 1})^{-1} + (q + \sqrt{q^2 - 1})) = \\
        &= \frac{(-1)^p}{2} \frac{1 + q^2 + 2q \sqrt{q^2 - 1} + q^2 - 1}{q + \sqrt{q^2 - 1}} = (-1)^p q
    \end{align*}
    and thus $\cos \pi (\cos \pi a) = \pm 1$. But $0, 1 \notin f(G)$, therefore $a \notin g(G)$ and $g(G) \cap A = \emptyset$, proving (iii).

    For (iv), if $\D \subseteq G$, then $\restr{g}{\D} \in H(\D)$. Fix $0 < \theta < 1$, then for $\vert z \vert \leq \theta$ we have
    $$ d(z, \partial \D) = \inf_{w \in \partial \D} \vert z - w \vert \geq \inf_{w \in \partial \D} \left( \vert w \vert - \vert z \vert \right) \geq 1 - \theta. $$
    From (iii) it follows that $\restr{g}{\D}(\D)$ does not contain a disk of radius $1$. Let $0 < s < 1 - \theta$, then applying \Cref{cor:bloch-domain} to $\restr{g}{\D}$ implies that $\beta s \vert g'(z) \vert < 1$, for any $z \in \D$. Taking the supremum over $s$ and rearranging yields $ \vert g'(z) \vert \leq 1 / (\beta (1 - \theta))$, thus our desired estimate is shown by
    \begin{align*}
        \vert g(z) \vert &\leq \vert g(0) \vert + \vert g(z) - g(0) \vert \leq \vert g(0) \vert + \int_{[0, z]} \vert g'(\zeta) \vert \diff \zeta \leq \vert g(0) \vert + \frac{\theta}{\beta (1 - \theta)}. \qedhere
    \end{align*}
\end{proof}

\begin{theorem}[Schottky] \label{thm:schottky}
    There exists a function ${\psi : (0, 1) \times (0, \infty) \to (0, \infty)}$ such that for any $f \in H(\D; \C \setminus \{ 0, 1 \})$ with $\vert f(0) \vert \leq \omega$ it holds that
    \begin{equation}
        \vert f(z) \vert \leq \psi(\theta, \omega), \quad \vert z \vert \leq \theta.
    \end{equation}
\end{theorem}

\begin{proof}
    Note that for all $w \in \C$ we have $\vert \cos w \vert \leq e^{\vert w \vert}$ and $\frac{1}{2} \vert 1 + \cos w \vert \leq e^{\vert w \vert}$. Hence, from \Cref{lem:schottky-3}, we get
    \begin{align*}
        \vert f(z) \vert &= \vert {\textstyle \frac{1}{2}} ( 1 + \cos \pi ( \cos \pi g(z) ) ) \vert \leq \exp ( \pi \exp \pi \vert g(z) \vert ) \leq \\ &\leq \exp \left(\pi \exp \pi \left( \vert g(0) \vert + \frac{\theta}{\beta(1 - \theta)} \right) \right) \leq \\
        &\leq \exp \left( \pi \exp \pi \left( 3 + 2 \omega + \frac{\theta}{\beta(1 - \theta)} \right) \right),
    \end{align*}
    and defining $\psi(\theta, \omega)$ as the final term establishes the assertion.
\end{proof}

\Cref{lem:schottky-3} is quite powerful, as it contains not only Schottky's Theorem, but Picard's Little Theorem as well:

\begin{theorem}[Picard's Little Theorem] \label{thm:picard-little}
    Let $f \in H(\C; \C \setminus \{ a, b \})$ for distinct points $a, b \in \C$. Then $f$ is constant.
\end{theorem}

\begin{proof}
    Consider $f_1(z) \coloneqq \frac{f(z) - a}{b - a} \in H(\C; \C \setminus \{ 0, 1 \})$. By \Cref{lem:schottky-3} there is some $g \in H(\C)$ such that $f_1 = \frac{1}{2}(1 + \cos \pi (\cos \pi g))$ and $g(\C)$ does not contain a disk of radius $1$. By \Cref{cor:bloch-entire}, we thereby have that $g$ must be constant, and therefore so are $f_1$ and $f$.
\end{proof}
\section{Normal families}
\label{sec:normal-families}

First, we recall a generalized form of locally uniform convergence:

\begin{definition}
    Let $G \subseteq \C$ be a domain, $f \in H(G)$ and $(f_n)_{n \in \N}$ a sequence in $H(G)$. We say that \emph{$f_n$ converges compactly in $G$ to $f$}, or \emph{$f_n$ converges compactly in $G$ to $\infty$} as $n \to \infty$, if for every compact set $K \subset G$
    \begin{equation}
        \lim_{n \to \infty} \sup_{z \in K} \vert f_n(z) - f(z) \vert = 0, \quad \textrm{or} \quad \lim_{n \to \infty} \inf_{z \in K} \vert f_n(z) \vert = \infty. \qedhere
    \end{equation}
\end{definition}

\begin{remark}
    One can define compact convergence generally for functions from a topological space $(X, \mathcal{T})$ into a metric space $(Y, d_Y)$. Compact convergence to $\infty$ can then be seen as compact convergence to a constant function with value $\infty$, where $d_Y(y, \infty)$ is appropriately defined for $y \in Y$. This is precisely the case when considering the chordal metric of the Riemann sphere, but we shall not elaborate further on this.
\end{remark}

\begin{definition}
    Let $f : \C \to \C$ and $a \in \C$, then the \emph{a-points of $f$} are defined as the zeros of $f(z) - a$, that is the set of all points $w \in \C$ with $f(w) = a$.
\end{definition}

A well-known theorem on compact convervence is:

\begin{theorem}[Hurwitz] \label{thm:hurwitz}
    Let $G \subseteq \C$ be a domain and $(f_n)_{n \in \N}$ a sequence in $H(G)$ that converges compactly to $f \in H(G)$. If for every $n \in \N$ the number of $a$-points of $f_n$ is bounded by some $m \in \N_0$, then either
    \begin{itemize}
        \item the number of $a$-points of $f$ are also bounded by $m$ or
        \item $f \equiv a$.
    \end{itemize}
\end{theorem}

As an immediate consequence we obtain that compact convergence is, in some ways, compatible with reciprocals:

\begin{lemma} \label{lem:compact-convergence-reciprocals}
Let $G \subseteq \C$ be a domain and $(f_n)_{n \in \N}$ a sequence in $H(G; \C \setminus \{ 0 \})$. If it converges compactly to some $f \in H(G)$, then it holds that either:
\begin{itemize}
    \item $0 \notin f(G)$ and $1 / f_n \to 1 / f$ compactly in $G$, or
    \item $f \equiv 0$ and $1 / f_n \to \infty$ compactly in $G$.
\end{itemize}
If it converges compactly to $\infty$, then $1 / f_n \to 0$ compactly in $G$.
\end{lemma}

\begin{proof}
    For the equivalence ``$f_n \to 0$ if and only if $1 / f_n \to \infty$'' it suffices to notice that for any compact $K \subset G$ we have
    \begin{equation*}
        \frac{1}{\sup_{z \in K} \vert f_n(z) \vert} = \inf_{z \in K} \left\vert \frac{1}{f_n(z)} \right\vert.
    \end{equation*}
    Since the $f_n$ vanish nowhere, by Hurwitz' Theorem we either have $0 \notin f(G)$ or $f \equiv 0$. In the latter case we have just shown that $1 / f_n \to \infty$ compactly.

    In the former case we have, again for any compact $K \subset G$, that $m \coloneqq \min_{z \in K} \vert f(z) \vert > 0$ and $\sup_{z \in K} \vert f_n(z) - f(z) \vert < \frac{m}{2}$ for sufficiently large $n \in \N$. Thus for all $z \in K$
    \begin{equation*}
        \frac{m}{2} > \vert f(z) - f_n(z) \vert \geq \vert f(z) \vert - \vert f_n(z) \vert \geq  m - \vert f_n(z) \vert
    \end{equation*}
    and $\vert f_n(z) \vert \geq \frac{m}{2}$. We obtain, for large $n \in \N$,
    \begin{equation*}
        \sup_{z \in K} \left\vert \frac{1}{f_n(z)} - \frac{1}{f(z)} \right\vert = \sup_{z \in K} \left\vert \frac{f(z) - f_n(z)}{f(z) f_n(z)} \right\vert \leq \sup_{z \in K} \left\vert f_n(z) - f(z) \right\vert \cdot \frac{1}{m} \cdot \frac{2}{m}
    \end{equation*}
    and therefore, after letting $n \to \infty$, that $1 / f_n \to 1 / f$ compactly in $G$.
\end{proof}

\begin{definition} \label{def:normal-family}
    Let $G \subseteq \C$ be a domain and $\F \subseteq H(G)$, then $\F$ is called:
    \begin{itemize}
        \item \emph{locally bounded}, if for every $w \in G$ there is a neighborhood $U$ of $w$ and a constant $C > 0$ such that $\vert f(z) \vert \leq C$ for all $f \in \F$ and $z \in U$.
        \item \emph{normal} in $G$, if every sequence in $\F$ has a subsequence which converges compactly in $G$ to some $f \in H(G)$. If the limit $\infty$ is also permitted it is instead called \emph{$\ast$-normal}.
    \end{itemize}
\end{definition}

The former two concepts are equivalent by the following well-known theorem:

\begin{theorem}[Montel] \label{thm:montel}
    Let $G \subseteq \C$ be a domain, then a family $\F \subseteq H(G)$ is normal if and only if it is locally bounded.
\end{theorem}

The following theorem can be interpreted as a sharpened version of Montel's Theorem and is sometimes referred to as the \emph{fundamental normality test}.

\begin{theorem} \label{lem:montel-sharpened}
    Let $G \subseteq \C$ be a domain, then any family $\F \subseteq H(G, \C \setminus \{ 0, 1 \})$ is $\ast$-normal in $G$.
\end{theorem}

\begin{proof}
    We present the proof in three steps:
    \begin{enumerate}
        \item Let $w \in G$, $c > 0$ and $\F_* \subseteq \F$ such that $\vert f(w) \vert \leq c$ for all $f \in \F_*$. We aim to show that there is an open disk at $w$ in which $\F_*$ is bounded. Select $t > 0$ such that $B_t(w) \subseteq G$. Let $f \in \F_*$, then $g(z) \coloneqq f(tz + w) \in H(\D)$. By the maximum modulus principle and Schottky's Theorem we obtain
        $$ \sup_{z \in B_{t/2}(w)} \vert f(z) \vert \leq \sup_{z \in B_{1/2}(0)} \vert g(z) \vert \leq \sup_{\vert z \vert = 1/2} \vert g(z) \vert \leq \psi(1/2, c) $$
        and $f$ is bounded on the disk $B_{t/2}(w)$. Since $f$ was arbitrary, $\F_*$ is bounded as well.

        \item Fix some $w_* \in G$ and set $\F_1 \coloneqq \{ f \in \F : \vert f(w_*) \vert \leq 1 \}$. We aim to show that $\F_1$ is locally bounded in $G$. Consider the set
        $$ U \coloneqq \{ w \in G : \F_1 \textrm{ is bounded in a neighborhood of } w \}, $$
        by (1) we have that $w_* \in U$. Note that $U$ is open in $G$, since if $\F_1$ is bounded in a disk $B_r(w)$, then for any $w' \in B_r(w)$ there is a disk $B_{r'}(w') \subseteq B_r(w)$, on which $\F_1$ is bounded as well.
        
        Assume towards a contradiction that $U \neq G$, then there exists some $w \in \partial U \cap G$ such that $\F_1$ is unbounded in every neighborhood of $w$.

        If there were some $c > 0$ such that $\vert f(w) \vert \leq c$ for all $f \in \F_1$, then by (1) there would exist an open disk centered at $w$ on which $\F_1$ would be bounded -- contradicting our assumption on $w$. Thus for every $n \in \N$, we can find some $f_n \in \F_1$ such that $\vert f_n(w) \vert \geq n$ and we obtain that $\lim_{n \to \infty} \vert f_n(w) \vert = \infty$.
        
        Set $g_n \coloneqq 1 / f_n \in \F$, then $\lim_{n \to \infty} \vert g_n(w) \vert = 0$. In particular, the family $(g_n)_{n \in \N}$ is bounded at $w$ by some constant, thus by (1) the family is bounded in some disk $B$ around $w$. By Montel's Theorem it is therefore normal in $B$, and there exists a subsequence $(g_{n_k})_{k \in \N}$ which converges compactly to a $g \in H(B)$. The functions $g_{n_k}$ have no zeros, but $g(w) = 0$; by Hurwitz's Theorem we therefore have $g \equiv 0$. Then for any $z \in B \cap U$ we have
        $$ \lim_{k \to \infty} \vert f_{n_k}(z) \vert = \lim_{k \to \infty} 1 / \vert g_{n_k}(z) \vert = \infty, $$
        contradicting the assumption that $\F_1$ is bounded in a neighborhood of such $z$. We thus have $U = G$, therefore $\F_1$ is locally bounded and by Montel's Theorem therefore normal.

        \item We can now conclude the proof. Let $(f_n)_{n \in \N}$ be a sequence in $\F$, we claim that it has some subsequence which converges compactly to some function in $H(G)$ or to $\infty$.
        
        If infinitely many $f_n$ lie in $\F_1$, then there is a subsequence $(f_{n_m})_{m \in \N}$ in $\F_1$, which by (2) has a subsequence $(f_{n_{m_k}})_{k \in \N}$ in $\F_1$ which convergences compactly in $G$ to some $f \in H(G)$. This sequence is also a subsequence of $(f_n)_{n \in \N}$, concluding the claim in this case.
        
        On the other hand, if there are only finitely many $f_n$ in $\F_1$, then infinitely many $1 / f_n$ lie in $\F_1$. As above, we thus obtain some subsequence in $\F_1$, say $(g_n)_{n \in \N}$, converging compactly in $G$ to some $g \in H(G)$. The sequence $(1 / g_n)_{n \in \N}$ is a subsequence of $(f_n)_{n \in \N}$, which -- by \Cref{lem:compact-convergence-reciprocals} -- converges compactly to $1 / g$ if $0 \notin g(G) $, and to $\infty$ otherwise. \qedhere
    \end{enumerate}
\end{proof}
\section{Picard's Great Theorem}
\label{sec:picards-great-theorem}

\begin{theorem}[Montel] \label{thm:montel}
    Let $D \subset \C$ be open. Then a family $\mathscr{F} \subset H(D)$ is normal if and only if it is locally uniformly bounded.
\end{theorem}

\begin{lemma} \label{lem:montel-sharpened}
    Let $G \subseteq \C$ be a domain and consider the family $\mathscr{F} \coloneqq H(G, \C \setminus \{ 0, 1 \})$.
    \begin{enumerate}[i.]
        \item For $z_0 \in G$ and $\omega > 0$ let $\mathscr{F}_*$ be a subfamily of $\mathscr{F}$ satisfying $\vert f(z_0) \vert \leq \omega$ for all $f \in \mathscr{F}_*$. Then there exists a neighborhood of $z_0$ in which $\mathscr{F}_*$ is bounded.
        \item For $z_0 \in G$ the family $\mathscr{F}_1 \coloneqq \{ f \in \mathscr{F} : \vert f(z_0) \vert \leq 1 \}$ is locally bounded in $G$.
        \item The family $\mathscr{F}$ is normal in $G$.
    \end{enumerate}
\end{lemma}

\begin{lemma} \label{lem:great-picard-bounded}
    Let $f \in H(\D^\times; \C \setminus \{ 0, 1 \})$. Then $f$ or $1/f$ is bounded in a punctured neighborhood of $0$.
\end{lemma}

\begin{proof}
    \todo{TODO.}
\end{proof}

\begin{theorem}[Picard's Great Theorem] \label{thm:picards-great-theorem}
    Let $G \subset \C$ be open, $w \in G$ and suppose $f \in H(G \setminus \{ w \})$ such that $f$ has an essential singularity at $w$. Then $f$ takes on all values in $\C$, with at most one exception, in any punctured neighborhood of $w$.
\end{theorem}

\begin{proof}
    Aiming for contradiction, assume that $f$ only takes on $z_0, z_1 \in \C$ finitely often in some punctured neighborhood $W$ of $w$. Then $W$ contains a punctured disk of radius $\varepsilon > 0$ around $w$, on which $f$ does not assume $z_0, z_1$. The function
    $$ g(z) \coloneqq \frac{f(\varepsilon z + w) - z_0}{z_1 - z_0} \in H(\D^\times; \C \setminus \{ 0, 1 \}) $$
    has an essential singularity at zero. By \Cref{lem:great-picard-bounded}, we have that either $g$ or $1/g$ must be bounded in a neighborhood of zero. In the former case the singularity must therefore be removable, whereas in the latter case it must be a pole, yielding a contradiction.
\end{proof}

\begin{corollary} \label{cor:transcendental-every-value-inf}
    Every entire transcendental function assumes every value in $\C$ infinitely often, with at most one exception.
\end{corollary}

\begin{proof}
    If $f(z) = \sum_{n \in \Z_{\geq 0}} a_n z^n$, then $f(z^{-1}) = \sum_{n \in \Z_{\leq 0}} a_{-n} z^{n}$, thus $f$ has an essential singularity at $\infty$ and Picard's Great Theorem (\ref{thm:picards-great-theorem}) concludes the claim.
\end{proof}

\begin{corollary}[Picard's Little Theorem] \label{thm:picards-little-theorem}
    Every nonconstant entire function omits at most one value.
\end{corollary}

\begin{proof}
    A non-constant entire function $f$ is either a non-constant polynomial or a transcendental function. In the latter case, the claim follows from \Cref{cor:transcendental-every-value-inf}.
    
    Otherwise let $w \in \C$, then $f(z) - w$ is a non-constant polynomial. By the Fundamental Theorem of Algebra it has a zero at some $z_0 \in \C$. Therefore $f(z_0) = w$ and $f$ attains all values.
\end{proof}