\section{Schottky's Theorem}
\label{sec:schottkys-theorem}

Holomorphic functions, which omit the values $0$ and $1$, have a universal estimate on the growth of their modulus, which will be given by Schottky's Theorem.

For a domain $G \subseteq \C$ and a set $E \subseteq \C$ we define $H(G; E)$ as the set\footnote{Note that, unlike $H(G)$, the set $H(G; E)$ is usually not a linear space.} of all $f \in H(G)$ such that $f(G) \subseteq E$.

\begin{lemma} \label{lem:schottky-1}
    It holds that:
    \begin{enumerate}[i.]
        \item If $a, b \in \R$ with $\cos \pi a = \cos \pi b$, then $b = \pm a + 2n$ for some $n \in \Z$.
        \item For every $w \in \C$ there exists a $v \in \C$ such that $\cos \pi v = w$ and $\vert v \vert \leq 1 + \vert w \vert$.
    \end{enumerate}
\end{lemma}

\begin{proof}
    For the first part it suffices to note that
    $$ 0 = \cos \pi a - \cos \pi b = \textstyle -2 \sin \frac{\pi}{2} ( a + b ) \sin \frac{\pi}{2} ( a - b ). $$
    Since the complex cosine function is surjective and $2 \pi$-periodic, we can choose $v = a + i b$ with $w = \cos \pi v$ and $\vert a \vert \leq 1$. Now we have
    \begin{align*}
        \vert w \vert^2 &= \vert \cos (\pi a + i \pi b) \vert^2 = \vert \cos \pi a \cos i \pi b + \sin \pi a \sin i \pi b \vert^2 = \\
        &= \vert \cos \pi a \cosh \pi b - i \sin \pi a \sinh \pi b \vert^2 = \\
        &= \cos^2 \pi a \cosh^2 \pi b + \sin^2 \pi a \sinh^2 \pi b = \\
        &= \cos^2 \pi a + \cos^2 \pi a \sinh^2 \pi b + \sin^2 \pi a \sinh^2 \pi b = \\
        &= \cos^2 \pi a + \sinh^2 \pi b \geq \sinh^2 \pi b \geq \pi^2 b^2,
    \end{align*}
    where the last inequality holds since $\sinh x \geq x$ for $x \geq 0$. We conclude
    \begin{equation*}
        \vert v \vert = \sqrt{a^2 + b^2} \leq \sqrt{1 + \vert w \vert^2 / \pi^2} \leq 1 + \vert w \vert
    \end{equation*}
\end{proof}

We recall the following result: Let $G \subseteq \C$ be a simply connected domain and $f \in H(G)$, such that $f$ vanishes nowhere on $G$. There is a $g \in H(G)$ such that $f = e^g$, which can also be used to obtain multiplicative $n$-th roots of such functions by defining $\sqrt[n]{f} \coloneqq e^{g/n}$.

\begin{lemma} \label{lem:schottky-2}
    Let $G \subseteq \C$ be a simply connected domain and $f \in H(G; \C \setminus \{ -1, 1 \})$, then there exists an $F \in H(G)$ such that $f = \cos F$.
\end{lemma}

\begin{proof}
    Since $1 - f^2$ vanishes nowhere in $G$, it has a square root $g \in H(G)$, therefore
    \begin{equation*}
        1 = f^2 + g^2 = (f + ig)(f - ig).
    \end{equation*}
    Thus $f + ig$ vanishes nowhere and there exists an $F \in H(G)$ with $f + ig = e^{iF}$. By the above we also have $f - ig = e^{-iF}$ and therefore
    \begin{equation*}
        f = {\textstyle \frac{1}{2}} (e^{iF} + e^{-iF}) = \cos F. \qedhere
    \end{equation*}
\end{proof}

\begin{lemma} \label{lem:schottky-3}
    Let $G \subseteq \C$ be a simply connected domain and $f \in H(G; \C \setminus \{ 0, 1 \})$. Then there exists a $g \in H(G)$ such that:
    \begin{enumerate}[i.]
        \item $f = \frac{1}{2} ( 1 + \cos \pi (\cos \pi g))$.
        \item $\vert g(0) \vert \leq 3 + 2 \vert f(0) \vert$.
        \item $g(G)$ contains no disk of radius $1$.
        \item If $\D \subseteq G$ then $\vert g(z) \vert \leq \vert g(0) \vert + \frac{\theta}{\beta (1 - \theta)}$, for all $\vert z \vert \leq \theta$ where $0 < \theta < 1$.
    \end{enumerate}
\end{lemma}

\begin{proof}
    By \Cref{lem:schottky-2}, there exists a $\widetilde{F} \in H(G)$ such that $2f - 1 = \cos \pi \widetilde{F}$ and by \Cref{lem:schottky-1} there is a $b \in \C$ with $\cos \pi b = 2f(0) - 1$ and $\vert b \vert \leq 2 + \vert 2 f(0) - 1 \vert \leq 2 + 2 \vert f(0) \vert$. Furthermore, since $\cos \pi b = \cos \pi \widetilde{F}(0)$, we have $b = \pm \widetilde{F}(0) + 2k$ for some $k \in \Z$. Then $F \coloneqq \pm \widetilde{F} + 2k \in H(G)$ satisfies $F(0) = b$ and $2f - 1 = \cos \pi F$.
    
    Since $F$ must omit all integers, there exists a $\widetilde{g} \in H(G)$ such that $F = \cos \pi \widetilde{g}$. Similarly, there is an $a \in \C$ such that $\cos \pi a = b$ and $\vert a \vert \leq 1 + \vert b \vert \leq 3 + 2 \vert f(0) \vert$. Like in the above above, since $\cos \pi a = \cos \pi \widetilde{g}(0)$, we have $a = \pm \widetilde{g}(0) + 2\ell$ for some $\ell \in \Z$, thus $g \coloneqq \pm \widetilde{g} + 2\ell \in H(G)$ satisfies $g(0) = a$ and $F = \cos \pi g$. Together we obtain
    $$ \textstyle f = \frac{1}{2} (1 + \cos \pi (\cos \pi g)), \quad \textrm{and} \quad \vert g(0) \vert = \vert a \vert \leq 3 + 2 \vert f(0) \vert $$
    and thus have shown (i) and (ii).

    To show (iii) we consider the set
    $$ A \coloneqq \{ m \pm i \pi^{-1} \log (n + \sqrt{n^2 - 1}) : m \in \Z, n \in \N \setminus \{ 0 \} \}, $$
    the points of which can be considered the vertices of a rectangular grid in $\C$. The width of such a rectangular cell is $1$, and since
    \begin{align*}
        \log ((n+1) &+ \sqrt{(n+1)^2 - 1}) - \log (n + \sqrt{n^2 - 1}) = \\
        &= \log \frac{1 + \frac{1}{n} + \sqrt{1 + \frac{2}{n}}}{1 + \sqrt{1 - \frac{1}{n^2}}} \leq \log (2 + \sqrt{3}) < \pi
    \end{align*}
    their height is bounded above by some $C < 1$. Therefore, for all $z \in \C$ there is a $w_z \in A$ such that $\vert \Re z - \Re w_z \vert \leq \frac{1}{2}$ and $\vert \Im z - \Im w_z \vert \leq \frac{C}{2}$. Thus we have
    $$ \vert z - w_z \vert \leq \vert \Re z - \Re w_z \vert + \vert \Im z - \Im w_z \vert \leq \frac{1}{2} + \frac{C}{2} < 1. $$
    If we can show that $g(G) \cap A = \emptyset$, then $g(G)$ therefore cannot contain a disk of radius $1$. Let $a = p + i \pi^{-1} \log(q + \sqrt{q^2 - 1}) \in A$, then
    \begin{align*}
        \cos \pi a &= {\textstyle\frac{1}{2}}( e^{i \pi a} + e^{-i \pi a}) = {\textstyle\frac{1}{2}} (-1)^p ((q + \sqrt{q^2 - 1})^{-1} + (q + \sqrt{q^2 - 1})) = \\
        &= (-1)^p \frac{1}{2} \frac{1 + q^2 + 2q \sqrt{q^2 - 1} + q^2 - 1}{q + \sqrt{q^2 - 1}} = (-1)^p q
    \end{align*}
    and thus $\cos \pi (\cos \pi a) = \pm 1$. But $0, 1 \notin f(G)$, therefore $a \notin g(G)$ and $g(G) \cap A = \emptyset$, proving (iii).

    For (iv), if $\D \subseteq G$, then $\restr{g}{\D} \in H(\D)$. Fix $0 < \theta < 1$, then for $\vert z \vert \leq \theta$ we have
    $$ d(z, \partial \D) = \inf_{w \in \partial \D} \vert z - w \vert \geq \inf_{w \in \partial \D} \left( \vert w \vert - \vert z \vert \right) \geq 1 - \theta. $$
    From (iii) it follows that $\restr{g}{\D}(\D)$ does not contain a disk of radius $1$. Let $0 < s < 1 - \theta$, then applying \Cref{cor:bloch-domain} to $\restr{g}{\D}$ implies that $\beta s \vert g'(z) \vert < 1$. Taking the supremum over $s$, and rearranging yields $ \vert g'(z) \vert \leq 1 / (\beta (1 - \theta))$, thus our desired estimate is shown by
    \begin{align*}
        \vert g(z) \vert &\leq \vert g(0) \vert + \vert g(z) - g(0) \vert \leq \vert g(0) \vert + \int_{[0, z]} \vert g'(\zeta) \vert \diff \zeta \leq \vert g(0) \vert + \frac{\theta}{\beta (1 - \theta)}. \qedhere
    \end{align*}
\end{proof}

\begin{theorem}[Schottky] \label{thm:schottky}
    There exists a function ${\psi : (0, 1) \times (0, \infty) \to (0, \infty)}$ such that for any $f \in H(\D; \C \setminus \{ 0, 1 \})$ with $\vert f(0) \vert \leq \omega$ it holds that
    \begin{equation}
        \vert f(z) \vert \leq \psi(\theta, \omega), \quad \vert z \vert \leq \theta.
    \end{equation}
\end{theorem}

\begin{proof}
    Note that for all $w \in \C$ we have $\vert \cos w \vert \leq e^{\vert w \vert}$ and $\frac{1}{2} \vert 1 + \cos w \vert \leq e^{\vert w \vert}$. Hence, from \Cref{lem:schottky-3}, we get
    \begin{align*}
        \vert f(z) \vert &= \vert {\textstyle \frac{1}{2}} ( 1 + \cos \pi ( \cos \pi g(z) ) ) \vert \leq \exp ( \pi \exp \pi \vert g(z) \vert ) \leq \\ &\leq \exp (\pi \exp \pi (\vert g(0) \vert + \theta / \beta(1 - \theta))) \leq \\
        &\leq \exp ( \pi \exp \pi ( 3 + 2 \omega + \theta / \beta(1 - \theta) ) ),
    \end{align*}
    and defining $\psi(\theta, \omega)$ as the final term establishes the assertion.
\end{proof}

\Cref{lem:schottky-3} is quite powerful, as it contains not only Schottky's Theorem, but Picard's Little Theorem as well:

\begin{theorem}[Picard's Little Theorem] \label{thm:picard-little}
    Let $f \in H(\C; \C \setminus \{ a, b \})$ for distinct points $a, b \in \C$. Then $f$ is constant.
\end{theorem}

\begin{proof}
    Consider $f_1(z) \coloneqq \frac{f(z) - a}{b - a} \in H(\C; \C \setminus \{ 0, 1 \})$. By \Cref{lem:schottky-3} there is some $g \in H(\C)$ such that $f_1 = \frac{1}{2}(1 + \cos \pi (\cos \pi g))$ and $g(\C)$ does not contain a disk of radius $1$. By \Cref{cor:bloch-entire}, we thereby have that $g$ must be constant, and therefore so are $f_1$ and $f$.
\end{proof}