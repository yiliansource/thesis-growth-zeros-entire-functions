\section{Schottky's Theorem}
\label{sec:schottkys-theorem}

Holomorphic functions which omit the values $0$ and $1$ have a universal estimate on the growth of their modulus, given by Schottky's Theorem.

For sets $G, E \subseteq \C$ we define by $H(G; E)$ the set of all $f \in H(G)$ such that $f(G) \subseteq E$.

\begin{lemma} \label{lem:schottky-1}
    It holds that:
    \begin{enumerate}[i.]
        \item If $a, b \in \R$ with $\cos \pi a = \cos \pi b$, then $b = \pm a + 2n$ for some $n \in \Z$.
        \item For every $w \in \C$ there exists a $v \in \C$ such that $\cos \pi v = w$ and $\vert v \vert \leq 1 + \vert w \vert$.
    \end{enumerate}
\end{lemma}

\begin{proof}
    For the first part, it suffices to notice that
    $$ 0 = \cos \pi a - \cos \pi b = \textstyle -2 \sin \frac{\pi}{2} ( a + b ) \sin \frac{\pi}{2} ( a - b ). $$
    Since the complex cosine function is surjective and $\R$-periodic, we can choose $v = a + i b$ with $w = \cos \pi v$ and $\vert a \vert \leq 1$. Now we have
    \begin{align*}
        \vert w \vert^2 &= \vert \cos (\pi a + i \pi b) \vert^2 = \vert \cos \pi a \cos i \pi b + \sin \pi a \sin i \pi b \vert^2 = \\
        &= \vert \cos \pi a \cosh \pi b - i \sin \pi a \sinh \pi b \vert^2 = \\
        &= \cos^2 \pi a \cosh^2 \pi b + \sin^2 \pi a \sinh^2 \pi b = \\
        &= \cos^2 \pi a + \cos^2 \pi a \sinh^2 \pi b + \sin^2 \pi a \sinh^2 \pi b = \\
        &= \cos^2 \pi a + \sinh^2 \pi b \geq \sinh^2 \pi b \geq \pi^2 b^2,
    \end{align*}
    where the last inequality holds since $\sinh x \geq x$ for $x \geq 0$. We conclude
    \begin{equation*}
        \vert v \vert = \sqrt{a^2 + b^2} \leq \sqrt{1 + \vert w \vert^2 / \pi^2} \leq 1 + \vert w \vert
    \end{equation*}
\end{proof}

\begin{lemma} \label{lem:schottky-2}
    Let $G \subseteq \C$ be a simply connected domain and $f \in H(G; \C \setminus \{ -1, 1 \})$. Then there exists an $F \in H(G)$ such that $f = \cos F$.
\end{lemma}

\begin{proof}
    Since $1 - f^2$ vanishes nowhere in $G$ it has a square root $g \in H(G)$. It follows that
    $$ 1 = f^2 + g^2 = (f + ig)(f - ig). $$
    Thus $f + ig$ vanishes nowhere and there exists an $F \in H(G)$ with $f + ig = e^{iF}$. Additionally we have $f - ig = e^{-iF}$ and therefore
    $$ f = {\textstyle \frac{1}{2}} (e^{iF} + e^{-iF}) = \cos F. $$
\end{proof}

\begin{lemma} \label{lem:schottky-3}
    Let $f \in H(\cl{\D}; \C \setminus \{ 0, 1 \})$. Then there exists a $g \in H(\cl{\D})$ such that:
    \begin{enumerate}[i.]
        \item $f = \frac{1}{2} ( 1 + \cos \pi (\cos \pi g))$, and $\vert g(0) \vert \leq 3 + 2 \vert f(0) \vert$.
        \item $g(\cl{\D})$ contains no disk of radius $1$.
        \item $\vert g(z) \vert \leq \vert g(0) \vert + \theta / b (1 - \theta)$, for all $\vert z \vert \leq \theta$ where $0 < \theta < 1$.
    \end{enumerate}
\end{lemma}

\begin{proof}
    By \Cref{lem:schottky-2} there exists a $\widetilde{F} \in H(\cl{\D})$ such that $2f - 1 = \cos \pi \widetilde{F}$. By \Cref{lem:schottky-1} there is a $b \in \C$ such that $\cos \pi b = 2f(0) - 1$ and $\vert b \vert \leq 1 + \vert 2 f(0) - 1 \vert \leq 2 + 2 \vert f(0) \vert$. Additionally we have that $b = \pm \widetilde{F}(0) + 2k$ for some $k \in \Z$. Then $F \coloneqq \pm \widetilde{F} + 2k \in H(\cl{\D})$ satifies $F(0) = b$. Since $F$ must omit all integers, there exists a $\widetilde{g} \in H(\cl{\D})$ such that $F = \cos \pi \widetilde{g}$. Again, there is an $a \in \C$ such that $\cos \pi a = b$ and $\vert a \vert \leq 1 + \vert b \vert \leq 3 + 2 \vert f(0) \vert$. Like above, we have $a = \pm \widetilde{g}(0) + 2\ell$ for some $\ell \in \Z$, thus $g \coloneqq \pm \widetilde{g} + 2\ell \in H(\cl{\D})$ satisfies $g(0) = a$. Ultimately, we obtain
    $$ \textstyle f = \frac{1}{2} (1 + \cos \pi (\cos \pi g)), \quad \textrm{and} \quad \vert g(0) \vert = \vert a \vert \leq 3 + 2 \vert f(0) \vert. $$

    To show (ii) we consider the set
    $$ A \coloneqq \{ m \pm i \pi^{-1} \log (n + \sqrt{n^2 - 1}) : m \in \Z, n \in \N \setminus \{ 0 \} \}, $$
    the points of which can be considered the vertices of a rectangular grid in $\C$. The width of said rectangles is $1$, and since
    \begin{align*}
        \log ((n+1) &+ \sqrt{(n+1)^2 - 1}) - \log (n + \sqrt{n^2 - 1}) = \\
        &= \log \frac{1 + \frac{1}{n} + \sqrt{1 + \frac{2}{n}}}{1 + \sqrt{1 - \frac{1}{n^2}}} \leq \log (2 + \sqrt{3}) < \pi
    \end{align*}
    their height is strictly bounded above by $1$. Let $a = p + i \pi^{-1} \log(q + \sqrt{q^2 - 1}) \in A$, then
    \begin{align*}
        \cos \pi a &= {\textstyle\frac{1}{2}}( e^{i \pi a} + e^{-i \pi a}) = {\textstyle\frac{1}{2}} (-1)^p ((q + \sqrt{q^2 - 1})^{-1} + (q + \sqrt{q^2 - 1})) = \\
        &= (-1)^p \frac{1}{2} \frac{1 + q^2 + 2q \sqrt{q^2 - 1} + q^2 - 1}{q + \sqrt{q^2 - 1}} = (-1)^p q.
    \end{align*}
    Thus $\cos \pi (\cos \pi a) = \pm 1$. But $0, 1 \notin f(\cl{\D})$, therefore $a \notin g(\cl{\D})$, thus $g(\cl{\D}) \cap A = \emptyset$ and $g(\cl{\D})$ cannot contain a disk of radius $1$.

    For (iii), let $0 < \theta < 1$, then for $\vert z \vert \leq \theta$ we have $1 - \theta \leq d(z, \partial \D)$. By (ii) and \Cref{cor:bloch-domain} it follows that $\beta (1 - \theta) \vert g'(z) \vert \leq 1$. Let $\gamma$ denote the line connecting $0$ and $z$, then
    \begin{align*}
        \vert g(z) \vert &\leq \vert g(0) \vert + \vert g(z) - g(0) \vert \leq \vert g(0) \vert + \int_\gamma \vert g'(\zeta) \vert \diff \zeta \leq \vert g(0) \vert + \frac{\theta}{\beta (1 - \theta)}
    \end{align*}
    yields our desired estimate.
\end{proof}

\begin{remark}
    One may notice that in the proof of \Cref{lem:schottky-3} (i) and (ii) we never used the fact that we act on the unit disk. Indeed, applying the proof for $\C$, we can quickly obtain Picard's Little Theorem, which states that any $f \in H(\C; \C \setminus \{ a, b \})$, for distinct points $a, b \in \C$, is constant:

    Let $g(z) \coloneqq \frac{f(z) - a}{b - a} \in H(\C; \C \setminus \{ 0, 1 \})$ then, by the argument above, $g$ does not contain a disk of radius $1$. \Cref{cor:bloch-entire} thus implies that $g$ is constant, and therefore so is $f$.
\end{remark}

\begin{theorem}[Schottky] \label{thm:schottky}
    There exists a function ${\psi(\theta, \omega) : (0, 1) \times (0, \infty) \to (0, \infty)}$ such that for any $f \in H(\cl{\D}; \C \setminus \{ 0, 1 \})$ with $\vert f(0) \vert \leq \omega$ it holds that
    \begin{equation}
        \vert f(z) \vert \leq \psi(\theta, \omega), \quad \vert z \vert \leq \theta.
    \end{equation}
\end{theorem}

\begin{proof}
    % \begin{equation*}
    %     \psi(\theta, \omega) \coloneqq \exp \left( \pi \exp \pi \left( 3 + 2 \omega + \frac{\theta}{\beta (1 - \theta)} \right) \right).
    % \end{equation*}
    Note that for all $w \in \C$ we have $\vert \cos w \vert \leq e^{\vert w \vert}$ and $\frac{1}{2} \vert 1 + \cos w \vert \leq e^{\vert w \vert}$. Hence, from \Cref{lem:schottky-3}, we get
    \begin{align*}
        \vert f(z) \vert &= \vert {\textstyle \frac{1}{2}} ( 1 + \cos \pi ( \cos \pi g(z) ) ) \vert \leq \exp ( \pi \exp \pi \vert g(z) \vert ) \leq \\ &\leq \exp (\pi \exp \pi (\vert g(0) \vert + \theta / \beta(1 - \theta))) \leq \\
        &\leq \exp ( \pi \exp \pi ( 3 + 2 \omega + \theta / \beta(1 - \theta) ) ),
    \end{align*}
    and defining $\psi(\theta, \omega)$ as the rightmost term establishes the assertion.
\end{proof}