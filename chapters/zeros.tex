\chapter{Zeros}
\label{ch:zeros}

\todo{Maybe Jensen? Not sure if I need it earlier. Anyhow, include the equivalent version using the zero counting function.}

\begin{definition}
    Let $f$ be an entire function satisfying $f(0) \neq 0$. Let $(r_j)_{j \in \N}$ denote the moduli of the zeros of $f$ (if any) arranged in non-decreasing order. Then
    $$ \rho_1 \coloneqq \inf \{ \alpha > 0 : \sum_{n=1}^\infty \frac{1}{r^\alpha_n} < \infty \} $$
    is called the \emph{exponent of convergence of the zeros of f}. If $f$ has finitely many zeros, then we set $\rho_1 = 0$ by convention. \todo{I am not sure about this -- the paper only establishes this convention if $f$ has no zeros at all.}

    Furthermore, the \emph{exponent of convergence of the $a$-points of $f$} is defined as exponent of convergence of $f(z) - a$.
\end{definition}

\begin{theorem}
    Let $f$ be an entire function of finite order $\rho$ and exponent of convergence of zeros $\rho_1$. Then $\rho_1 \leq \rho$.
\end{theorem}

\begin{proof}
    \todo{TODO.}
\end{proof}

\begin{example}
    \todo{An example of a series where we see some convergence for some appropriate function using the above.}
\end{example}

\begin{theorem}
    Let $P$ be a Weierstrass canonical product of finite order $\rho$ and exponent of convergence of zeros $\rho_1$. Then $\rho_1 = \rho$.
\end{theorem}

\begin{proof}
    \todo{TODO.}
\end{proof}

\begin{theorem}
    Let $f$ be an entire function of finite order $\rho$ and exponent of convergence of zeros $\rho_1$. If $\rho$ is not an integer, then $\rho = \rho_1$.
\end{theorem}

\begin{proof}
    \todo{TODO.}
\end{proof}

\begin{theorem}
    Let $f$ be an entire function of finite, non-integer order. Then $f$ has infinitely many zeros.
\end{theorem}

\begin{proof}
    \todo{TODO.}
\end{proof}

\todo{Maybe introduce Borel exceptional values as a definition? But then again, I will never need them again. Maybe also add a remark on the relation to lacunary values (Picard).}

\begin{theorem}[Borel]
    \todo{Existence of Borel exceptional values.}
\end{theorem}

\begin{proof}
    \todo{TODO.}
\end{proof}