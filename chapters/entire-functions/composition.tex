\section{Composition}
\label{seq:composition}

As seen by \Cref{prop:algebraic-properties-order}, the order of the sum or product of two entire functions is reasonably bounded by the order of the functions involved. This is not the case when composition is involved. Indeed, consider $e^{e^z}$, which has infinite order, yet is the composition of two functions of order $1$. Necessary conditions for the order of a composition to be finite will be illustrated by Pòlya's Theorem, the proof of which relies on the following result:

\begin{lemma}[Bohr]
    Let $0 < R < 1$ and suppose $f$ is analytic on $\cl{\D}$, such that $f(0) = 0$ and $M_f(R) = 1$. Let $r_f$ denote the largest $r \geq 0$ such that $C_r \subseteq f(\cl{\D})$. Then we have $r_f > C > 0$, where $C$ is a constant depending only on $R$.
\end{lemma}

\begin{proof}
    \todo{TODO. Note that the proof relies on the strong form of Schottky's Theorem.}
\end{proof}

\begin{theorem}[Pólya] \label{thm:polya}
    Let $g, h$ be entire. For the order of $g \circ h$ to be finite, it must hold that either
    \begin{enumerate}[i.]
        \item $h$ is a polynomial and $g$ of finite order, or
        \item $h$ is of finite order, not a polynomial, and $g$ is of order zero.
    \end{enumerate}
\end{theorem}

\begin{proof}
    \todo{TODO.}
\end{proof}

\begin{theorem}[Thron] \label{thm:thron}
    Let $g$ be an entire function of finite order, not a polynomial, which takes some value $w$ only finitely often. Suppose further that there exists some function $f$ such that $f \circ f = g$. Then $f$ is not entire.
\end{theorem}

\begin{proof}
    Seeking contradiction, suppose $f$ were entire. Since $g$ is not a polynomial, \Cref{thm:polya} implies that $f$ is of order $0$ and not a polynomial. Let $(z_j)_{j \in J}$ denote the points where $f$ equals $w$. For each $m \in J$ we additionally denote by $(z_{j,m})_{j \in J_m}$ the points where $f$ equals $z_m$. Thus, for each $m \in J$ and $n \in J_m$ we have
    $$ g(z_{n,m}) = f(f(z_{n,m})) = f(z_m) = w. $$
    Our assumption on $g$ assures that there must only be finitely many distinct points among the $(z_{n,m})_{m \in J,n \in J_m}$. Thus, each point in $(z_j)_{j \in J}$ is only taken on by $f$ finitely often.

    By \Cref{cor:transcendental-every-value-inf}, $f$ attains all values in the complex plane infinitely often, with at most one exception. This implies that that there is at most one $z_0$ in $(z_j)_{j \in J}$ that is only taken on finitely often by $f$.

    If there is no such $z_0$, then $h(z) \coloneqq f(z) - w$ is entire, of order $0$ and nowhere $0$. Thus, by Hadamard's Theorem (\ref{thm:hadamard}), $h$ must be constant, and therefore $f$ aswell, a contradiction.

    If such a $z_0$ exists, then $h(z) \coloneqq f(z) - w$ has a zero of finite order $n \in \N$ at $z_0$. Therefore we can write $h(z) = (z - z_0)^n p(z)$, where $p$ is entire, of order $0$ and nowhere $0$. Again, this implies that $p$ is constant, and therefore $f$ a polynomial, a contradiction.
\end{proof}

\begin{example}
    A natural application of \Cref{thm:thron} is taking $g$ to be $e^z$, which never takes on $0$ as a value. Indeed, this implies that there is no entire function $f$ satisfying
    $$ f(f(z)) = e^z. $$
    On the other hand, there does exist a real-analytic function satisfying the above, as demonstrated by H. Kneser. \todo{I probably still need a citation here.}
\end{example}