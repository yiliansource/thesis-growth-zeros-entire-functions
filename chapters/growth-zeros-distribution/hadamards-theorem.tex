\section{Hadamard's Theorem}
\label{sec:hadamards-theorem}

\begin{theorem}[Weierstrass \cite{bluemlinger-complex-analysis,stein-shakarchi-princeton}] \label{thm:weierstrass}
    Let $(z_j)_{j \in \N}$ be a sequence in $\C$ without accumulation points. Then there exists an $E \in H(\C)$ (called the \emph{Weierstrass canonical product} formed from said sequence) that has zeros precisely at $(z_j)_{j \in \N}$, and $z_j$ has multiplicity equal to how often $z_j$ occurs in the sequence.
\end{theorem}

In particular, we have
\begin{equation} \label{eq:weierstrass-canonical-product}
    E(z) = z^k \prod_{n=1}^\infty \left( 1 - \frac{z}{z_n} \right) e^{R_n(z / z_n)},
\end{equation}
where $k$ is the order of the zero at $z = 0$ and $R_n(z/z_n)$ is a polynomial, namely a truncation of the power series for $-\log(1 - \frac{z}{z_n})$ chosen of smallest degree to ensure convergence of the product~\cite{segal-complex-analysis}.

Weierstrass' Theorem is also known as the \emph{Weierstrass Factorization Theorem}, due to the following corollary:

\begin{corollary} \label{cor:weierstrass-factorization}
    Let $f \in H(\C)$ have zeros $(z_j)_{j \in \N}$. Then there exists a $g \in H(\C)$ such that
    $$ f(z) = e^{g(z)} E(z), $$
    where $E \in H(\C)$ is a Weierstrass canonical product formed from $(z_j)_{j \in \N}$.
\end{corollary}

\begin{proof}
    Since $f / E$ has removable singularities at all $(z_j)_{j \in \N}$, we have that $f / E$ is entire and nowhere zero. Thus there exists an entire function $g$ with $f / E = e^g$, which yields $f = e^g E$.
\end{proof}

Hadamard's Theorem will show that, for functions of finite order $\rho$, the function $g$ in \Cref{cor:weierstrass-factorization} can be taken to be a polynomial of degree less than $\rho$ and the degree of the polynomials $R_n$ in \eqref{eq:weierstrass-canonical-product} can be taken to be independant of $n$. To prove this we want to study the connection between the order and zeros of an entire function.

We first recall a rather explicit connection between the moduli of the zeros of an analytic function and the modulus of the function itself:

\begin{theorem}[Jensen \cite{stein-shakarchi-princeton}] \label{thm:jensen}
    Let $f \in H(B_R(0))$ with $f(0) \neq 0$ and let $r_1, r_2, \hdots$ denote the moduli of the zeros of $f$ in $B_{R}(0)$ arranged in a non-decreasing sequence. Then, for $r_n < r < r_{n+1}$, we have
    \begin{equation}
        \frac{1}{2 \pi} \int_0^{2\pi} \log \vert f(r e^{i \theta}) \vert \diff \theta = \log \vert f(0) \vert + \log \frac{r^n}{r_1 \hdots r_n}.
    \end{equation}
\end{theorem}

\begin{definition}
    Let $f \in H(B_R(0))$. Then, for $0 < r < R$, we denote by $n_f(r)$ the number of zeros of $f$ in $\cl{B_{r}(0)}$.
\end{definition}

Jensen's Theorem has a useful equivalent form:

\begin{corollary} \label{cor:jensen-nf}
    Let $f \in H(B_R(0))$ with $f(0) \neq 0$. Then, for $0 < r < R$, we have
    \begin{equation}
        \frac{1}{2 \pi} \int_0^{2\pi} \log \vert f(re^{i \theta}) \vert \diff \theta = \log \vert f(0) \vert + \int_0^r \frac{n_f(s)}{s} \diff s
    \end{equation}
\end{corollary}

\begin{proof}
    Let $r_1, r_2, \hdots$ denote the moduli of the zeros of $f$ in $B_{R}(0)$ arranged in a non-decreasing sequence. Then, for any $r_n < r < r_{n+1}$, we obtain
    \begin{align*}
        \log \frac{r^n}{r_1 \hdots r_n} &= \sum_{k=1}^n \log \frac{r}{r_k} = \sum_{k=1}^n \int_{r_k}^r \frac{1}{s} \diff s = \\
        &= \sum_{k=1}^n \int_0^r \1_{(r_k, \infty)}(s) \frac{1}{s} \diff s = \int_0^r \left( \sum_{k=1}^n \1_{(r_k, \infty)}(s) \right) \frac{1}{s} \diff s = \\
        &= \int_0^r \frac{n_f(s)}{s} \diff s
    \end{align*}
    and \Cref{thm:jensen} concludes the claim.
\end{proof}

This gives an immediate connection between the zeros and the growth of the modulus of an entire function:

\begin{lemma} \label{lem:zeros-bounded-by-order}
    Let $f \in H(\C)$ be of finite order. Then for $\rho > \rho_f$ we have
    $$ n_f(r) = O(r^{\rho}), \quad \textrm{as} \quad r \to \infty. $$
\end{lemma}

\begin{proof}
    We may assume $f(0) \neq 0$. Let $r > 0$, then since $n_f$ is non-negative and non-decreasing we have
    \begin{align*}
        \int_0^{2r} \frac{n_f(s)}{s} \diff s \geq \int_r^{2r} \frac{n_f(s)}{s} \diff s \geq n_f(r) \int_r^{2r} \frac{1}{s} \diff s = n_f(r) (\log 2r - \log r) = n_f(r) \log 2.
    \end{align*}
    \Cref{cor:jensen-nf} now yields
    \begin{equation*}
        n_f(r) \log 2 \leq \int_0^{2r} \frac{n_f(s)}{s} \diff s = \log \vert f(0) \vert + \frac{1}{2\pi} \int_0^{2 \pi} \log \vert f(2r e^{i \theta}) \vert \diff \theta \leq \log \vert f(0) \vert + \log M_f(2r)
    \end{equation*}
    and holds for all $r > 0$ since $f$ is entire. Now, $f$ is of finite order, thus for any $\varepsilon > 0$ there is a constant $K_1 > 0$ such that for sufficiently large $r$ we have $M_f(r) \leq K_1 \exp {r^{\rho_f + \varepsilon}}$. Thus
    \begin{align*}
        n_f(r) \log 2 \leq \log \vert f(0) \vert + \log K_1 + (2r)^{\rho_f + \varepsilon}.
    \end{align*}
    and therefore $n_f(r) \leq K_2 r^{\rho_f + \varepsilon}$ for some constant $K_2 > 0$ and sufficiently large $r$.
\end{proof}

The result above shows that the more zeros a function $f$ has, the faster $M_f$ must grow as $r \to \infty$. The converse is naturally false; by composing exponentials one can obtain an entire function $f$ for which $M_f$ grow arbitrarily fast, yet $f$ has no zeros.

\begin{definition} \label{def:zero-exponent}
    Let $f \in H(\C)$ and denote with $(r_j)_{j \in \N}$ the non-zero moduli of its zeros, arranged in non-decreasing order. Then
    $$ \lambda_f \coloneqq \inf \left\{ \lambda > 0 : \sum_{n=1}^\infty \frac{1}{r^\lambda_n} < \infty \right\} $$
    is called the \emph{exponent of convergence of the zeros of f}. If $f$ has finitely many zeros, then we set $\lambda_f = 0$ by convention.

    Furthermore, for any $a \in \C$, the \emph{exponent of convergence of the $a$-points of $f$} is defined as exponent of convergence of zeros of $f(z) - a$.
\end{definition}

\begin{theorem} \label{thm:inequality-order-exponent-of-convergence}
    If $f \in H(\C)$ is of finite order, then $\lambda_f \leq \rho_f$.
\end{theorem}

\begin{proof}
    We may assume $f(0) \neq 0$. Let $\varepsilon > 0$ then \Cref{lem:zeros-bounded-by-order} yields a constant $K_1 > 0$ such that for sufficiently large $r$ we have $n_f(r) \leq K_1 r^{\rho_f + \varepsilon}$. If $(r_j)_{j \in \N}$ denote the non-zero moduli of its zeros, arranged in non-decreasing order, then for any $m \in \N$ we have $m = n_f(r_m) \leq K_1 r_m^{\rho_f + \varepsilon}$. Let $\delta > 0$ then this implies
    \begin{align*}
        \left( \frac{1}{r_m} \right)^{\lambda_f - \delta} \leq K_2 \left( \frac{1}{m} \right)^{(\lambda_f - \delta)/(\rho_f + \varepsilon)}.
    \end{align*}
    for some constant $K_2 > 0$. Therefore
    \begin{equation*}
        \sum_{m=1}^\infty \left( \frac{1}{r_m} \right)^{\lambda_f - \delta} \leq K_2 \sum_{m=1}^\infty \left( \frac{1}{m} \right)^{(\lambda_f - \delta)/(\rho_f + \varepsilon)}.
    \end{equation*}
    The left-hand side diverges by \Cref{def:zero-exponent}, thus so does the right-hand side. But the latter diverges if and only if $\frac{\lambda_f - \delta}{\rho_f + \varepsilon} \leq 1$, therefore letting $\delta \to 0$ and then $\varepsilon \to 0$ yields $\lambda_f \leq \rho_f$.
\end{proof}

\begin{remark}
    The function $f(z) \coloneqq \exp z$ is of order one and has no zeros -- thus we observe that we may have $\lambda_f < \rho_f$ in some cases.
\end{remark}

\begin{example} \label{exm:exponent-of-convergence}
    Consider $f(z) \coloneqq \sin(z) \in H(\C)$, we want to calculate $\lambda_f$ and $\rho_f$. First, let $\lambda > 0$ and recall that $f$ has zeros at $(n \pi)_{n \in \Z}$. Since
    $$ \sum_{n \in \Z \setminus \{0\}} \frac{1}{\vert n \pi \vert^\lambda} = \frac{2}{\pi^\lambda} \sum_{n=1}^\infty \frac{1}{n^\lambda} $$
    is finite if and only if $\lambda > 1$, we obtain $\lambda_f = 1$. Furthermore, we have $\vert \sin(z) \vert \leq e^{\vert z \vert}$
    and therefore $M_f(r) \leq e^r$, whereby $\rho_f \leq 1$. Finally, \Cref{thm:inequality-order-exponent-of-convergence} concludes $\rho_f = 1$.
\end{example}

The final key to proving Hadamard's Theorem is the following lemma, which can be considered as a version of the maximum modulus principle applied to the real part of a holomorphic function.

\begin{lemma}[Borel-Carathéodory] \label{lem:borel-caratheodory}
    Let $G$ be a domain, $R > 0$ and suppose $\cl{B_R(0)} \subseteq G$ and $f \in H(G)$. Define $ A_f(r) \coloneqq \max_{\vert z \vert = r} \Re f(z) $, then, for $0 < r < R$,
    \begin{equation} \label{eq:borel-caratheodory-1}
        M_f(r) \leq \frac{2r}{R - r} A_f(R) + \frac{R + r}{R - r} \vert f(0) \vert
    \end{equation}
    and, if additionally $A_f(R) \geq 0$, then for any $n \in \N$
    \begin{equation} \label{eq:borel-caratheodory-2}
        M_{f^{(n)}}(r) \leq \frac{2^{n+2} n! R}{(R - r)^{n+1}} (A_f(R) + \vert f(0) \vert).
    \end{equation}
\end{lemma}

\begin{proof}
    If $f$ is constant there is nothing to show.

    We assume $f$ being non-constant. First, for $r > 0$ we have
    \begin{align*}
        A_f(r) = \max_{\vert z \vert = r} \Re f(z) = \log \max_{\vert z \vert = r} \exp {\Re f(z)} = \log \max_{\vert z \vert = r} \vert \exp f(z) \vert = \log M_{\exp f}(r)
    \end{align*}
    and since $M_f$ is strictly increasing and continuous therefore so is $A_f$.

    We will show (\ref{eq:borel-caratheodory-1}) by considering two cases. First assume $f$ is non-constant and $f(0) = 0$. Then by the above we have $A_f(R) > A_f(0) = 0$. Define
    \begin{equation*}
        \phi(z) \coloneqq \frac{f(z)}{2 A_f(R) - f(z)} \in H(B_R(0)).
    \end{equation*}
    Note that we have $\phi(0) = 0$ and
    \begin{align*}
        \vert \phi(z) \vert^2 &= \phi(z) \overline{\phi(z)} = \frac{\vert f(z) \vert^2}{(2 A_f(R) - f(z))(2 A_f(R) - \overline{f(z)})} = \\
        &= \frac{\vert f(z) \vert^2}{(2 A_f(R) - \Re f(z))^2 + \vert f(z) \vert^2 - (\Re f(z))^2} \leq 1,
    \end{align*}
    since clearly $2 A_f(R) - \Re f(z) \geq \Re f(z)$. Now, since $\phi(zR) \in H(\D)$ and $\vert \phi(zR) \vert \leq 1$ for $z \in \D$, Schwartz' Lemma implies $\vert \phi(zR) \vert \leq \vert z \vert$ for $z \in \D$. Therefore, for $z \in B_R(0)$, we have $\vert \phi (z) \vert \leq \frac{\vert z \vert}{R}$, and for $z \in B_r(0)$ we have $\vert \phi (z) \vert < \frac{r}{R} < 1$. Since
    \begin{equation*}
        f(z) = \frac{2 A_f(R) \phi(z)}{1 + \phi(z)}
    \end{equation*}
    we obtain
    \begin{align*}
        \vert f(z) \vert &= 2 A_f(R) \left\vert \sum_{n=1}^\infty (-1)^n \phi(z)^n \right\vert \leq 2 A_f(R) \sum_{n=1}^\infty \left( \frac{r}{R} \right)^n = \frac{2 r}{R - r} A_f(R).
    \end{align*}

    Now assume that $f$ is non-constant and $f(0) \neq 0$. Set $g(z) \coloneqq f(z) - f(0)$, then by the above we have, for $w \in B_r(0)$,
    \begin{align*}
        \vert f(w) \vert - \vert f(0) \vert \leq \vert g(w) \vert \leq M_g(r) \leq \frac{2 r}{R - r} A_g(R) \leq \frac{2r}{R - r} A_f(R) - \frac{2r}{R - r} \Re f(0).
    \end{align*}
    Since $-\Re f(0) \leq \vert f(0) \vert$, this implies
    \begin{align*}
        \vert f(w) \vert \leq \frac{2r}{R - r}A_f(R) + \frac{2r}{R - r} \vert f(0) \vert + \vert f(0) \vert \leq \frac{2r}{R - r} A_f(R) + \frac{R + r}{R - r} \vert f(0) \vert,
    \end{align*}
    thus proving (\ref{eq:borel-caratheodory-1}).

    To show (\ref{eq:borel-caratheodory-2}), let $z \in \partial B_r(0)$ and set $s \coloneqq \frac{R - r}{2}$, then by Cauchy's integral formula
    \begin{equation*}
        f^{(n)}(z) = \frac{n!}{2 \pi i} \oint_{\partial B_s(z)} \frac{f(\zeta)}{(\zeta - z)^{n+1}} \diff \zeta.
    \end{equation*}
    Applying (\ref{eq:borel-caratheodory-1}) with $r = \frac{R + r}{2} < R$ we get
    \begin{align*}
        \vert f^{(n)}(z) \vert &\leq \frac{n!}{2 \pi} 2 \pi \frac{R - r}{2} \left( \frac{2}{R - r} \right)^{n+1} \left( \frac{2 (R + r)/2}{R - (R+r)/2} A_f(R) + \frac{R + (R + r)/2}{R - (R + r)/2} \vert f(0) \vert \right) \leq \\
        &\leq \frac{2^{n+1} n!}{(R - r)^{n+1}} \frac{R - r}{2} \left( 2 \frac{R+r}{R-r} A_f(R) + \frac{3R + r}{R - r} \vert f(0) \vert \right) \leq \\
        &\leq \frac{2^{n+1} n!}{(R - r)^{n+1}} \left( (R+r) A_f(R) + \frac{3R + r}{2} \vert f(0) \vert \right) \leq \frac{2^{n+2} n! R}{(R - r)^{n+1}} (A_f(R) + \vert f(0) \vert).
    \end{align*}
\end{proof}

\begin{theorem}[Hadamard] \label{thm:hadamard}
    Let $f \in H(\C)$ be of finite order with zeros $(z_j)_{j \in \N}$. Then there exists a polynomial $Q$ with $\deg Q \leq \rho_f$, such that
    $$ f(z) = e^{Q(z)} E(z), $$
    where $E$ is a Weierstrass canonical product formed from $(z_j)_{j \in \N}$.
\end{theorem}

\begin{proof}
    \todo{TODO.}
\end{proof}