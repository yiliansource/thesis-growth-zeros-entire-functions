\section{Hadamard's Theorem}
\label{sec:hadamards-theorem}

\begin{theorem}[Weierstrass \cite{bluemlinger-complex-analysis,stein-shakarchi-princeton}] \label{thm:weierstrass}
    Let $(z_j)_{j \in \N}$ be a sequence in $\C$ without accumulation points. Then there exists an entire function $E$ (called the \emph{Weierstrass canonical product} formed from said sequence) that has zeros precisely at $(z_j)_{j \in \N}$, with multiplicities equal to how often $z_j$ occurs in the sequence.
\end{theorem}

In particular, we have
\begin{equation} \label{eq:weierstrass-canonical-product}
    E(z) = z^k \prod_{n=1}^\infty \left( 1 - \frac{z}{z_n} \right) e^{R_n(z / z_n)},
\end{equation}
where $k$ is the order of the zero at $z = 0$ and $R_n$ is a polynomial, namely a truncation of the power series for $-\log(1 - \frac{z}{z_n})$ chosen of smallest degree to ensure convergence of the product~\cite{segal-complex-analysis}.

Weierstrass' Theorem (\ref{thm:weierstrass}) is also known as the \emph{Weierstrass Factorization Theorem}, due to the following corollary:

\begin{corollary} \label{cor:weierstrass-factorization}
    Let $f$ be an entire function with zeros $(z_j)_{j \in \N}$. Then there exists an entire function $g$, such that
    $$ f(z) = e^{g(z)} E(z), $$
    where $E$ is a Weierstrass canonical product formed from the zeros of $f$.
\end{corollary}

\begin{proof}
    Since $f / E$ has removable singularities at all $(z_j)_{j \in \N}$, we have that $f / E$ is entire and nowhere $0$. Thus there exists an entire function $g$ with $f / E = e^g$, which yields $f = e^g E$.
\end{proof}

Hadamard's Theorem will show that, for functions of finite order $\rho$, the function $g$ in \Cref{cor:weierstrass-factorization} can be taken to be a polynomial of degree less than $\rho$ and the degree of the polynomials $R_n$ in \cref{eq:weierstrass-canonical-product} can be taken to be independant of $n$. To prove this we require the following lemma, which can be interpreted as a version of the maximum modulus principle applied to the real part of an analytic function.

\begin{lemma}[Borel-Carathéodory] \label{lem:borel-caratheodory}
    Let $f$ be analytic in $\cl{\D}$ and let
    $$ A_f(r) = \max_{\vert z \vert = r} \Re f(z). $$
    Then, for $0 < r < R$,
    $$ M_f(r) \leq \frac{2r}{R - r} A_f(R) + \frac{R + r}{R - r} \vert f(0) \vert $$
    and, if additionally $A_f(R) \geq 0$, then for $n \in \N$
    $$ \max_{\vert z \vert = r} \vert f^{(n)}(z) \vert \leq \frac{2^{n+2} n! R}{(R - r)^{n+1}} (A_f(R) + \vert f(0) \vert). $$
\end{lemma}

\begin{proof}
    \todo{TODO.}
\end{proof}

\begin{theorem}[Hadamard] \label{thm:hadamard}
    Let $f$ be an entire function of finite order with zeros $(z_j)_{j \in \N}$. Then there exists a polynomial $Q$ with $\deg Q \leq \rho_f$, such that
    $$ f(z) = e^{Q(z)} E(z), $$
    where $E$ is a Weierstrass canonical product formed from the zeros of $f$.
\end{theorem}

\begin{proof}
    \todo{TODO.}
\end{proof}