\section{Zeros Distribution}
\label{sec:zeros}

For Weierstrass canonical products the exponent of convergence of zeros and the order are even more closely related:

\begin{theorem} \label{thm:exponent-of-convergence-weierstrass-product}
    Let $E \in H(\C)$ be a Weierstrass canonical product formed from $(z_j)_{j \in \N}$. If $E$ is of finite order, then $\lambda_E = \rho_E$.
\end{theorem}

\begin{proof}
    \todo{TODO.}
\end{proof}

This result allows us to prove easily prove two results regarding functions of finite, non-integer order.

\begin{theorem} \label{thm:finite-non-integer-order-equals-exponent-of-convergence}
    Let $f \in H(\C)$ be of finite, non-integer order. Then $\rho_f = \lambda_f$.
\end{theorem}

\begin{proof}
    By \Cref{thm:inequality-order-exponent-of-convergence} we have $\lambda_f \leq \rho_f$. Invoking Hadamard's Theorem we can write $f = e^Q E$ for a polynomial $Q$ with $\deg Q \leq \rho_f$. Since $\rho_f$ is not an integer, this implies $\deg Q \leq \lfloor \rho_f \rfloor < \rho_f$. As seen in \Cref{rem:order-exponential-polynomial} $e^Q$ has order $\deg Q$ and by \Cref{thm:exponent-of-convergence-weierstrass-product} $E$ has order $\lambda_f$. Using \Cref{prop:order-sum-product-estimate} we obtain
    $$ \rho_f \leq \max \{ \deg Q, \lambda_f \} = \lambda_f \leq \rho_f, $$
    since $\rho_f \leq \max \{ \deg Q, \lambda_f \} = \deg Q < \rho_f$ is a contradiction, and we get $\rho_f = \lambda_f$.
\end{proof}

\begin{theorem} \label{thm:finite-non-integer-order-infinite-zeros}
    Let $f \in H(\C)$ be of finite, non-integer order. Then $f$ has infinitely many zeros.
\end{theorem}

\begin{proof}
    By \Cref{thm:finite-non-integer-order-equals-exponent-of-convergence} we have $\rho_f = \lambda_f$. Since $\rho_f$ is not an integer, $\lambda_f > 0$, which implies that $f$ has infinitely many zeros.
\end{proof}

\todo{Maybe introduce Borel exceptional values as a definition? But then again, I will never need them again. Maybe also add a remark on the relation to lacunary values (Picard).}

\begin{theorem}[Borel] \label{thm:existence-borel-exceptional-values}
    \todo{Existence of Borel exceptional values.}
\end{theorem}

\begin{proof}
    \todo{TODO.}
\end{proof}

\todo{Zeros distribution (upper density, etc.).}