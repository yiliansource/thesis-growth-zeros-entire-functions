\section{Order and Type}
\label{sec:order-type}

\begin{definition} \label{def:order}
    Let $f$ be an entire function. The \emph{order} of $f$ is defined by
    \begin{equation} \label{eq:def-order}
        \rho_f \coloneqq \limsup_{r \to \infty} \frac{\log \log M_f(r)}{\log r}.
    \end{equation}
    Constant functions, by convention, have order 0.
\end{definition}

Note that, for any entire function $f$, we have $0 \leq \rho_f \leq \infty$. Furthermore, $\rho_f < \infty$ if and only if $M_f(r) = O(\exp(r^{\rho_f + \varepsilon}))$ for every $\varepsilon > 0$ and no $\varepsilon < 0$, as $r \to \infty$~\cite{segal-complex-analysis}.

An elementary fact regarding the order of the sum or product of two entire functions of finite order can immediately be obtained by naively estimating their respective maximum modulus:

\begin{proposition} \label{prop:algebraic-properties-order}
    Let $f, g$ be entire functions of finite order. Then it holds that:
    \begin{enumerate}[i.]
        \item $\rho_{f + g} \leq \max \{ \rho_f, \rho_g \}$
        \item $\rho_{f g} \leq \max \{ \rho_f, \rho_g \}$
    \end{enumerate}
\end{proposition}

\begin{proposition} \label{prop:order-derivative}
    Let $f$ be an entire function of finite order, then $\rho_{f'} = \rho_f$.
\end{proposition}

\begin{proof}
    \todo{TODO.}
\end{proof}

For functions of finite and positive order, we can obtain a natural refinement of the concept of order:

\begin{definition}
    Let $f$ be an entire function of finite and positive order. The \emph{type} of $f$ is defined by
    \begin{equation} \label{eq:def-type}
        \tau_f \coloneqq \limsup_{r \to \infty} \frac{\log M_f(r)}{r^{\rho_f}}
    \end{equation}
\end{definition}

For any entire function $f$ with $0 < \rho_f < \infty$, we have $0 < \tau_f \leq \infty$. Additionally, we have $0 < \rho_f < \infty$ and $\tau_f < \infty$ if and only if $M_f(r) = O(\exp((\tau_f + \varepsilon) r^{\rho_f}))$ for every $\varepsilon > 0$ and no $\varepsilon < 0$, as $r \to \infty$~\cite{segal-complex-analysis}.

Imitating the proof of \Cref{prop:order-derivative}, we immediately get:

\begin{proposition} \label{prop:type-derivative}
    Let $f$ be an entire function of finite type, then $\tau_{f'} = \tau_f$.
\end{proposition}

\begin{definition}
    Let $f$ be an entire function. Then $f$ is said to be of \emph{growth} $(a, b)$ if
    \begin{itemize}
        \item $\rho_f < a$, or
        \item $\rho_f = a$ and $\tau_f \leq b$.
    \end{itemize}
\end{definition}

\begin{example}
    \todo{Maybe give some functions together with their order and type?}
\end{example}