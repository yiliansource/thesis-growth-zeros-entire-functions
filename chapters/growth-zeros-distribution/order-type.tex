\section{Order and Type}
\label{sec:order-type}

To study the rate of growth of an entire function we first introduce the following:
\begin{definition}
    Let $f \in H(\C)$, then for $r \geq 0$ we define
    \begin{equation*}
        M_f(r) \coloneqq \max_{\vert z \vert = r} \vert f(z) \vert.
    \end{equation*}
\end{definition}

By the maximum modulus principle, $M_f$ is monotonically increasing. We shall show it is continuous: Let $\varepsilon > 0$ and choose $\delta_\varepsilon > 0$ such that for ${\vert z_1 - z_2 \vert < \delta_\varepsilon}$ we have ${\left\vert \vert f(z_1) \vert - \vert f(z_2) \vert \right\vert < \varepsilon}$. Now let $r_1 < r_2$ and choose $\theta$ such that $M_f(r_2) = \vert f(r_2 e^{i \theta}) \vert$. Then
\begin{equation*}
    0 \leq M_f(r_2) - M_f(r_1) \leq \vert f(r_2 e^{i \theta}) \vert - \vert f(r_1 e^{i \theta}) \vert < \varepsilon,
\end{equation*}
for $r_2 - r_1 < \delta_\varepsilon$.

\begin{definition} \label{def:order}
    Let $f \in H(\C)$. The \emph{order} of $f$ is defined by
    \begin{equation} \label{eq:def-order}
        \rho_f \coloneqq \limsup_{r \to \infty} \frac{\log \log M_f(r)}{\log r}.
    \end{equation}
    Constant functions, by convention, have order 0.
\end{definition}

Note that, for any entire function $f$, we have $0 \leq \rho_f \leq \infty$.

For functions of finite order, we have an equivalent characterization:

\begin{proposition} \label{prop:order-infimum}
    Let $f \in H(\C)$, then $f$ is of finite order if and only if
    \begin{equation}
        \rho \coloneqq \inf \{ s > 0 : M_f(r) = O(\exp r^s) \textrm{ as } r \to \infty \}
    \end{equation}
    is finite, and in either case we have $\rho_f = \rho$.
\end{proposition}

\begin{proof}
    Suppose $0 \leq \rho < \infty$, then for all $\varepsilon > 0$ we have $M_f(r) = O(\exp r^{\rho + \varepsilon})$ as $r \to \infty$. Thus there exists a constant $K_\varepsilon > 0$ (we may assume $K_\varepsilon > 1$) and an $r_0 > 0$ such that for all $r \geq r_0$ we have $M_f(r) \leq K_\varepsilon \exp r^{\rho+\varepsilon}$. Using the fact that $\log (a+b) = \log a + \log (1 + \frac{b}{a})$ we get
    \begin{align*}
        \frac{\log \log M_f(r)}{\log r} &\leq \frac{\log (r^{\rho + \varepsilon} + \log K_\varepsilon)}{\log r} = \rho + \varepsilon + \frac{\log (1 + \log (K_\varepsilon) / r^{\rho + \varepsilon})}{\log r}. \tag{\textasteriskcentered} \label{eq:proof-order-infimum-1}
    \end{align*}
    Since
    \begin{align*}
        0 \leq \frac{\log (1 + \log (K_\varepsilon) / r^{\rho + \varepsilon})}{\log r} \leq \frac{\log (K_\varepsilon) / r^{\rho + \varepsilon}}{\log r} = \frac{\log K_\varepsilon}{r^{\rho + \varepsilon} \log r} \xrightarrow{r \to \infty} 0,
    \end{align*}
    by taking the limit superior as $r \to \infty$ in \eqref{eq:proof-order-infimum-1} and then letting $\varepsilon \to 0$ we obtain $\rho_f \leq \rho$.

    Now suppose $0 \leq \rho_f < \infty$ and let $\varepsilon > 0$. Then, by definition of the limit superior, there is an $r_0 > 0$ such that for all $r \geq r_0$ we have $ \log \log M_f(r) \leq (\rho_f + \varepsilon) \log r$ and therefore $M_f(r) \leq \exp(r^{\rho_f + \varepsilon})$. Thus $M_f(r) = O(\exp r^{\rho_f + \varepsilon})$, thereby $\rho \leq \rho_f + \varepsilon$ and letting $\varepsilon \to 0$ yields $\rho \leq \rho_f$.
\end{proof}

\begin{remark} \label{rem:order-zero}
    Any polynomial is of order zero. Indeed, let $P(z) = \sum_{k=0}^n a_k z^k$ be a polynomial and $m \in \N$. Then for any $r > 0$ we have
    $$ \exp r^{1/m} = \sum_{k=0}^\infty \frac{r^{k/m}}{k!} > \frac{r^{n}}{(mn)!} \geq \frac{r^{n}}{(mn)!} $$
    and therefore
    $$ M_P(r) = \max_{\vert z \vert = r} \vert P(z) \vert \leq n \max_{k = 0, \hdots, n} \vert a_k \vert r^n \leq \left( n \max_{k = 0, \hdots, n} \vert a_k \vert (mn)! \right) \exp r^{1/m}. $$
    Since $m \in \N$ was arbitrary, \Cref{prop:order-infimum} gives $\rho_P = 0$.

    There are however non-polynomial functions of order zero; one must simply construct an entire function of sufficiently rapid decay. Consider the entire function
    $$ f(z) \coloneqq \sum_{k=0}^\infty \frac{z^k}{(k^2)!} $$
    and let $m \in \N$. By the above we have $\sum_{k=0}^{m-1} \frac{r^k}{(k^2)!} \leq K \exp r^{1/m}$ for some constant $K_m > 0$ depending only on $m$ and all $r > 0$. Thus
    $$ \sum_{k=0}^\infty \frac{r^k}{(k^2)!} \leq K_m \exp r^{1/m} + \sum_{k=m}^\infty \frac{r^k}{(km)!} \leq K_m + \sum_{k=0}^\infty \frac{r^{k/m}}{k!} \leq (K_m + 1) \exp r^{1/m} $$
    and we obtain $\rho_f = 0$, as above.
\end{remark}

\begin{remark}\label{rem:order-exponential-polynomial}
    If $Q(z) = \sum_{k=0}^n a_k z^k$ is a polynomial we claim that $f(z) \coloneqq \exp {Q(z)}$ is of order $n$. Let $\varepsilon > 0$, then since $\lim_{r \to \infty}(\sum_{k=0}^n \vert a_k \vert r^k) / r^{n + \varepsilon} = 0$ there is some $r_0$ such that for all $r > r_0$ we have $(\sum_{k=0}^n \vert a_k \vert r^k) / r^{n + \varepsilon} < 1$ and thus
    \begin{align*}
        M_f(r) \leq \exp \sum_{k=0}^n \vert a_n \vert r^k \leq \exp(r^{n + \varepsilon}),
    \end{align*}
    showing $\rho_f \leq n + \varepsilon$. On the other hand, since
\end{remark}

\begin{proposition} \label{prop:order-sum-product-estimate}
    Let $f, g \in H(\C)$ be of finite order. Then it holds that:
    \begin{enumerate}[i.]
        \item $\rho_{f + g} \leq \max \{ \rho_f, \rho_g \}$, with equality holding if $\rho_f \neq \rho_g$.
        \item $\rho_{fg} \leq \max \{ \rho_f, \rho_g \}$.
    \end{enumerate}
\end{proposition}

\begin{proof}
    Let $\varepsilon > 0$, then by \Cref{prop:order-infimum} there are constants $C_f, C_g, r_f, r_g > 0$ such that for $r > r_f$ we have $M_f(r) \leq A \exp(r^{\rho_f + \varepsilon})$ and for $r > r_g$ we have $M_g(r) \leq C_g \exp(r^{\rho_g + \varepsilon})$. Thus, for $r > \max \{ r_f, r_g \}$, we obtain
    \begin{align*}
        M_{f + g}(r) &\leq 2 \max \{ M_f(r), M_g(r) \} \leq 2 \max \{ C_f \exp (r^{\rho_f + \varepsilon} ), C_g \exp (r^{\rho_g + \varepsilon} ) \} \leq \\
        &\leq 2 \max \{ C_f, C_g \} \exp (r^{\max \{ \rho_f, \rho_g \} + \varepsilon}),
    \end{align*}
    thus $M_{f + g}(r) = O(\exp r^{\max \{ \rho_f, \rho_g \} + \varepsilon})$ for all $\varepsilon > 0$, proving $\rho_{f + g} \leq \max \{ \rho_f, \rho_g \}$. Similarly, for $r > \max \{ r_f, r_g, 2^{1/\varepsilon} \}$, we have
    \begin{align*}
        M_{fg}(r) &\leq \max \{ M_f(r), M_g(r) \}^2 \leq \max \{ C_f \exp (r^{\rho_f + \varepsilon} ), C_g \exp (r^{\rho_g + \varepsilon} ) \}^2 \leq \\
        &\leq \max \{ C_f, C_g \}^2 \exp (2 r^{ \max \{ \rho_f, \rho_g \} + \varepsilon}) \max \{ C_f, C_g \}^2 \exp (r^{ \max \{ \rho_f, \rho_g \} + 2\varepsilon})
    \end{align*}
    which shows $\rho_{fg} \leq \max \{ \rho_f, \rho_g \}$.

    Now assume $\rho_f \neq \rho_g$, without loss of generality $\rho_f < \rho_g$. By the above we have
    \begin{equation*}
        \rho_{f+g} \leq \max \{ \rho_f, \rho_g \} = \rho_g, \quad\textrm{and}\quad \rho_g = \rho_{f+g+(-f)} \leq \max \{ \rho_f, \rho_{f+g} \} = \rho_{f+g},
    \end{equation*}
    since if $\rho_f > \rho_{f+g}$ then $\rho_f > \rho_{f+g} \geq \rho_g$ is a contradiction. Thus $\rho_{f+g} = \rho_f$, proving the assertion.

    % If $f$ and $g$ are polynomials the inequality clearly holds, since all occurring orders are zero. Otherwise let $S(r) \coloneqq \max \{ M_f(r), M_g(r) \}$, then
    % \begin{align*}
    %     \frac{\log \log M_{f + g}(r)}{\log r} &\leq \frac{\log \log (2 S(r))}{\log r} = \frac{\log (\log S(r) + \log 2)}{\log r} = \\
    %     &= \frac{\log \log S(r)}{\log r} + \frac{\log (1 + \log 2 / \log S(r))}{\log r}. \tag{\textasteriskcentered} \label{eq:order-sum-estimate-1}
    % \end{align*}
    % But, for sufficiently large $r$,
    % \begin{equation*}
    %     0 \leq \frac{\log (1 + \log 2 / \log S(r))}{\log r} \leq \frac{\log 2 / \log S(r)}{\log r} = \frac{\log 2}{\log r \log S(r)} \xrightarrow{r \to \infty} 0,
    % \end{equation*}
    % thus taking the limit superior in \eqref{eq:order-sum-estimate-1} yields
    % \begin{equation*}
    %     \rho_{f + g} \leq \max \left\{ \limsup_{r \to \infty} \frac{\log \log M_f(r)}{\log r}, \limsup_{r \to \infty} \frac{\log \log M_g(r)}{\log r} \right\} = \max \{ \rho_f, \rho_g \}.
    % \end{equation*}
\end{proof}

It is of note that \Cref{prop:order-sum-product-estimate} implies that the order of an entire function of finite order remains unchanged when adding a polynomial of any degree to it.

A slightly deeper result is that, for entire functions of finite order, the order is invariant under derivatives:

\begin{proposition} \label{prop:order-derivative}
    If $f \in H(\C)$ is of finite order, then $\rho_{f'} = \rho_f$.
\end{proposition}

\begin{proof}
    Without loss of generality we may assume $f(0) = 0$. If $f$ is a polynomial the assertion is clear. Otherwise for any $r > 0$ we have, using the maximum modulus principle,
    \begin{align*}
        M_f(r) &= \max_{\vert z \vert = r} \vert f(z) \vert = \max_{\vert z \vert = r} \left\vert \int_{[0, z]} f'(\zeta) \diff \zeta \right\vert \leq \\
        &\leq \max_{\vert z \vert = r} \left( \vert z \vert \max_{w \in \cl{B_r(0)}} \vert f'(w) \vert \right) + \leq r M_{f'}(r).
    \end{align*}
    By Cauchy's integral formula we have
    \begin{align*}
        M_{f'}(r) &= \max_{z \in \partial B_r(0)} \vert f'(z) \vert = \max_{z \in \partial B_r(0)} \left\vert \frac{1}{2\pi i} \int_{\partial B_r(z)} \frac{f(\zeta)}{(\zeta-z)^2} \diff \zeta \right\vert \leq \\
        &\leq \frac{1}{2 \pi} 2 \pi r \max_{\substack{z \in \partial B_r(0) \\ w \in \partial B_r(z)}} \vert f(w) \vert \frac{1}{r^2} \leq \frac{M_f(2r)}{r}.
    \end{align*}
    Combining the above we get
    \begin{equation*}
        M_f(r) \leq r M_{f'}(r) \leq M_f(2r).
    \end{equation*}
    Applying the logarithm twice and dividing by $\log r$ thus yields, for sufficiently large $r$,
    \begin{align*}
        \frac{\log \log M_f(r)}{\log r} \leq \frac{\log \log (r M_{f'}(r))}{\log r} \leq \frac{\log \log M_f (2r)}{\log 2r} \frac{\log 2r}{\log r}. \tag{\textasteriskcentered} \label{eq:proof-order-derivative-1}
    \end{align*}
    But we have
    \begin{align*}
        \frac{\log \log (r M_{f'}(r))}{\log r} = \frac{\log (\log M_{f'}(r) + \log r)}{\log r} = \frac{\log \log M_{f'}(r)}{\log r} + \frac{\log(1 + \log r / \log M_{f'}(r))}{\log r}
    \end{align*}
    and
    \begin{align*}
        0 \leq \frac{\log(1 + \log r / \log M_{f'}(r))}{\log r} \leq \frac{\log r / \log M_{f'}(r)}{\log r} = \frac{1}{\log M_{f'}(r)} \xrightarrow{r \to \infty} 0,
    \end{align*}
    therefore by taking the limit superior as $r \to \infty$ in \eqref{eq:proof-order-derivative-1} we obtain
    \begin{align*}
        \rho_f = \limsup_{r \to \infty} \frac{\log \log (r M_{f'}(r))}{\log r} = \limsup_{r \to \infty} \frac{\log \log M_{f'}(r)}{\log r} = \rho_{f'},
    \end{align*}
    concluding the claim.
\end{proof}

For functions of finite and positive order, we can obtain a natural refinement of the concept of order:

\begin{definition}
    Let $f \in H(\C)$ be of order $0 < \rho_f < \infty$. The \emph{type} of $f$ is defined by
    \begin{equation} \label{eq:def-type}
        \tau_f \coloneqq \limsup_{r \to \infty} \frac{\log M_f(r)}{r^{\rho_f}}
    \end{equation}
\end{definition}

For any $f \in H(\C)$ with $0 < \rho_f < \infty$, we have $0 \leq \tau_f \leq \infty$.

Once again, we have an equivalent characterization for functions of finite order and type:

\begin{proposition} \label{prop:type-infimum}
    Let $f \in H(\C)$ be of finite order, then $f$ is of finite type if and only if
    \begin{equation}
        \tau \coloneqq \inf \{ t > 0 : M_f(r) = O(\exp (t r^{\rho_f})) \textrm{ as } r \to \infty \}
    \end{equation}
    is finite, and in either case we have $\tau_f = \tau$.
\end{proposition}

\begin{proof}
    Suppose $0 \leq \tau < \infty$, then for all $\varepsilon > 0$ we have $M_f(r) = O(\exp (\tau + \varepsilon) r^{\rho_f})$ as $r \to \infty$. Thus there exists a constant $K_\varepsilon > 0$ and an $r_0 > 0$ such that for all $r \geq r_0$ we have $M_f(r) \leq K_\varepsilon \exp (\tau + \varepsilon) r^{\rho_f}$. Therefore
    \begin{align*}
        \frac{\log M_f(r)}{r^{\rho_f}} \leq \frac{\log K_\varepsilon + (\tau + \varepsilon) r^{\rho_f}}{r^{\rho_f}} = \frac{\log K_\varepsilon}{r^{\rho_f}} + \tau + \varepsilon
    \end{align*}
    and taking the limit superior as $r \to \infty$ and letting $\varepsilon \to 0$ afterwards yields $\tau_f \leq \tau$.

    Now suppose $0 \leq \tau_f < \infty$ and let $\varepsilon > 0$. Then, by definition of the limit superior, there is an $r_0 > 0$ such that for all $r \geq r_0$ we have $ \log M_f(r) \leq (\tau_f + \varepsilon) r^{\rho_f}$ and therefore $M_f(r) \leq \exp((\tau_f + \varepsilon) r^{\rho_f})$. Thus $M_f(r) = O(\exp ((\tau_f + \varepsilon) r^{\rho_f}))$, thereby $\tau \leq \tau_f + \varepsilon$ and letting $\varepsilon \to 0$ yields $\tau \leq \tau_f$.
\end{proof}

% Equivalently, $\tau_f$ can also be defined as the infimum over all $\tau > 0$ such that $M_f(r) = O(\exp(\tau r^{\rho_f}))$ as $r \to \infty$.

% For functions of finite order not only their order is invariant under derivatives, their type is as well:

% \begin{proposition} \label{prop:type-derivative}
%     If $f \in H(\C)$ is of finite order and type, then $\tau_{f'} = \tau_f$.
% \end{proposition}

% There are some 

\begin{definition}
    Let $f \in H(\C)$. Then $f$ is said to be of \emph{growth} $(a, b)$ if
    \begin{itemize}
        \item $\rho_f < a$, or
        \item $\rho_f = a$ and $\tau_f \leq b$.
    \end{itemize}
\end{definition}

\begin{example} \label{exm:order-and-type}
    For $\rho, \tau \in (0, \infty)$, the function
    $$ f(z) \coloneqq \exp {\tau z^\rho} $$
    is of order $\rho$ and type $\tau$.

    We have already seen in \Cref{rem:order-zero} that polynomials are of order zero.

    The function
    $$ f(z) \coloneqq \exp \exp z $$
    is of infinite order, as seen by \todo{Todo.}
\end{example}