\chapter{Order}
\label{ch:order}

\begin{definition} \label{def:order}
    Let $f$ be an entire function. The \emph{order} of $f$ is defined by
    \begin{equation} \label{eq:def-order}
        \rho_f \coloneqq \limsup_{r \to \infty} \frac{\log \log M_f(r)}{\log r}.
    \end{equation}
    Constant functions, by convention, have order 0.
\end{definition}

\begin{remark} \label{rem:order}
    \todo{Initial explanation and intuition of the order. Make sure to note the possible values of the order ($0 \leq \rho \leq \infty$). And that $\rho$ can also be seen as the infimum over all $\rho$ that satisfy $\vert f(z) \vert \leq A e^{B \vert z \vert^\rho}$ for suitable $A, B > 0$ and sufficiently large $\vert z \vert$.}

    \todo{Maybe this should not be a remark, but rather an equivalent characterization?}
\end{remark}

\begin{proposition} \label{prop:algebraic-properties-order}
    Let $f, g$ be entire functions of finite order. Then it holds that:
    \begin{enumerate}[i.]
        \item $\rho_{f + g} \leq \max \{ \rho_f, \rho_g \}$
        \item $\rho_{f g} \leq \max \{ \rho_f, \rho_g \}$
    \end{enumerate}
\end{proposition}

\begin{proof}
    To prove (i), note that
    \begin{align*}
        M_{f+g}(r) &= \max_{\vert z \vert = r} \vert f(z) + g(z) \vert \leq \max_{\vert z \vert = r} \vert f(z) \vert + \vert g(z) \vert \leq \max_{\vert z \vert = r} \vert f(z) \vert + \max_{\vert z \vert = r} \vert g(z) \vert \leq \\
        &= M_f(r) + M_g(r) \leq 2 \max \{ M_f(r), M_g(r) \}
    \end{align*}
    thus
    $$ \log M_{f+g}(r) \leq \log 2 + \log \max \{ M_f(r), M_g(r) \} = \log 2 + \max \{ \log M_f(r), \log M_g(r) \}. $$
    If $M_f(r)$ and $M_g(r)$ are bounded, then applying the above in \cref{eq:def-order} implies that $f, g$ and $f+g$ all have order $0$. If either one is not, then $\max \{ \log M_f(r), \log M_g(r) \}$ necessarily dominates $\log 2$ and we obtain
    \begin{align*}
        \rho_{f+g} &= \limsup_{r \to \infty} \frac{\log \log M_{f+g}(r)}{\log r} \leq \limsup_{r \to \infty} \frac{\log (\log 2 + \max \{ \log M_f(r), \log M_g(r) \})}{\log r} = \\
        &= \limsup_{r \to \infty} \frac{\log \max \{ \log M_f(r), \log M_g(r) \}}{\log r} = \\
        &= \limsup_{r \to \infty} \max \left\{ \frac{\log \log M_f(r)}{\log r}, \frac{\log \log M_g(r)}{\log r} \right\} = \\
        &= \max \left\{ \limsup_{r \to \infty} \frac{\log \log M_f(r)}{\log r}, \limsup_{r \to \infty} \frac{\log \log M_g(r)}{\log r} \right\} = \max \{ \rho_f, \rho_g \}.
    \end{align*}
    To prove (ii), we similarly note that
    \begin{align*}
        \log \log M_{fg}(r) &\leq \log \log ( M_f(r) M_g(r) ) = \log (\log M_f(r) + \log M_g(r)) \leq \\
        &\leq \log (2 \max \{ \log M_f(r), \log M_g(r) \}) = \\
        &= \log 2 + \max \{ \log \log M_f(r), \log \log M_g(r) \},
    \end{align*}
    from where we can proceed as in (i).
\end{proof}

\todo{I could remark that if $\rho_f \neq \rho_g$ we can actually achieve equality in the above, but the proof for the multiplication case is difficult.}

% \begin{remark}
% maybe a remark on equality in the above proposition
% This requires an estimate of the entire function of smaller order from below, which is not trivial. Look in B. Levin, Distribution of zeros of entire functions. American Mathematical Society, Providence, R.I., 1980, Chap. I, § 9, Theorem 12 a).
% \end{remark}

For entire functions of finite order, we can obtain a representation of the order via the coefficients in their power series expansion.

\begin{theorem} \label{thm:order-power-series}
    Let $f(z) = \powerseries{a_n}$ be an entire function. Then $f$ is of finite order $\rho$ if and only if
    $$ \mu \coloneqq \limsup_{n \to \infty} \frac{n \log n}{\log \frac{1}{\vert a_n \vert}} $$
    is finite, where we take the quotient to be zero if $a_n = 0$. In either case we have $\rho = \mu$.
\end{theorem}

\begin{proof}
    We shall first show that $\mu \leq \rho$. If $\mu = 0$ there is nothing to show. Suppose we have $0 < \varepsilon < \mu$. Then, for infinitely many $n$, we have
    \begin{equation*}
        n \log n \geq \kappa \log \frac{1}{\vert a_n \vert}, \quad \textrm{where } \kappa \coloneqq \left\{ \begin{array}{ll}
            \mu - \varepsilon, & \mu < \infty, \\
            \varepsilon, & \textrm{otherwise}.
        \end{array} \right.
    \end{equation*}
    By Cauchy's formula we have, for all $r > 0$,
    \begin{equation*}
        \vert a_n \vert = \left\vert \frac{f^{(n)}(0)}{n!} \right\vert = \left\vert \frac{1}{2 \pi i} \int_{\vert z \vert = r} \frac{f(\zeta)}{\zeta^{n+1}} \diff \zeta \right\vert \leq \left\vert \frac{2 \pi r}{2 \pi i} \frac{M_f(r)}{r^{n+1}} \right\vert = \frac{M_f(r)}{r^n}.
    \end{equation*}
    Thus we obtain, by combining the above,
    \begin{alignat*}{3}
        &                \quad & n \log n    &\geq \kappa \log \frac{r^n}{M_f(r)} \\
        &\Leftrightarrow \quad & \log M_f(r) &\geq n \log r - \frac{n}{\kappa} \log n. \tag{\textasteriskcentered} \label{eq:order-power-series-rhs}
    \end{alignat*}
    Differentiating the right hand side with respect to $n$ yields
    \begin{alignat*}{3}
        &                \quad & 0      &\overset{!}{=} \log r - \frac{1}{\kappa} \log n - \frac{1}{\kappa} \\
        &\Leftrightarrow \quad & \log n &= \kappa \log r - 1 \\
        &\Leftrightarrow \quad & n      &= \frac{r^\kappa}{e},
    \end{alignat*}
    thus the right hand side of \eqref{eq:order-power-series-rhs} becomes maximal for such $n$. Since the above holds for infinitely many $n$, we obtain an unbounded sequence of values for $r$ satisfying
    \begin{alignat*}{3}
        &                 \quad & \log M_f(r) &\geq \frac{r^\kappa}{e} \log r - \frac{r^\kappa}{e \kappa} \log \frac{r^\kappa}{e} = \frac{r^\kappa}{e \kappa} \\
        & \Leftrightarrow \quad & \log \log M_f(r) &\geq \kappa \log r - (1 + \log \kappa) \\
        & \Leftrightarrow \quad & \frac{\log \log M_f(r)}{\log r} &\geq \kappa - \frac{1 + \log \kappa}{\log r}.
    \end{alignat*}
    Taking the limit superior as $r \to \infty$ yields $\rho \geq \kappa$. Finally, if $\mu < \infty$, then letting $\varepsilon \to 0$, $\varepsilon \to \infty$ otherwise, concludes $\rho \geq \mu$.

    \todo{TODO: $\rho \leq \mu$.}
\end{proof}

\begin{example} \label{exm:examples-orders}
    We shall apply \Cref{thm:order-power-series} to obtain the order of certain entire functions via their power series expansion $ \powerseries{a_n} $:

    \begin{enumerate}[i.]
        \item It is immediately apparent that polynomials have order zero. Non-polynomial functions of zero order do exist, given that their coefficients decrease sufficiently rapidly, as in the example of $a_n \coloneqq n^{-(n^2)} $.
        \item By Stirling's approximation we have $ \log n! = n \log n + O(n) $, from which we conclude that the exponential function, sine and cosine all have order 1.
        \item For any given $\rho > 0$ the coefficients $ a_n \coloneqq n^{- \rho n} $ define an entire function of order $\rho$.
        \item For $n \geq 2$, the coefficients $ a_n \coloneqq n^{- \frac{n}{\sqrt{\log n}}} $ define an entire function of infinite order. Note that by \Cref{rem:order}, we can also conclude that $e^{e^z}$ is of infinite order.
    \end{enumerate}
\end{example}



\begin{proposition} \label{prop:order-derivative}
    Let $f$ be an entire function of finite order with derivative $f'$. Then $\rho_{f'} = \rho_f$.
\end{proposition}

\begin{proof}
    Given $f(z) = \powerseries{a_n}$ we have $f'(z) = \powerseries{(n+1)a_{n+1}}$.
    
    Since $ n \log n \sim (n+1) \log(n+1) $ we have
    \begin{align*}
        \limsup_{n \to \infty} \frac{n \log n}{\log \frac{1}{\vert (n+1) a_{n+1} \vert}} &= \liminf_{n \to \infty} \left( \frac{- \log (n+1) + \log \frac{1}{\vert a_{n+1} \vert}}{n \log n} \right)^{-1} = \\
        &= \liminf_{n \to \infty} \left( \frac{\log \frac{1}{\vert a_{n+1} \vert}}{n \log n} \right)^{-1} \cdot \lim_{m \to \infty} \left( \frac{m \log m}{(m+1) \log(m+1)} \right)^{-1} = \\
        &= \liminf_{n \to \infty} \left( \frac{\log \frac{1}{\vert a_{n+1} \vert}}{n \log n} \cdot \frac{n \log n}{(n+1) \log(n+1)} \right)^{-1} = \\
        &= \liminf_{n \to \infty} \left( \frac{\log \frac{1}{\vert a_{n+1} \vert}}{(n+1) \log(n+1)} \right)^{-1} = \limsup_{n \to \infty} \frac{(n+1) \log (n+1)}{\log \frac{1}{\vert a_{n+1} \vert}} = \\
        &= \limsup_{n \to \infty} \frac{n \log n}{\log \frac{1}{\vert a_{n} \vert}}
    \end{align*}
    and, since $\rho_f < \infty$, \Cref{thm:order-power-series} concludes $\rho_{f'} = \rho_f$.
\end{proof}