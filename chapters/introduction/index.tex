\chapter{Introduction}
\label{ch:introduction}

In this thesis, we explore the theory of entire functions, that is, complex-valued functions that are holomorphic in the entire complex plane.

Our early focus lies on Picard's Great Theorem, an important corollary of which states that an entire function assumes every complex value infinitely often, with at most one exception. A natural example for this behaviour is the complex exponential function, which does not assume zero, but does assume all other values infinitely often. \Cref{ch:picards-great-theorem} is dedicated to deriving this theorem and providing groundwork for the results in the later chapters.

\Cref{sec:order,sec:type} then briefly introduce a scale of asymptotic growth for the modulus of entire functions, in the form of order and type. Roughly speaking, if $f$ is of finite order, then $f$ grows no faster than a function of the form $e^{Q(z)}$, where $Q$ is a polynomial, as $\vert z \vert \to \infty$. Results introduced here are not particularly interesting, but will frequently be utilized towards the end of the chapter.

\Cref{sec:zeros,sec:canonical-products,sec:hadamards-theorem} explore the nature of zeros of entire functions. As is well-known, a complex polynomial can be represented solely by its zeros and some scaling factor. In \Cref{sec:canonical-products}, this property is generalized for entire functions by the Weierstraß Factorization Theorem: For any entire function $f$ there exists an entire function $g$ and a so-called ``canonical product'' $\Pi$, which is determined by the zeros of $f$, such that
$$ f(z) = z^m e^{g(z)} \Pi(z), \quad \textrm{for all } z \in \C, $$
where $m \in \N_0$ denotes the order of the zero of $f$ at the origin. Hadamard's Theorem, which is the focus of \Cref{sec:hadamards-theorem}, then provides an immediate refinement for entire functions of finite order $\rho $, showing that $g$ can be taken to be a polynomial of degree at most $\rho$. This has immediate corollaries, in particular if the order is not an integer, which we will explore towards the end of the section.

\todo{I should clarify what ``degree at most'' means if $\rho$ is not an integer. Maybe note which values $\rho$ can take overall.}

Finally, \Cref{ch:composition-entire-functions} studies when the composition of functions of finite order is, again, of finite order. Pólya's Theorem will provide necessary conditions for this to hold. A consequence is Thron's Theorem, which states that for any entire function $g$ of finite order, which assumes some complex value only finitely often, there does not exist an entire function $f$ for which $f \circ f = g$ holds on the entire complex plane.

Introductory results on complex analysis are assumed as prerequisites and will be used as found in \cite{bluemlinger-complex-analysis,stein-shakarchi-princeton}. These include Cauchy's integral formula, the maximum modulus principle, the theorems of Montel, Hurwitz and Jensen, the Schwarz lemma and standard results on power series.

The following is an overview of the notation used:

\begin{itemize}[label=$\rightsquigarrow$]
    \item The sets $\N$ and $\Z$ denote, respectively, the natural numbers (starting with $1$) and the integers. Furthermore we define $\N_0 \coloneqq \N \cup \{ 0 \}$.
    \item For a set $A$, its indicator function $\1_A(a)$ is defined as $1$ if $a \in A$ and $0$ otherwise.
    \item The restriction of a function $f : M \to N$ to a set $D \subseteq M$ will be denoted $\restr{f}{D}$.
    \item The sets $\R$ and $\C$ denote, respectively, the real line and the complex plane, which will both be endowed with the standard Euclidean metric and topology.
    \item The closure of a subset $A$ of $\R$ or $\C$ (with respect to the appropriate topology) will be denoted as $\cl{A}$ and the boundary as $\partial A$.
    \item For subsets $M, N$ of $\R$ or $\C$ we denote by $C(M, N)$ the space of continuous maps from $M$ into $N$.
    \item For a real- or complex-valued function $f$ defined on some set $M$ we define $\Vert f \Vert_M \coloneqq \sup_{x \in M} \vert f(x) \vert$.
    \item If a function $f$ is differentiable, in the real or complex sense, then $f'$ and $f''$ denote the respective first and second derivatives of $f$. For $n \in \N$, the $n$-th derivative will be denoted $f^{(n)}$.
    \item Let $a \in \R$ and let $f, g$ be functions on $[a, \infty)$ such that $f$ is either real- or complex-valued and $g$ is real-valued. Then we write $f(x) = \O(g(x))$ as $x \to \infty$ if there exist constants $C > 0$ and $x_0 \geq a$ such that for $x \geq x_0$ we have $\vert f(x) \vert \leq C g(x)$.
    \item For $a, b \in \C$ we denote by $[a, b]$ the line segment connecting $a$ and $b$, parametrized by $\gamma : [0, 1] \to \C, t \mapsto (1 - t) a + tb$.
    \item If $G \subseteq \C$, then $G$ is said to be a \emph{domain} if $G$ is non-empty, open and connected.
    \item For a point $a \in \C$ and $r > 0$ we denote by $B_r(a)$ the open disk of radius $r$ centered at $a$. The open unit disk $B_1(0)$ will be denoted $\D$. Furthermore we define $\C^\times \coloneqq \C \setminus \{ 0 \}$ and $\D^\times \coloneqq \D \setminus \{ 0 \}$.
    \item Let $G \subseteq \C$ be open, then a function $f : G \to \C$ is said to be \emph{holomorphic} at $z \in G$ if $f$ is complex differentiable at $z$. If $f$ is holomorphic at all $z \in G$, then $f$ is said to be holomorphic on $G$. The space of all such holomorphic functions on $G$ will be denoted as $H(G)$. If $f \in H(\C)$, then $f$ is called \emph{entire}.
    \item If $G \subseteq \C$ is a domain and $D \subseteq \C$ is a set with $\cl{D} \subseteq G$, then the contour integral of a function $f \in H(G)$ along the boundary of $D$ will be denoted
    $$ \oint_{\partial D} f(z) \diff z $$
    and is, unless otherwise indicated, always parametrized with mathematically positive (counter-clockwise) orientation. \todo{Only for NICE boundaries.}
    \item The exponential function will be denoted $\exp : \C \to \C^\times$. The (local) inverse will be denoted $\log$, where we, unless specified otherwise, always choose the principle branch, i.e., the branch where $\log 1 = 0$.
\end{itemize}
