\section{Pólya's Theorem}
\label{sec:polyas-theorem}

Necessary conditions for the order of a composition to be finite will be given by Pólya's Theorem, the proof of which relies on a result that was first proven by Harald Bohr \cite{segal-complex-analysis}. While Bloch's Theorem dealt with disks contained in the image of holomorphic functions, we now focus on circles contained in such images.

\begin{proposition} \label{prop:existence-circle-maximal-radius}
    Let $G \subset \C$ be a bounded domain with $0 \in G$. Then the set
    \begin{equation} \label{eq:set-circle-maximal-radius}
        S \coloneqq \{ r \geq 0 : \partial B_r(0) \subseteq \cl{G} \}
    \end{equation}
    has a positive maximum; that is, $\cl{G}$ contains a circle of positive, maximal radius\footnote{The circle of radius $0$ is understood as the singleton $\{ 0 \}$.}.
\end{proposition}

\begin{proof}
    Since $G$ is open and contains $0$ there is some $t > 0$ such that $B_t(0) \subseteq G$. Since $\partial B_t(0) \subset \cl{B_t(0)} \subseteq \cl{G}$ we thus have $t \in S$ and $S$ is non-empty. It now suffices to show that $S$ is compact, since then it would contain its maximum $m \geq t$. Clearly $S$ is bounded from below by $0$ and since $\cl{G}$ is bounded it must also be bounded from above. Thus it remains to show that $S$ is closed.

    Let $(r_n)_{n \in \N}$ be a convergent sequence in $S$ with limit $r \in [0, \infty)$. If $r = 0$, then $r \in S$. Otherwise we claim that $\partial B_r(0) \subseteq \cl{G}$. Let $z \in \partial B_r(0)$, and set $z_n \coloneqq r_n z / r$ for $n \in \N$. Now $\vert z_n \vert = r_n$, thus $z_n \in \partial B_{r_n}(0) \subseteq \cl{G}$ and therefore $(z_n)_{n \in \N}$ is a sequence in $\overline{G}$. Clearly $z_n \to z$ and since $\cl{G}$ is closed we therefore have $z \in \overline{G}$. Since $z$ was arbitrary, this concludes the claim, we have $r \in S$ and thus $S$ is closed.
\end{proof}

If $G \subset \C$ is a bounded domain and $f \in H(G) \cap C(\cl{G}; \C)$ is non-constant with $f(0) = 0$, then $f(G)$ is a bounded domain with $0 \in f(G)$. This, together with \Cref{prop:existence-circle-maximal-radius}, justifies the following definition:

\begin{definition}
    Let $f \in H(\D) \cap C(\cl{\D}; \C)$ be non-constant with $f(0) = 0$, then we define $r_f$ as the positive maximum of the set $S$ in \eqref{eq:set-circle-maximal-radius}, where $G = f(\D)$.
\end{definition}

Bohr's Theorem now asserts that $r_f$ does (almost) not depend on $f$ itself:

\begin{theorem}[Bohr] \label{thm:bohr}
    There exists a function ${\phi : (0, 1) \to (0, \infty)}$ such that for any $\theta \in (0, 1)$ and $f \in H(\D) \cap C(\cl{\D}; \C)$ with $f(0) = 0$ and $M_f(\theta) = 1$ it holds that $r_f \geq \phi(\theta)$.
\end{theorem}

\begin{proof}
    Suppose $f$ satisfies the hypothesis of the theorem, let $\varepsilon > 0$ and set $R_\varepsilon \coloneqq r_f + \varepsilon$. By definition of $r_f$, for all $r \geq R_\varepsilon$ there exists some point $w_r \in \partial B_r(0)$ with $w_r \notin f(\cl{\D})$. Choose such points $w_{R_\varepsilon}, w_{2 R_\varepsilon}$ and define
    $$ h(z) \coloneqq \frac{f(z) - w_{R_\varepsilon}}{w_{2 R_\varepsilon} - w_{R_\varepsilon}} \in H(\D; \C \setminus \{ 0, 1 \}). $$
    Since
    $$ \vert h(0) \vert = \left\vert \frac{f(0) - w_{R_\varepsilon}}{w_{2 R_\varepsilon} - w_{R_\varepsilon}} \right\vert \leq \frac{\vert w_{R_\varepsilon} \vert}{\vert \vert w_{2 R_\varepsilon} \vert - \vert w_{R_\varepsilon} \vert \vert} = \frac{R_\varepsilon}{2 R_\varepsilon - R_\varepsilon} = 1, $$
    by Schottky's Theorem we have $\vert h(z) \vert \leq \psi(\theta, 1)$ for all $\vert z \vert \leq \theta$. Therefore
    $$ \vert f(z) \vert - R_\varepsilon \leq \vert g(z) - w_{R_\varepsilon} \vert \leq \vert w_{2 R_\varepsilon} - w_{R_\varepsilon} \vert \psi(\theta, 1) \leq 3 R_\varepsilon \psi(\theta, 1) $$
    and thus $\vert f(z) \vert \leq R_\varepsilon + 3 R_\varepsilon \psi(\theta, 1)$ for all $\vert z \vert \leq \theta$. Using the hypothesis that $M_f(\theta) = 1$ and the maximum modulus principle we obtain $1 \leq R_\varepsilon + 3 R_\varepsilon \psi(\theta, 1)$, and letting $\varepsilon \to 0$ we have $1 \leq r_f + 3 r_f \psi(\theta, 1)$. Thus
    $$ r_f \geq \frac{1}{1 + 3 \psi(\theta, 1)}$$
    and defining $\phi(\theta)$ as the right-hand side establishes the assertion.
\end{proof}

\begin{theorem}[Pólya] \label{thm:polya}
    Let $g, h \in H(\C)$ be non-constant. For the order of $g \circ h$ to be finite, it must hold that either
    \begin{enumerate}[i.]
        \item $h$ is a polynomial and $\rho_g < \infty$, or
        \item $h$ is not a polynomial, $\rho_h < \infty$ and $\rho_g = 0$.
    \end{enumerate}
\end{theorem}

\begin{proof}
    Without loss of generality we can assume $h(0) = 0$; otherwise we just consider $h_0(z) \coloneqq h(z) - h(0)$ and $g_0(w) \coloneqq g(w + h(0))$. Set $f \coloneqq g \circ h$ and define
    \begin{equation*}
        k_r(z) \coloneqq \frac{h(rz)}{M_h(r / 2)} \in H(\D) \cap C(\cl{\D}; \C), \quad \textrm{for } r > 0.
    \end{equation*}
    Note that by definition we have $M_{k_r}(1/2) = 1$ and $k_r(0) = 0$, thus by Bohr's Theorem there is some constant $C > 0$ and an $R > C M_h(r/2)$ such that $\partial B_{R / M_h(r/2)}(0) \subseteq k_r(\cl{\D})$ and thus $\partial B_R(0) \subseteq h(\cl{B_r(0)})$. By the maximum modulus principle, $g$ assumes its maximum modulus over $\cl{B_R(0)}$ at some $w_0 \in \partial B_R(0)$. By the above there is a $z_0 \in \cl{B_r(0)}$ with $h(z_0) = w_0$. Thus we get
    \begin{equation*}
        M_g(C M_h(r / 2)) < M_g(R) = \vert g(w_0) \vert = \vert g(h(z_0)) \vert = \vert f(z_0) \vert \leq M_f(r).
    \end{equation*}
    Assuming $\rho_f < \infty$, we have $M_f(r) < K \exp(r^\alpha)$ for every $\alpha > \rho_f$, with suitable $K > 0$ and all $r > r_0$. Consider the power series expansion $h(z) = \sum_{n=0}^\infty a_n z^n$. Let $a_m$ denote any non-zero coefficient; note that since $h(0) = 0$ we have $m \geq 1$. By Cauchy's integral formula we have, for all $s > 0$,
    \begin{equation*} \label{eq:proof-polya-cauchy-inequality}
        \vert a_m \vert = \left\vert \frac{h^{(m)}(0)}{m!} \right\vert = \frac{1}{2\pi} \left\vert \oint_{\partial B_{s}(0)} \frac{h(\zeta)}{\zeta^{n+1}} \diff \zeta \right\vert \leq \frac{M_h(s)}{s^m} \tag{\textasteriskcentered}
    \end{equation*}
    and thus
    \begin{equation*}
        M_g(C \vert a_m \vert (r/2)^m) \leq M_g(C M_h(r/2)) < M_f(r) < K \exp(r^\alpha), \quad \textrm{for all } r > r_0.
    \end{equation*}
    Replacing $(r/2)^m$ with $r$ we obtain $\rho_g \leq \alpha / m $. If $h$ is not a polynomial we may let $m \to \infty$, thus $\rho_g = 0$.

    Now consider $g(z) = \sum_{n=0}^\infty b_n z^n$. Replacing $h$ with $g$ in (\ref{eq:proof-polya-cauchy-inequality}) we obtain $\vert b_n \vert s^n \leq M_g(s)$ for all $s > 0$ and $n \geq 1$ and thus
    \begin{equation*}
        \vert b_n \vert (C M_h(r/2))^n \leq M_g(C M_h(r/2)) < M_f(r) < K \exp(r^\alpha), \quad \textrm{for all } r > r_0,
    \end{equation*}
    which implies $\rho_h \leq \alpha < \infty$.
\end{proof}

\begin{theorem}[Thron] \label{thm:thron}
    Let $g \in H(\C)$ be transcendental, $\rho_g < \infty$ and suppose that $g$ assumes some value $w \in \C$ only finitely often. Then there exists no $f \in H(\C)$ with $f \circ f = g$.
\end{theorem}

\begin{proof}
    Assume towards a contradiction that $f \in H(\C)$ with $f \circ f = g$ exists. Since $g$ is not a polynomial, $f$ is not a polynomial either. Thus Pólya's Theorem implies $\rho_f = 0$.
    
    Consider the sets
    $$ Z \coloneqq f^{-1}(\{ w \}), \quad Z' \coloneqq \bigcup_{z \in Z} f^{-1}(\{ z \}). $$
    By definition, for each $z' \in Z'$ there is some $z \in Z$ with $z' \in f^{-1}(\{ z \})$, thus
    $$ g(z') = f(f(z')) = f(z) = w. $$
    Our hypothesis on $g$ implies that $Z'$ must be finite. Since pre-images of singletons are disjoint, $Z'$ is a disjoint union, therefore
    $$ \sum_{z \in Z} \vert f^{-1}(\{ z \}) \vert = \left\vert \bigcup_{z \in Z} f^{-1}(\{ z \}) \right\vert = \vert Z' \vert < \infty $$
    and thus all points in $Z$ are only assumed finitely often by $f$. But by \Cref{cor:transcendental-every-value-inf}, $f$ assumes at most one value only finitely often; therefore $\vert Z \vert \leq 1$.

    If $Z = \emptyset$, then $h(z) \coloneqq f(z) - w$ is entire, of order $0$ and nowhere $0$. As discussed in \Cref{rem:consequences-hadamard}, as a consequence of Hadamard's Theorem we have that $h$ is constant and consequently so is $f$, a contradiction.

    If $Z = \{ z_0 \}$, then $h(z) \coloneqq f(z) - w$ has a single zero of finite order $n \in \N$ at $z_0$. Therefore we can write $h(z) = (z - z_0)^n p(z)$, where $p$ is entire, of order $0$ and nowhere $0$. Again, this implies that $p$ is constant, and therefore $f$ a polynomial, a contradiction.
\end{proof}

\begin{example}
    A natural application of Thron's Theorem is to set $g(z) \coloneqq \exp z$, which never assumes zero as a value. Indeed, this implies that there is no $f \in H(\C)$ satisfying
    $$ f(f(z)) = \exp z. $$
    On the other hand, there does exist a real-analytic function satisfying the above. The construction of such a function is difficult, but was demonstrated by H. Kneser \cite{kneser-real-analytic-solution}.
\end{example}