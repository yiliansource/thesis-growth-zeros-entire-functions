\section{Pólya's Theorem}
\label{sec:polyas-theorem}

Necessary conditions for the order of a composition to be finite will be given by Pòlya's Theorem, the proof of which relies on the following result:

\begin{theorem}[Bohr] \label{thm:bohr}
    Let $0 < \theta < 1$ and $f \in H(\cl{\D})$, such that $f(0) = 0$ and $M_f(\theta) = 1$. Let $r_f$ denote the largest $r > 0$ such that $\partial B_r(0) \subseteq f(\cl{\D})$. Then we have $r_f \geq C_\theta$, where $C_\theta > 0$ is a constant depending only on $\theta$.
\end{theorem}

\begin{proof}
    Suppose $g$ satisfies the hypothesis of the theorem and let $C > 0$ such that for all $r \geq C$ there exists some point $w_r \in \partial B_r(0)$ with $w_r \notin f(\cl{\D})$. Choose such points $w_C, w_{2C}$ and define
    $$ h(z) \coloneqq \frac{g(z) - w_C}{w_{2C} - w_C} \in H(\cl{\D}, \C \setminus \{ 0, 1 \}). $$
    Since
    $$ \vert h(0) \vert = \left\vert \frac{g(0) - w_C}{w_{2C} - w_C} \right\vert \leq \frac{C}{2C - C} = 1, $$
    by Schottky's Theorem we have $\vert h(z) \vert \leq \psi(\theta, 1)$ for all $\vert z \vert \leq \theta$. Therefore
    $$ \vert g(z) \vert - C \leq \vert g(z) - w_C \vert \leq \vert w_{2C} - w_C \vert \psi(\theta, 1) \leq 3 C \psi(\theta, 1) $$
    and thus $\vert g(z) \vert \leq C + 3C \psi(\theta, 1)$ for all $\vert z \vert \leq \theta$. Using the hypothesis that $M_g(\theta) = 1$ and the maximum modulus principle we obtain $C \geq \frac{1}{1 + 3 \psi(\theta, 1)}$ and choosing $C_\theta$ to be less than this constant yields the theorem.
\end{proof}

\begin{theorem}[Pólya] \label{thm:polya}
    Let $g, h \in H(\C)$. For the order of $g \circ h$ to be finite, it must hold that either
    \begin{enumerate}[i.]
        \item $h$ is a polynomial and $\rho_g < \infty$, or
        \item $h$ is not a polynomial, $\rho_h < \infty$ and $\rho_g = 0$.
    \end{enumerate}
\end{theorem}

\begin{proof}
    Without loss of generality we can assume $h(0) = 0$; otherwise we just consider $h_0(z) \coloneqq h(z) - h(0)$ and $g_0(w) \coloneqq g(w + h(0))$. Set $f \coloneqq g \circ h$ and define
    \begin{equation*}
        k_r(z) \coloneqq \frac{h(rz)}{M_h(r / 2)} \in H(\cl{\D}), \quad \textrm{for } r > 0.
    \end{equation*}
    Note that by definition we have $M_{k_r}(1/2) = 1$ and $k_r(0) = 0$, thus by \Cref{thm:bohr} there is some constant $C > 0$ and an $R / M_h(r/2) > C$ such that $\partial B_{R / M_h(r/2)}(0) \subseteq k_r(\cl{\D})$ and thus $\partial B_R(0) \subseteq h(\cl{B_r(0)})$. By the maximum modulus principle, $\vert g \vert$ assumes its maximum over $\cl{B_R(0)}$ at some $w_0 \in \partial B_R(0)$. By the above there is a $z_0 \in \cl{B_r(0)}$ with $h(z_0) = w_0$. Thus we get
    \begin{equation*}
        M_g(C M_h(r / 2)) < M_g(R) = \vert g(w_0) \vert = \vert g(h(z_0)) \vert = \vert f(z_0) \vert \leq M_f(r).
    \end{equation*}
    Assuming $\rho_f < \infty$, we have $M_f(r) < K \exp(r^\alpha)$ for some $\alpha > \rho_f$. Consider the power series expansion $h(z) = \sum_{n=0}^\infty a_n z^n$ and let $a_m$ denote any non-zero coefficient; note that since $h(0) = 0$ we have $m \geq 1$. By Cauchy's integral formula we have, for all $s > 0$,
    \begin{equation*} \label{eq:proof-polya-cauchy-inequality}
        \vert a_m \vert = \left\vert \frac{h^{(m)}(0)}{m!} \right\vert = \frac{1}{2\pi} \left\vert \int_{\partial B_{s}(0)} \frac{h(\zeta)}{\zeta^{n+1}} \diff \zeta \right\vert \leq \frac{M_h(s)}{s^m} \tag{\textasteriskcentered}
    \end{equation*}
    and thus
    \begin{equation*}
        M_g(C \vert a_m \vert (r/2)^m) \leq M_g(C M_h(r/2)) < M_f(r) < K \exp(r^\alpha), \quad \textrm{for all } r > 0.
    \end{equation*}
    Replacing $(r/2)^m$ with $r$ we obtain $\rho_g \leq \alpha / m $. If $h$ is not a polynomial we may let $m \to \infty$, thus $\rho_g = 0$.

    Now consider $g(z) = \sum_{n=0}^\infty b_n z^n$. Replacing $h$ with $g$ in (\ref{eq:proof-polya-cauchy-inequality}) we obtain $\vert b_n \vert s^n \leq M_g(s)$ for all $s > 0$ and $n \geq 1$ and thus
    \begin{equation*}
        \vert b_n \vert (C M_h(r/2))^n \leq M_g(C M_h(r/2)) < M_f(r) < K \exp(r^\alpha),
    \end{equation*}
    which implies $\rho_h \leq \alpha < \infty$.
\end{proof}

\begin{theorem}[Thron] \label{thm:thron}
    Let $g \in H(\C)$ be transcendental, of finite order, such that $g$ assumes some value $w \in \C$ only finitely often. Then there does not exist any $f \in H(\C)$ such that $f \circ f = g$.
\end{theorem}

\begin{proof}
    Seeking contradiction, suppose there were such a $f \in H(\C)$. Since $g$ is not a polynomial, Pólya's Theorem implies that $f$ is of order $0$ and not a polynomial. Let $(z_j)_{j \in J}$ denote the points where $f$ equals $w$. For each $m \in J$ we additionally denote by $(z_{j,m})_{j \in J_m}$ the points where $f$ equals $z_m$. Thus, for each $m \in J$ and $n \in J_m$ we have
    $$ g(z_{n,m}) = f(f(z_{n,m})) = f(z_m) = w. $$
    Our assumption on $g$ assures that there must only be finitely many distinct points among the $(z_{n,m})_{m \in J,n \in J_m}$. Thus, each point in $(z_j)_{j \in J}$ is only taken on by $f$ finitely often.

    By \Cref{cor:transcendental-every-value-inf}, $f$ assumes all values in the complex plane infinitely often, with at most one exception. This implies that that there is at most one $z_0$ in $(z_j)_{j \in J}$ that is only taken on finitely often by $f$.

    If there is no such $z_0$, then $h(z) \coloneqq f(z) - w$ is entire, of order $0$ and nowhere $0$. Thus, by Hadamard's Theorem, $h$ must be constant, and therefore $f$ aswell, a contradiction.

    If such a $z_0$ exists, then $h(z) \coloneqq f(z) - w$ has a zero of finite order $n \in \N$ at $z_0$. Therefore we can write $h(z) = (z - z_0)^n p(z)$, where $p$ is entire, of order $0$ and nowhere $0$. Again, this implies that $p$ is constant, and therefore $f$ a polynomial, a contradiction.
\end{proof}

\begin{example}
    A natural application of \Cref{thm:thron} is taking $g$ to be $e^z$, which never assumes zero as a value. Indeed, this implies that there is no entire function $f$ satisfying
    $$ f(f(z)) = e^z. $$
    On the other hand, there does exist a real-analytic function satisfying the above, as demonstrated by H. Kneser. \todo{I probably still need a citation here.}
\end{example}