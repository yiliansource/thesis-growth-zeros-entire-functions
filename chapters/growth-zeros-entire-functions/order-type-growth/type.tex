\subsection{Type}

For functions of finite and positive order, we can obtain a natural refinement of the concept of order:

\begin{definition}
    Let $f \in H(\C)$ be of order $0 < \rho_f < \infty$. The \emph{type} of $f$ is defined by
    \begin{equation} \label{eq:def-type}
        \tau_f \coloneqq \limsup_{r \to \infty} \frac{\log M_f(r)}{r^{\rho_f}}
    \end{equation}

    If $\tau_f = 0$, then $f$ is said to be of \emph{minimal} type, if $0 < \tau_f < \infty$ of \emph{normal} type, and if $\tau_f = \infty$ of \emph{maximal} type.
\end{definition}

Once again, we have an equivalent characterization for functions of finite order and type:

\begin{proposition} \label{prop:type-infimum}
    Let $f \in H(\C)$ be of finite, positive order, then $f$ is of finite type if and only if
    \begin{equation}
        \tau \coloneqq \inf \{ t > 0 : M_f(r) = O(\exp (t r^{\rho_f})) \textrm{ as } r \to \infty \}
    \end{equation}
    is finite, and in either case we have $\tau_f = \tau$.
\end{proposition}

\begin{proof}
    Suppose $0 \leq \tau < \infty$, then for all $t > \tau$ we have $M_f(r) = O(\exp t r^{\rho_f})$ as $r \to \infty$. Thus there exists a constant $K > 0$ such that for all sufficiently large $r > 0$ we have $M_f(r) \leq K \exp t r^{\rho_f}$. Therefore
    \begin{align*}
        \frac{\log M_f(r)}{r^{\rho_f}} \leq \frac{\log K + t r^{\rho_f}}{r^{\rho_f}} = \frac{\log K}{r^{\rho_f}} + t
    \end{align*}
    and taking limits superior as $r \to \infty$ and letting $t \to \tau$ afterwards yields $\tau_f \leq \tau$.

    Now suppose $0 \leq \tau_f < \infty$ and let $t > \tau_f$. Then, by definition of the limit superior, for all sufficiently large $r > 0$ we have $ \log M_f(r) \leq t r^{\rho_f}$ and therefore $M_f(r) \leq \exp(t r^{\rho_f})$. Thus $M_f(r) = O(\exp (t r^{\rho_f}))$, thereby $\tau \leq t$ and letting $t \to \tau_f$ yields $\tau \leq \tau_f$.
\end{proof}

% Equivalently, $\tau_f$ can also be defined as the infimum over all $\tau > 0$ such that $M_f(r) = O(\exp(\tau r^{\rho_f}))$ as $r \to \infty$.

If $f \in H(\C)$ is of finite, positive order, then $f$ and $f'$ not only have the same order, they have the same type:

\begin{proposition} \label{prop:type-derivative}
    If $f \in H(\C)$ is of finite, positive order and finite type, then $\tau_{f'} = \tau_f$.
\end{proposition}

\begin{proof}
    Reusing the inequalities obtained in \Cref{prop:order-derivative} we have, for all $r > 0$,
    \begin{equation*}
        M_f(r) \leq r M_{f'}(r), \quad \textrm{and} \quad M_{f'}(r) \leq M_f(r+1).
    \end{equation*}
    Therefore
    \begin{align*}
        \frac{\log M_f(r)}{r^{\rho_f}} &\leq \frac{\log r + \log M_{f'}(r)}{r^{\rho_f}}, \quad \textrm{and} \quad \frac{\log M_{f'}(r)}{r^{\rho_f}} \leq \frac{\log M_{f}(r+1)}{(r+1)^{\rho_f}} \frac{(r+1)^{\rho_f}}{r^{\rho_f}}
    \end{align*}
    and taking limits superior as $r \to \infty$ yields $\tau_f \leq \tau_{f'}$ and $\tau_{f'} \leq \tau_f$.
\end{proof}