\chapter{Growth and Zeros of Entire Functions}
\label{ch:growth-zeros-entire-functions}

Entire functions admit an interesting relationship between the asymptotic growth of their modulus and the number of their zeros. Consider a complex polynomial
$$ p(z) = a_0 + a_1 z + \hdots + a_n z^n. $$
The asymptotic growth of $p$ is determined by its degree $n$, which also corresponds to the number of its zeros. Thus, the higher the number of zeros, the faster the asymptotic growth of $p$. Another interesting property is that $p$ is uniquely determined by its zeros and some scaling factor, as shown by the Fundamental Theorem of Algebra. Thus, for complex polynomials, their rate of growth and their zeros are closely related.

A function $f \in H(\C)$ can be represented as an everywhere convergent power series
$$ f(z) = a_0 + a_1 z + \hdots + a_n z^n + \hdots, $$
and thus the entire functions form a natural generalization of the complex polynomials. Motivated by the properties mentioned above, we want to study if similar results hold for entire functions.

\section{Order and Type of Entire Functions} \label{sec:order-type}

The first goal is to introduce a growth scale for the asymptotic modulus of an entire function. To characterize this scale we introduce a ``maximum modulus function''.

\begin{definition}
    Let $f \in H(\C)$, then for $r \geq 0$ we define
    \begin{equation*}
        M_f(r) \coloneqq \max_{\vert z \vert = r} \vert f(z) \vert. \qedhere
    \end{equation*}
\end{definition}

\begin{remark}    
    By the maximum modulus principle, $M_f$ is either strictly increasing (if $f$ is non-constant) or constant (otherwise), and we have
    $$ M_f(r) = \max_{\vert z \vert \leq r} \vert f(z) \vert. $$
    We shall show that $M_f$ is continuous: Let $\varepsilon > 0$ and choose $\delta_\varepsilon > 0$ such that for ${\vert z_1 - z_2 \vert < \delta_\varepsilon}$ we have ${\vert f(z_1) - f(z_2) \vert < \varepsilon}$. Now let $r_1 < r_2$ and choose $\theta$ such that $M_f(r_2) = \vert f(r_2 e^{i \theta}) \vert$. Then
    \begin{equation*}
        0 \leq M_f(r_2) - M_f(r_1) \leq \vert f(r_2 e^{i \theta}) \vert - \vert f(r_1 e^{i \theta}) \vert \leq \vert f(r_2 e^{i \theta}) - f(r_1 e^{i \theta}) \vert < \varepsilon,
    \end{equation*}
    for $r_2 - r_1 < \delta_\varepsilon$.
\end{remark}

\begin{definition} \label{def:order}
    Let $f \in H(\C)$. The \emph{order} of $f$ is defined by
    \begin{equation} \label{eq:def-order}
        \rho_f \coloneqq \limsup_{r \to \infty} \frac{\log \log M_f(r)}{\log r}.
    \end{equation}
    Constant functions, by convention, have order 0.
\end{definition}

Note that for $f \in H(\C)$ we have $0 \leq \rho_f \leq \infty$.

\begin{proposition} \label{prop:order-exponential-polynomial}
    If $Q$ is a polynomial of degree $n \in \N$, then $\exp Q$ has order $n$.
\end{proposition}

\begin{proof}
    Write $Q(z) = \sum_{k=0}^n a_k z^k$ and let $\zeta$ be an $n$-th root of $\overline{a_n} / \vert a_n \vert$. Then
    \begin{align*}
        \vert a_n \vert - \sum_{k=0}^{n-1} \vert a_k \vert r^{k-n} \leq \Re \sum_{k=0}^{n} a_k \zeta^k r^{k-n} = \Re \frac{Q(r \zeta)}{r^n} \leq \max_{\vert z \vert = r} \Re \frac{Q(z)}{r^n} \leq \vert a_n \vert + \sum_{k=0}^{n-1} \vert a_k \vert r^{k-n}
    \end{align*}
    and taking limits as $r \to \infty$ we get $\lim_{r \to \infty} \max_{\vert z \vert = r} \Re \frac{Q(z)}{r^n} = \vert a_n \vert$. Set $f(z) \coloneqq \exp Q(z)$, then
    \begin{align*}
        \frac{\log \log M_f(r)}{\log r} &= \frac{\log \log \max_{\vert z \vert = r} \vert \exp Q(z) \vert}{\log r} = \frac{\log \log \max_{\vert z \vert = r} \exp \Re Q(z)}{\log r} = \\
        &= \frac{\log \max_{\vert z \vert = r} \Re Q(z)}{\log r} = \frac{\log \max_{\vert z \vert = r} \Re Q(z) / r^n}{\log r} + n
    \end{align*}
    and taking the limit superior as $r \to \infty$ yields $\rho_f = n$.
\end{proof}

% \begin{remark}\label{rem:order-exponential-polynomial}
%     If $Q(z) = \sum_{k=0}^n a_k z^k$ is a polynomial we claim that $f(z) \coloneqq \exp {Q(z)}$ is of order $n$. Let $\varepsilon > 0$, then since $\lim_{r \to \infty}(\sum_{k=0}^n \vert a_k \vert r^k) / r^{n + \varepsilon} = 0$ there is some $r_0$ such that for all $r > r_0$ we have $(\sum_{k=0}^n \vert a_k \vert r^k) / r^{n + \varepsilon} < 1$ and thus
%     \begin{align*}
%         M_f(r) \leq \exp \sum_{k=0}^n \vert a_n \vert r^k \leq \exp(r^{n + \varepsilon}),
%     \end{align*}
%     showing $\rho_f \leq n + \varepsilon$. On the other hand, since
% \end{remark}

If the order of an entire function is finite, we have an equivalent characterization.

\begin{proposition} \label{prop:order-infimum}
    Let $f \in H(\C)$, then $f$ is of finite order if and only if
    \begin{equation}
        \rho \coloneqq \inf \{ s > 0 : M_f(r) = \O(\exp r^s) \textrm{ as } r \to \infty \}
    \end{equation}
    is finite, and in either case we have $\rho_f = \rho$.
\end{proposition}

\begin{proof}
    Suppose $0 \leq \rho < \infty$, then for all $s > \rho$ we have $M_f(r) = \O(\exp r^s)$ as $r \to \infty$. Thus there exists a constant $K > 0$ (we may assume $K > 1$) and such that for all sufficiently large $r > 0$ we have $M_f(r) \leq K \exp r^s$. Using the fact that $\log (a+b) = \log a + \log (1 + \frac{b}{a})$ we get
    \begin{align*}
        \frac{\log \log M_f(r)}{\log r} &\leq \frac{\log (r^s + \log K)}{\log r} = s + \frac{\log (1 + (\log K) / r^s)}{\log r}. \tag{\textasteriskcentered} \label{eq:proof-order-infimum-1}
    \end{align*}
    Since
    \begin{align*}
        0 \leq \frac{\log (1 + (\log K) / r^s)}{\log r} \leq \frac{(\log K) / r^s}{\log r} = \frac{\log K}{r^s \log r} \xrightarrow{r \to \infty} 0,
    \end{align*}
    by taking the limit superior as $r \to \infty$ in \eqref{eq:proof-order-infimum-1} and then letting $s \to \rho$ we obtain $\rho_f \leq \rho$.

    Now suppose $0 \leq \rho_f < \infty$ and let $s > \rho_f$. Then, by definition of the limit superior, for all sufficiently large $r > 0$ we have $ \log \log M_f(r) \leq s \log r$ and therefore $M_f(r) \leq \exp r^s$. Thus $M_f(r) = \O(\exp r^s)$, thereby $\rho \leq s$ and letting $s \to \rho_f$ yields $\rho \leq \rho_f$.
\end{proof}

\begin{remark} \label{rem:order-zero}
    Any polynomial is of order zero. Indeed, let $Q(z) = \sum_{k=0}^n a_k z^k$ be a polynomial and $m \in \N$. Then for any $r > 1$ we have
    $$ \exp r^{1/m} = \sum_{k=0}^\infty \frac{r^{k/m}}{k!} > \frac{r^{n}}{(mn)!} $$
    and therefore
    $$ M_Q(r) = \max_{\vert z \vert = r} \vert Q(z) \vert \leq \left( \sum_{k=0}^n \vert a_k \vert \right) r^n \leq \left( (mn)! \sum_{k=0}^n \vert a_k \vert \right) \exp r^{1/m}. $$
    Since $m \in \N$ was arbitrary, \Cref{prop:order-infimum} gives $\rho_Q = 0$.

    There are, however, non-polynomial functions of order zero; one may simply construct an entire function with power series coefficients of sufficiently rapid decay. Consider
    $$ f(z) \coloneqq \sum_{k=0}^\infty \frac{z^k}{(k^2)!} $$
    and let $m \in \N$. By the above, we have $\sum_{k=0}^{m-1} \frac{r^k}{(k^2)!} \leq K \exp r^{1/m}$ for some constant $K > 0$ depending only on $m$, holding for all $r > 1$. Thus,
    $$ \sum_{k=0}^\infty \frac{r^k}{(k^2)!} \leq K \exp r^{1/m} + \sum_{k=m}^\infty \frac{r^k}{(km)!} \leq K \exp r^{1/m} + \sum_{k=0}^\infty \frac{r^{k/m}}{k!} = (K + 1) \exp r^{1/m} $$
    and we obtain $\rho_f = 0$, as above.
\end{remark}

From \Cref{prop:order-infimum}, together with some rough estimates, it follows that the order of the sum or product of two entire functions is bounded above by the larger of the two:

\begin{proposition} \label{prop:order-sum-product-estimate}
    Let $f, g \in H(\C)$ be of finite order. Then it holds that:
    \begin{enumerate}[i.]
        \item $\rho_{f + g} \leq \max \{ \rho_f, \rho_g \}$.
        \item $\rho_{fg} \leq \max \{ \rho_f, \rho_g \}$.
    \end{enumerate}
\end{proposition}

For two entire functions of different, finite order, their sum has the order of the higher of the two:

\begin{proposition} \label{prop:order-sum-equality}
    Let $f, g \in H(\C)$ be of finite order. If $\rho_f < \rho_g$, then $\rho_{f + g} = \rho_g$.
\end{proposition}

\begin{proof}
    By \Cref{prop:order-sum-product-estimate} we have 
    \begin{equation*}
        \rho_{f+g} \leq \max \{ \rho_f, \rho_g \} = \rho_g, \quad\textrm{and}\quad \rho_g = \rho_{f+g+(-f)} \leq \max \{ \rho_f, \rho_{f+g} \} = \rho_{f+g},
    \end{equation*}
    since if $\rho_f > \rho_{f+g}$ then $\rho_f > \rho_{f+g} \geq \rho_g$ would be a contradiction. Thus $\rho_{f+g} = \rho_g$, proving the assertion.
\end{proof}

It is of note that \Cref{prop:order-sum-equality} implies that the order of an entire function of finite order remains unchanged when adding a polynomial of arbitrary degree to it.

There is a, perhaps surprising, connection between the order of an entire function and the order of its derivative:

\begin{proposition} \label{prop:order-derivative}
    If $f \in H(\C)$ is of finite order, then $\rho_{f'} = \rho_f$.
\end{proposition}

\begin{proof}
    Without loss of generality we may assume $f(0) = 0$. If $f$ is a polynomial the assertion is clear. Otherwise for any $r > 0$ we have,
    \begin{align*}
        M_f(r) &= \max_{\vert z \vert = r} \vert f(z) \vert = \max_{\vert z \vert = r} \left\vert \int_{[0, z]} f'(\zeta) \diff \zeta \right\vert \leq \\
        &\leq \max_{\vert z \vert = r} \left( \vert z \vert \max_{w \in \cl{B_r(0)}} \vert f'(w) \vert \right) \leq r M_{f'}(r).
    \end{align*}
    Thus
    \begin{align*}
        \frac{\log \log M_f(r)}{\log r} &\leq \frac{\log \log r M_{f'}(r)}{\log r} = \frac{\log \log M_{f'}(r)}{\log r} + \frac{\log(1 + \log r / \log M_{f'}(r))}{\log r} \leq \\
        &\leq \frac{\log \log M_{f'}(r)}{\log r} + \frac{\log r / \log M_{f'}(r)}{\log r} = \frac{\log \log M_{f'}(r)}{\log r} + \frac{1}{\log M_{f'}(r)}
    \end{align*}
    and taking limits superior as $r \to \infty$ yields $\rho_f \leq \rho_{f'}$.

    On the other hand, Cauchy's integral formula gives
    \begin{align*}
        M_{f'}(r) &= \max_{z \in \partial B_r(0)} \vert f'(z) \vert = \max_{z \in \partial B_r(0)} \left\vert \frac{1}{2\pi i} \oint_{\partial B_1(z)} \frac{f(\zeta)}{(\zeta-z)^2} \diff \zeta \right\vert \leq \\
        &\leq \max_{\substack{z \in \partial B_r(0) \\ w \in \partial B_1(z)}} \vert f(w) \vert \leq \max_{z \in \cl{B_{r+1}(0)}} \vert f(z) \vert = M_f(r + 1).
    \end{align*}
    Therefore
    \begin{align*}
        \frac{\log \log M_{f'}(r)}{\log r} \leq \frac{\log \log M_f(r+1)}{\log (r+1)} \frac{\log (r+1)}{\log r}
    \end{align*}
    and, again, taking limits superior as $r \to \infty$ we get $\rho_{f'} \leq \rho_f$.
\end{proof}

For functions of finite and positive order, we can obtain a natural refinement of the concept of order:

\begin{definition}
    Let $f \in H(\C)$ be of order $0 < \rho_f < \infty$. The \emph{type} of $f$ is defined by
    \begin{equation} \label{eq:def-type}
        \tau_f \coloneqq \limsup_{r \to \infty} \frac{\log M_f(r)}{r^{\rho_f}}.
    \end{equation}

    If $\tau_f = 0$, then $f$ is said to be of \emph{minimal} type, if $0 < \tau_f < \infty$, of \emph{normal} type, and if $\tau_f = \infty$, of \emph{maximal} type.
\end{definition}

Once again, we have an equivalent characterization for functions of finite type:

\begin{proposition} \label{prop:type-infimum}
    Let $f \in H(\C)$ be of finite, positive order, then $f$ is of finite type if and only if
    \begin{equation}
        \tau \coloneqq \inf \{ t > 0 : M_f(r) = \O(\exp (t r^{\rho_f})) \textrm{ as } r \to \infty \}
    \end{equation}
    is finite, and in either case we have $\tau_f = \tau$.
\end{proposition}

\begin{proof}
    Suppose $0 \leq \tau < \infty$, then for all $t > \tau$ we have $M_f(r) = \O(\exp t r^{\rho_f})$ as $r \to \infty$. Thus there exists a constant $K > 0$ such that for all sufficiently large $r > 0$ we have $M_f(r) \leq K \exp t r^{\rho_f}$. Therefore
    \begin{align*}
        \frac{\log M_f(r)}{r^{\rho_f}} \leq \frac{\log K + t r^{\rho_f}}{r^{\rho_f}} = \frac{\log K}{r^{\rho_f}} + t
    \end{align*}
    and taking limits superior as $r \to \infty$ and letting $t \to \tau$ afterwards yields $\tau_f \leq \tau$.

    Now suppose $0 \leq \tau_f < \infty$ and let $t > \tau_f$. Then, by definition of the limit superior, for all sufficiently large $r > 0$ we have $ \log M_f(r) \leq t r^{\rho_f}$ and therefore $M_f(r) \leq \exp(t r^{\rho_f})$. Thus $M_f(r) = \O(\exp (t r^{\rho_f}))$, thereby $\tau \leq t$ and letting $t \to \tau_f$ yields $\tau \leq \tau_f$.
\end{proof}

% Equivalently, $\tau_f$ can also be defined as the infimum over all $\tau > 0$ such that $M_f(r) = \O(\exp(\tau r^{\rho_f}))$ as $r \to \infty$.

If $f \in H(\C)$ is of finite, positive order, then $f$ and $f'$ not only have the same order, they have the same type:

\begin{proposition} \label{prop:type-derivative}
    If $f \in H(\C)$ is of finite, positive order and finite type, then $\tau_{f'} = \tau_f$.
\end{proposition}

\begin{proof}
    Reusing the inequalities obtained in the proof of \Cref{prop:order-derivative} we have, for all $r > 0$,
    \begin{equation*}
        M_f(r) \leq r M_{f'}(r), \quad \textrm{and} \quad M_{f'}(r) \leq M_f(r+1).
    \end{equation*}
    Therefore
    \begin{align*}
        \frac{\log M_f(r)}{r^{\rho_f}} &\leq \frac{\log r + \log M_{f'}(r)}{r^{\rho_f}}, \quad \textrm{and} \quad \frac{\log M_{f'}(r)}{r^{\rho_f}} \leq \frac{\log M_{f}(r+1)}{(r+1)^{\rho_f}} \frac{(r+1)^{\rho_f}}{r^{\rho_f}}
    \end{align*}
    and, since by \Cref{prop:order-derivative} $\rho_f = \rho_{f'}$, taking limits superior as $r \to \infty$ yields $\tau_f \leq \tau_{f'}$ and $\tau_{f'} \leq \tau_f$.
\end{proof}

\begin{example} \label{exm:order-and-type}
    \leavevmode
    \begin{arrowlist}
        \item We have already seen in \Cref{rem:order-zero} that polynomials are of order zero.
        \item For $\rho, \tau \in (0, \infty)$, the function
        $$ f(z) \coloneqq \exp ({\tau z^\rho}) $$
        is of order $\rho$ and type $\tau$.
        \item The function
        $$ f(z) \coloneqq \exp \exp z $$
        is of infinite order. For any $m \in \N$
        $$ M_f(r) \geq \exp \exp r > \exp (r^m / m!), $$
        therefore $\rho_f \geq m$, and since $m$ was arbitrary it follows that $\rho_f = \infty$.
        \item The functions defined by
        $$ \sum_{n=2}^\infty \left( \frac{\log n}{n} \right)^n z^n, \quad \textrm{and} \quad \sum_{n=2}^\infty \left( \frac{1}{n \log n} \right)^n z^n $$
        are of order 1 and of maximal and minimal type, respectively. This can be seen via Theorem 2.1 in Chapter 3 of Segal \cite{segal-complex-analysis}.
        \qedhere
    \end{arrowlist}
\end{example}
\section{Order, Density and Exponent of Convergence of Zeros} \label{sec:zeros}

% We will mostly follow the contents of chapter 1 in Levin \cite{levin-distribution-of-zeros}.

We first recall a rather explicit connection between the moduli of the zeros of a holomorphic function and the modulus of the function itself:

\begin{theorem}[Jensen] \label{thm:jensen}
    Let $R > 0$, $f \in H(B_R(0))$ with $f(0) \neq 0$ and let $r_1, r_2, \hdots$ denote the moduli of the zeros of $f$ in $B_{R}(0)$ arranged in a non-decreasing sequence. Then, for $r_n < r < r_{n+1}$, we have
    \begin{equation}
        \frac{1}{2 \pi} \int_0^{2\pi} \log \vert f(r e^{i \theta}) \vert \diff \theta = \log \vert f(0) \vert + \log \frac{r^n}{r_1 \hdots r_n}.
    \end{equation}
\end{theorem}

\begin{definition}
    Let $f \in H(B_R(0))$ be not identically zero. Then, for $0 < r < R$, we define
    \begin{equation}
        n_f(r) \coloneqq \vert \{ z \in \cl{B_r(0)} : f(z) = 0 \} \vert,
    \end{equation}
    that is, the number of zeros of $f$ in $\cl{B_{r}(0)}$.
\end{definition}

This zero-counting function is, in some scenarios, simpler to work with than the sequence of zeros. For instance, it can be used to obtain an equivalent version of Jensen's Theorem:

\begin{corollary} \label{cor:jensen-nf}
    Let $f \in H(B_R(0))$ with $f(0) \neq 0$. Then, for $0 < r < R$, we have
    \begin{equation}
        \frac{1}{2 \pi} \int_0^{2\pi} \log \vert f(re^{i \theta}) \vert \diff \theta = \log \vert f(0) \vert + \int_0^r \frac{n_f(s)}{s} \diff s
    \end{equation}
\end{corollary}

\begin{proof}
    Let $r_1, r_2, \hdots$ denote the moduli of the zeros of $f$ in $B_{R}(0)$ arranged in a non-decreasing sequence. Then, for any $r_n < r < r_{n+1}$, we obtain
    \begin{align*}
        \log \frac{r^n}{r_1 \hdots r_n} &= \sum_{k=1}^n \log \frac{r}{r_k} = \sum_{k=1}^n \int_{r_k}^r \frac{1}{s} \diff s = \\
        &= \sum_{k=1}^n \int_0^r \1_{(r_k, \infty)}(s) \frac{1}{s} \diff s = \int_0^r \left( \sum_{k=1}^n \1_{(r_k, \infty)}(s) \right) \frac{1}{s} \diff s = \\
        &= \int_0^r \frac{n_f(s)}{s} \diff s
    \end{align*}
    and \Cref{thm:jensen} establishes the assertion.
\end{proof}

This gives an immediate connection between the zeros and the growth of the modulus of an entire function:

\begin{lemma} \label{lem:zeros-bounded-by-modulus}
    If $f \in H(\C)$, then
    $$ n_f(r) = \O( \log M_f(er) ). $$
\end{lemma}

\begin{proof}
    We may assume $\vert f(0) \vert = 1$. Let $r > 0$, then since $n_f$ is non-negative and non-decreasing we have
    \begin{equation*}
        n_f(r) \leq \int_r^{er} \frac{n_f(t)}{t} \diff t \leq \frac{1}{2 \pi} \log \vert f(er e^{i \theta}) \vert \diff \theta \leq \log M_f(er),
    \end{equation*}
    establishing the assertion.
\end{proof}

\Cref{lem:zeros-bounded-by-modulus} shows that the more zeros a function $f$ has, the faster $M_f$ must grow as $r \to \infty$. The converse is naturally false; by composing exponentials one can obtain an entire function $f$ for which $M_f$ grow arbitrarily fast, yet $f$ has no zeros.

\begin{definition} \label{def:zero-exponent}
    Let $M \subseteq \N$ and $\mathbf{z} = (z_m)_{m \in M}$ be a family of points\footnote{Note that the exponent of convergence is only affected by their moduli, not by their arguments.} in $\C^\times$. Then
    $$ \lambda_{\mathbf{z}} \coloneqq \inf \left\{ \lambda > 0 : \sum_{m \in M} \frac{1}{\vert z_m \vert^\lambda} < \infty \right\} $$
    is called the \emph{exponent of convergence} of the family $\mathbf{z}$.

    Let $f \in H(\C)$, then the \emph{exponent of convergence of the zeros of $f$} is defined as the exponent of convergence of the non-zero roots of $f$ and is denoted $\lambda_f$. If $f$ is constant, we set $\lambda_f = 0$ by convention.

    Furthermore, for any $a \in \C$, the \emph{exponent of convergence of the $a$-points of $f$} is defined as exponent of convergence of the zeros of $f(z) - a$ and will be denoted $\lambda_f^{(a)}$.
\end{definition}

If $f \in H(\C)$ assumes some value $a \in \C$ only finitely often, then clearly $\lambda_f^{(a)} = 0$.

\begin{theorem} \label{thm:inequality-order-exponent-of-convergence}
    If $f \in H(\C)$ is of finite order, then $\lambda_f \leq \rho_f$.
\end{theorem}

\begin{proof}
    We may assume $f(0) \neq 0$. If $\lambda_f = 0$ there is nothing to show. Let $\rho > \rho_f$ then
    by \Cref{lem:zeros-bounded-by-modulus} and \Cref{prop:order-infimum} we have $n_f(r) = \O(\log M_f(er))$ and $M_f(r) = \O(\exp r^\rho)$, respectively. Therefore $n_f(r) = \O(r^\rho)$, thus there exists a constant $K_1 > 0$ such that for sufficiently large $r > 0$ we have $n_f(r) \leq K_1 r^\rho$.
    
    If $(r_j)_{j \in \N}$ denote the non-zero moduli of the zeros of $f$, arranged in non-decreasing order, then for any $m \in \N$ we have $m \leq n_f(r_m) \leq K_1 r_m^{\rho}$. Let $0 < \lambda < \lambda_f$ then this implies
    \begin{align*}
        \left( \frac{1}{r_m} \right)^{\lambda} \leq K_2 \left( \frac{1}{m} \right)^{\lambda/\rho}.
    \end{align*}
    for some constant $K_2 > 0$. Therefore
    \begin{equation*}
        \sum_{m=1}^\infty \left( \frac{1}{r_m} \right)^{\lambda} \leq K_2 \sum_{m=1}^\infty \left( \frac{1}{m} \right)^{\lambda/\rho}.
    \end{equation*}
    The left-hand side diverges by \Cref{def:zero-exponent}, thus so does the right-hand side. But the latter diverges if and only if $\frac{\lambda}{\rho} \leq 1$, therefore letting $\lambda \nearrow \lambda_f$ and then $\rho \searrow \rho_f$ yields $\lambda_f \leq \rho_f$.
\end{proof}

\begin{remark}
    The function $f(z) \coloneqq \exp z$ is of order one and has no zeros -- thus we observe that we may have $\lambda_f < \rho_f$ in some cases.
\end{remark}

A more precise analysis of the density of the zeros of an entire function $f$ is contained in the growth of the zero-counting function $n_f$. We therefore introduce the following definitions:

\begin{definition} \label{def:order-density-zero-counting}
    Let $f \in H(\C)$. As the \emph{order} of the zero-counting function $n_f$ we define
    \begin{equation}
        \nu_f \coloneqq \limsup_{r \to \infty} \frac{\log n_f(r)}{\log r}.
    \end{equation}
    If $\nu_f$ is finite, we additionally define the \emph{upper density} of the zeros of $f$ as
    \begin{equation}
        \Delta_f \coloneqq \limsup_{r \to \infty} \frac{n_f(r)}{r^{\nu_f}}.
    \end{equation}
    If the limit exists, then $\Delta_f$ is simply called the \emph{density}.
\end{definition}

The following lemma provides great utility in further analyzing the function $n_f$.

\begin{lemma} \label{lem:zeros-measure}
    Let $f \in H(\C)$ and denote by $(r_m)_{m \in M}$ the moduli of the zeros of $f$ arranged in non-decreasing order. Then $f$ induces a unique -Stieltjes measure $\omega_f$ on $(0, \infty)$, such that, for any non-negative, measureable function $\varphi$ on $(0, \infty)$, it holds that
    \begin{equation*}
        \sum_{m \in M} \varphi(r_m) = \int_{(0, \infty)} \varphi \diff \omega_f.
    \end{equation*}
    If in addition $\varphi$ is monotone and continuously differentiable on $(0, \infty)$, then for all $r > 0$ it holds that
    \begin{equation*}
        \int_{(0, r)} \varphi \diff \omega_f = n_f(r) \varphi(r) - \int_0^r n_f(t) \varphi'(t) \diff t.
    \end{equation*}
\end{lemma}

\begin{proof}
    From the representation
    $$ n_f(r) = \sum_{m \in M} \1_{[r_m, \infty)}(r) $$
    we observe that $n_f$ is right-continuous, non-negative and non-decreasing. Therefore $n_f$ induces a unique Lebesgue--Stieltjes measure $\omega_f$. Since an indicator function of the form $\1_{[a, \infty)}$, for $a \in \R$, corresponds to a Dirac measure $\delta_{\{ a \}}$, and the correspondence between Lebesgue-Stieltjes measures and their distribution functions is linear in nature, we have
    $$ \omega = \sum_{m \in M} \delta_{\{r_m\}}. $$
    From this it follows that
    \begin{equation*}
        \sum_{m \in M} \varphi(r_m) = \sum_{m \in M} \int_{(0, \infty)} \varphi \diff \delta_{\{r_m\}} = \int_{(0, \infty)} \varphi \diff \left( \sum_{m \in M} \delta_{\{r_m\}} \right) = \int_{(0, \infty)} \varphi \diff \omega_f.
    \end{equation*}
    Assume $\varphi$ is monotone and continuously differentiable on $(0, \infty)$. Then $\varphi$ also induces a Lebesgue--Stieltjes measure $\eta_\varphi$, with density $\varphi'$ with respect to the Lebesgue measure. Thus, for any $0 < \varepsilon < r < \infty$, integration by parts gives
    \begin{equation*}
        \int_{(\varepsilon, r)} \varphi \diff \omega_f + \int_{(\varepsilon, r)} n_f \diff \eta_\varphi = n_f(r) \varphi(r) - n_f(\varepsilon) \varphi(\varepsilon).
    \end{equation*}
    Since $\kappa \coloneqq \min_{m \in M} r_m > 0$ we have $n_f(s) = 0$ for $0 < s < \kappa$. Letting $\varepsilon \to 0$ and using the density of $\eta_\varphi$ therefore yields
    \begin{equation*}
        \int_{(0, r)} \varphi \diff \omega_f + \int_0^r n_f(t) \varphi'(t) \diff t = n_f(r) \varphi(r)
    \end{equation*}
    and rearranging terms provides the desired result.
\end{proof}

\begin{proposition} \label{prop:zeros-order-equals-zeros-exponent}
    Let $f \in H(\C)$, then $\lambda_f$ is finite if and only if $\nu_f$ is finite and in either case we have $\lambda_f = \nu_f$.
\end{proposition}

\begin{proof}
    If $f$ only has finitely many zeros the assertion is clear. Let $\lambda > \lambda_f$ and $r > 0$. Invoking \Cref{lem:zeros-measure} with $\varphi(t) \coloneqq t^{-\lambda}$ we get
    \begin{equation*}
        \int_{(0, r)} \frac{1}{t^\lambda} \diff \omega_f(t) = \frac{n_f(r)}{r^\lambda} + \lambda \int_0^r \frac{n_f(t)}{t^{\lambda + 1}} \diff t.
    \end{equation*}
    Since $\lambda > \lambda_f$, the same lemma implies that the left integral converges as $r \to \infty$. In particular, the right-hand side must be bounded. Since the right integrand is non-negative, the right-most term is increasing in $r$ and must therefore converge as $r \to \infty$. Therefore the right-most term in
    $$ 0 \leq \frac{n_f(r)}{r^\lambda} = \lambda n_f(r) \int_r^\infty \frac{1}{t^{\lambda + 1}} \diff t \leq \lambda \int_r^\infty \frac{n_f(t)}{t^{\lambda + 1}} \diff t $$
    converges to $0$ as $r \to \infty$, from which we get
    \begin{equation} \label{eq:zeros-order-equals-zeros-exponent:density-zero}
        \lim_{r \to \infty} \frac{n_f(r)}{r^\lambda} = 0.
    \end{equation}
    Taking limits superior as $r \to \infty$ in
    $$ \frac{\log n_f(r)}{\log r} = \left( \log \frac{n_f(r)}{r^\lambda} + \log r^\lambda\right)/{\log r} = \log \frac{n_f(r)}{r^\lambda} / \log r + \lambda $$
    and then letting $\lambda \searrow \lambda_f$ yields $\nu_f \leq \lambda_f$.

    Conversely suppose $\nu_f < \infty$ and let $\varepsilon > 0$. By the definition of $\nu_f$ and the nature of the limit superior there is an $r_0 > 0$ such that
    $$ n_f(r) \leq r^{\nu_f + \varepsilon} $$
    for all $r \geq r_0$. Set $\lambda \coloneqq \nu_f + 2 \varepsilon$, then
    \begin{equation*}
        \int_{r_0}^r \frac{n_f(t)}{t^{\lambda + 1}} \diff t \leq
        \int_{r_0}^r \frac{t^{\nu_f + \varepsilon}}{t^{\nu_f + 2\varepsilon + 1}} \diff t =
        \int_{r_0}^r t^{-\varepsilon - 1} \diff t = \frac{1}{\varepsilon r_0^{\varepsilon}} - \frac{1}{\varepsilon r^{\varepsilon}}
    \end{equation*}
    and since the right-hand side remains finite as $r \to \infty$ therefore the leftmost integral converges as $r \to \infty$. Clearly \eqref{eq:zeros-order-equals-zeros-exponent:density-zero} holds for our $\lambda$, therefore \Cref{lem:zeros-measure} implies the convergence of
    \begin{equation*}
        \int_{(0, \infty)} \frac{1}{t^\lambda} \diff \omega_f(t) = \sum_{m \in M} \frac{1}{r_m^\lambda},
    \end{equation*}
    yielding $\lambda_f \leq \lambda$, and letting $\varepsilon \to 0$ results in $\lambda_f \leq \nu_f$.
\end{proof}

\iffalse
\begin{example} \label{exm:exponent-of-convergence}
    Consider $f(z) \coloneqq \sin(z) \in H(\C)$, we want to calculate $\lambda_f$ and $\rho_f$. First, let $\lambda > 0$ and recall that $f$ has zeros at $(n \pi)_{n \in \Z}$. Since
    $$ \sum_{n \in \Z \setminus \{0\}} \frac{1}{\vert n \pi \vert^\lambda} = \frac{2}{\pi^\lambda} \sum_{n=1}^\infty \frac{1}{n^\lambda} $$
    is finite if and only if $\lambda > 1$, we obtain $\lambda_f = 1$. Furthermore, we have $\vert \sin(z) \vert \leq e^{\vert z \vert}$
    and therefore $M_f(r) \leq e^r$, whereby $\rho_f \leq 1$. Finally, \Cref{thm:inequality-order-exponent-of-convergence} concludes $\rho_f = 1$.
\end{example}
\fi
\section{Canonical Products} \label{sec:canonical-products}

By the Fundamental Theorem of Algebra every complex polynomial can be written as a product of linear factors involving only its roots and some scaling factor. Conversely, for a finite sequence of points $z_1, \hdots, z_n$ the polynomial $p(z) \coloneqq \prod_{k=1}^n (z - z_k)$ vanishes only at said points.

The question arises if one can construct an entire function that vanishes precisely at the points of some infinite sequence $(z_k)_{k \in \N}$ and nowhere else. Clearly the product $\prod_{k=1}^\infty (z - z_k)$ need not be convergent in the entire complex plane. This issue is solved by introducing additional exponential factors, which improve convergence without introducing new zeros. We first study sufficient conditions for convergence of infinite products of numbers and holomorphic functions.

The early parts of the section will closely follow Stein and Shakarchi \cite{stein-shakarchi-princeton}, while the later estimates and results will follow Levin \cite{levin-distribution-of-zeros}.

\begin{lemma} \label{lem:infinite-product-criteria}
    Suppose $G \subseteq \C$ is open and let $(f_k)_{k \in \N}$ be a sequence of non-vanishing functions in $H(G)$. If $\sum_{k=1}^\infty (1 - f_k(z))$ converges absolutely compactly in $G$, then $\prod_{k=1}^\infty f_k(z)$ converges compactly in $G$ to a non-vanishing $f \in H(G)$.
\end{lemma}
    
\begin{proof}
    Let $K \subset G$ be compact, then the absolute, compact convergence of the sum implies that there is a $k_0$ such that $\vert 1 - f_k(z) \vert \leq 1/2$ for $z \in K, k > k_0$. For $\vert w \vert \leq 1/2$ we have
    \begin{equation*}
        \vert \log(1 - w) \vert \leq \sum_{k=1}^\infty \left\vert \frac{1}{k} w^k \right\vert \leq \vert w \vert \sum_{k=0}^\infty \vert w \vert^k = \frac{\vert w \vert}{1 - \vert w \vert} \leq 2 \vert w \vert.
    \end{equation*}
    Therefore
    \begin{equation*}
        \sum_{k=1}^\infty \vert \log f_k(z) \vert = \sum_{k=1}^\infty \vert \log (1 - (1 - f_k(z))) \vert \leq \sum_{k=1}^{k_0 - 1} \vert \log f_k(z) \vert + 2 \sum_{k=k_0}^\infty \vert 1 - f_k(z) \vert
    \end{equation*}
    and since the rightmost sum converges compactly it follows that $\sum_{k=1}^\infty \log f_k(z)$ converges absolutely compactly in $G$. Since the exponential function is uniformly continuous, the sequence
    $$ \exp \left( \sum_{k=k_0}^n \log f_k(z) \right) = \prod_{k=k_0}^n f_k(z), \quad n \in \N $$
    converges absolutely compactly to some $f \in H(G)$. Again by continuity of the exponential function we have
    $$ f(z) = \prod_{k=1}^\infty f_k(z) = \exp \left( \sum_{k=1}^\infty \log f_k(z) \right) \neq 0 $$
    and thus $f$ is non-vanishing.
\end{proof}

\begin{definition} \label{def:canonical-factors}
    The \emph{canonical factors} are defined at $z \in \C$ by
    $$ W(z; 0) \coloneqq 1-z, \quad \textrm{and} \quad W(z; n) \coloneqq (1-z) \exp \left( \sum_{k=1}^n \frac{z^k}{k} \right), \quad n \in \N. $$
    The number $n$ is called the \emph{degree} of the canonical factor.
\end{definition}

\begin{lemma} \label{lem:estimate-canonical-factors}
    Let $n \in \N_0$ and $\vert z \vert \leq 1/2$, then $ \vert 1 - W(z; n) \vert \leq 2e \vert z \vert^{n+1} $.
\end{lemma}

\begin{proof}
    For $n = 0$ there is nothing to show. Note that for such given $z$ we can expand $\log (1 - z) = - \sum_{k=1}^\infty \frac{z^k}{k}$, therefore
    \begin{equation*}
        W(z; n) = \exp \left( \log(1-z) + \sum_{k=1}^n \frac{z^k}{k} \right) = \exp \zeta,
    \end{equation*}
    where $\zeta \coloneqq - \sum_{k=n+1}^\infty z^k / k$. Furthermore, we have that
    \begin{equation*}
        \vert \zeta \vert \leq \vert z \vert^{n+1} \sum_{k=n+1}^\infty \vert z \vert^{k-n-1} / k \leq \vert z \vert^{n+1} \sum_{j=0}^\infty 2^{-j} \leq 2 \vert z \vert^{n+1}
    \end{equation*}
    and in particular $\vert \zeta \vert \leq 1$. We conclude
    \begin{equation*}
        \vert 1 - W(z; n) \vert = \vert \exp \zeta - 1 \vert \leq \sum_{k=1}^\infty \frac{\vert \zeta \vert^k}{k!} \leq \vert \zeta \vert \sum_{k=0}^\infty \frac{1}{k!} \leq 2 e \vert z \vert^{n+1}.
    \end{equation*}
\end{proof}

The next theorem -- which is also known as the \emph{Weierstraß Product Theorem} -- now constructively asserts the existence of entire function with prescribed zeros.

\begin{theorem} \label{thm:function-with-prescribed-zeros}
    Let $M \subseteq \N$ and $\mathbf{z} = (z_m)_{m \in M}$ be a family in $\C^\times$ without accumulation points. Then there is a family $(p_m)_{m \in M}$ in $\N_0$ such that
    \begin{equation}
        \Pi(z) \coloneqq \prod_{m \in M} W\left(\frac{z}{z_m}; p_m\right)
    \end{equation}
    is an entire function that vanishes at all $z_m, m \in M$ and nowhere else. If
    \begin{equation}
        \mu \coloneqq \min \left\{ p \in \N_0 : \sum_{m \in M} \frac{1}{\vert z_m \vert^{p+1}} < \infty \right\}
    \end{equation}
    is finite, then one can take $p_m = \mu, m \in M$.
\end{theorem}

\begin{proof}
    If $M$ is finite there is nothing to show. We may thus assume that $M$ is infinite and without loss of generality $M = \N$. Thus $\mathbf{z}$ is a sequence, which we can assume to be arranged in non-decreasing order according to the moduli its elements.

    We claim that there exists a sequence $(p_n)_{n \in \N}$ in $\N_0$ such that
    \begin{equation}
        \sum_{n=1}^\infty \left\vert \frac{z}{z_n} \right\vert^{p_n + 1}
    \end{equation}
    converges compactly in $\C$. Let $K \subset B_R(0) \subset \C$ be compact. Note that since $\mathbf{z}$ has no accumulation points we necessarily have $\lim_{n \to \infty} \vert z_n \vert = \infty$, thus there is some $n_0$ such that for $n > n_0$ we have $\vert z_n \vert \geq R + 1$. Therefore, for any $z \in K$,
    \begin{equation*}
        \sum_{n=1}^\infty \left\vert \frac{z}{z_n} \right\vert^{p_n + 1} \leq \sum_{n=1}^{n_0 - 1}  \left\vert \frac{z}{z_n} \right\vert^{p_n + 1} + \sum_{n=n_0}^\infty \left( \frac{R}{R+1} \right)^{p_n + 1},
    \end{equation*}
    and the rightmost series is finite if, for example, $p_n = n, n \in \N$. Therefore the series is uniformly bounded and thus compactly convergent, concluding our claim. Note that if $\mu$ is finite then we can take $p_n = \mu, n \in \N$, since for $z \in K$
    \begin{equation*}
        \sum_{n=1}^\infty \left\vert \frac{z}{z_n} \right\vert^{\mu + 1} \leq \vert z \vert^\mu \sum_{n=1}^\infty \frac{1}{\vert z_n \vert^{\mu+1}} \leq R^\mu \sum_{n=1}^\infty \frac{1}{\vert z_n \vert^{\mu+1}},
    \end{equation*}
    where the rightmost sum is a finite constant by the definition of $\mu$.

    It remains to show that is $\Pi$ is an entire function that satisfies the desired properties.Let $R > 0$, then it suffices to show that $\Pi$ has is holomorphic in $B_R(0)$ and vanishes only at points of $\mathbf{z}$ that are also in $B_R(0)$. We write
    $$ \Pi(z) = \left( \prod_{\substack{n=1 \\ \vert z_n \vert < 2R }}^\infty W\left(\frac{z}{z_n}; p_n\right) \right) \left( \prod_{\substack{n=1 \\ \vert z_n \vert \geq 2R }}^\infty W\left(\frac{z}{z_n}; p_n\right) \right) \eqqcolon \Pi_1(z) \Pi_2(z). $$
    Clearly $\Pi_1$ is a finite product which vanishes at points of $\mathbf{z}$ in $B_R(0)$ and nowhere else. Note that for $\vert z_m \vert \geq 2R$ we have $\vert z / z_m \vert \leq R / \vert z_m \vert \leq 1 / 2$, hence by \Cref{lem:estimate-canonical-factors} we have
    $$ \sum_{\substack{n=1 \\ \vert z_n \vert \geq 2R }}^\infty \left\vert 1 - W\left(\frac{z}{z_n}; p_n\right) \right\vert \leq 2e \sum_{\substack{n=1 \\ \vert z_n \vert \geq 2R }}^\infty \left\vert \frac{z}{z_n} \right\vert^{p_n + 1} $$
    and the compact convergence of the right sum, which stems from our choice of the $p_n, n \in \N$, implies compact convergence of the left sum. By \Cref{lem:infinite-product-criteria} we thereby have that $\Pi_2 \in H(B_R(0))$ and vanishes nowhere. Therefore $\Pi$ has the desired properties and, since $R$ was arbitrary, this concludes the proof.
\end{proof}

\begin{definition} \label{def:canonical-product}
    With the notation of \Cref{thm:function-with-prescribed-zeros}, $\Pi$ is called a \emph{Weierstraß product}\footnote{Note that this product is not uniquely determined, since the sequence $(p_m)_{m \in M}$ is not unique.} formed from the family $\mathbf{z}$. If $\mu < \infty$, then
    \begin{equation}
        \Pi(z) \coloneqq \prod_{m \in M} W\left(\frac{z}{z_m}; \mu \right)
    \end{equation}
    is called a \emph{(Weierstraß) canonical product} and $\mu$ is called the \emph{genus} of the canonical product, which will also be denoted as $\mu_\Pi$. Otherwise the genus of $\Pi$ is said to be infinite.
\end{definition}

\begin{remark}
    By comparing \Cref{def:zero-exponent} and \Cref{def:canonical-product} we immediately observe that if $\lambda_\Pi < \infty$, then $\mu_\Pi \leq \lambda_\Pi \leq \mu_\Pi + 1$. If therefore $\lambda_\Pi$ is an integer, and $\Pi$ has infinitely many zeros, the series
    $$ \sum_{m \in M} \frac{1}{\vert z_m \vert^{\lambda}}, $$
    will converge if $\lambda_\Pi = \mu_\Pi + 1$ and diverge if $\lambda_\Pi = \mu_\Pi$.
\end{remark}

Just as any polynomial decomposes into linear factors involving their roots by the Fundamental Theorem of Algebra, Weierstraß products allow us to obtain a similar factorization result for arbitrary entire functions, known as the \emph{Weierstraß Factorization Theorem}.

\begin{theorem}[Weierstraß] \label{thm:Weierstraß}
    Let $f \in H(\C)$ be not identically zero and let $m \in \N_0$ be the order of the zero of $f$ at the origin. Then there exists $g \in H(\C)$ such that
    \begin{equation}
        f(z) = z^m e^{g(z)} \Pi(z),
    \end{equation}
    where $\Pi$ denotes a Weierstraß product formed from the zeros of $f$.
\end{theorem}

\begin{proof}
    Since $f(z)$ and $z^m \Pi(z)$ have the same zeros with identical multiplicities the function
    $$ \varphi(z) \coloneqq \frac{f(z)}{z^m \Pi(z)} $$
    is entire and vanishes nowhere. Therefore we can write
    $$ \varphi(z) = e^{g(z)} $$
    for some $g \in H(\C)$ and rearranging terms yields the desired representation.
\end{proof}

\begin{definition} \label{def:genus}
    In the context of Weierstraß' Theorem, if $\Pi$ is of finite genus and $g$ is a polynomial, then the \emph{genus of the entire function $f$} is defined as
    \begin{equation}
        \mu_f \coloneqq \max \{ \mu_\Pi, \deg g \}.
    \end{equation}

    If $g$ is not a polynomial or if $\Pi$ is of infinite genus, then $f$ is said to be of infinite genus.
\end{definition}

We wish to further estimate canonical products. To do this we first focus on the canonical factors.

\begin{lemma} \label{lem:elementary-factor-estimate}
    For $p \in \N$ and all $z \in \C$ we have
    \begin{equation}
        \log \vert W(z; p) \vert < A_p \frac{\vert z \vert^{p+1}}{1 + \vert z \vert}, \quad \textrm{where} \quad A_p \coloneqq 3e (2 + \log p).
    \end{equation}
    For $p = 0$ we have
    \begin{equation}
        \log \vert W(z; 0) \vert \leq \log (1 + \vert z \vert).
    \end{equation}
\end{lemma}

\begin{proof}
    The second assertion is clear since
    $$ \log \vert W(z; 0) \vert = \log \vert 1 - z \vert \leq \log (1 + \vert z \vert). $$
    If $\vert z \vert \leq \frac{p}{p+1}$ then in particular we have $\vert z \vert < 1$. Since furthermore $1 + \vert z \vert < 2 < A_p$ we get
    \begin{align*}
        \log \vert W(z; p) \vert &= \log \left\vert (1-u) \exp \left( \sum_{k=1}^p \frac{z^k}{k} \right) \right\vert = \Re \left( \log(1-u) + \sum_{k=1}^p \frac{z^k}{k} \right) = \\
        &= -\Re \sum_{k=p+1}^\infty \frac{z^k}{k} \leq \sum_{k=p+1}^\infty \frac{\vert z \vert^k}{k} < \frac{\vert z \vert^{p+1}}{(p+1)(1 - \vert z \vert)} \leq \vert z \vert^{p+1} \leq \\ &\leq A_p \frac{\vert z \vert^{p+1}}{1 + \vert z \vert}.
    \end{align*}
    If otherwise $\vert z \vert > \frac{p}{p+1}$ then we have $\frac{1}{\vert z \vert} < 1 + \frac{1}{p}$ and $2 \vert z \vert > 1$. From there we conclude
    \begin{align*}
        \log \vert W(z; p) \vert &= \log \left\vert (1-z) \exp \left( \sum_{k=1}^p \frac{z^k}{k} \right) \right\vert \leq \log(1 + \vert z \vert) + \sum_{k=1}^p \frac{\vert z \vert^k}{k} \leq \\
        &\leq 2 \vert z \vert + \sum_{k=2}^p \frac{\vert z \vert^k}{k} = \vert z \vert^p \left( 2 \left\vert \frac{1}{z} \right\vert^{p-1} + \sum_{k=2}^p \frac{1}{k} \left\vert \frac{1}{z} \right\vert^{p-k} \right) < \\
        &< \vert z \vert^p \left( 2 \left(1 + \frac{1}{p} \right)^{p-1} + \sum_{k=2}^p \frac{1}{k} \left( 1 + \frac{1}{p} \right)^{p-k} \right) < \\
        &< \vert z \vert^p \left( 1 + \frac{1}{p} \right)^p \left( 2 + \sum_{k=1}^p \frac{1}{k} \right) < \vert z \vert^p e \left( 2 + \int_1^p \frac{1}{t} \diff t \right) = \\
        &= \vert z \vert^p e \left( 2 + \log p \right) = \frac{1 + \vert z \vert}{1 + \vert z \vert} \vert z \vert^p e \left( 2 + \log p \right) < \\
        &< \frac{3 \vert z \vert}{1 + \vert z \vert} \vert z \vert^p e \left( 2 + \log p \right) = A_p \frac{\vert z \vert^{p+1}}{1 + \vert z \vert}. \qedhere
    \end{align*}
\end{proof}

\begin{lemma} \label{lem:canonical-product-estimate}
    Let $M \subseteq \N$ and $(z_m)_{m \in M}$ be a family in $\C^\times$ without accumulation points. If for given $p \in \N_0$ it holds that
    $$ \sum_{m \in M} \frac{1}{\vert z_m \vert^{p+1}} < \infty, $$
    then the entire function
    $$ \Pi(z) \coloneqq \prod_{m \in M} W \left( \frac{z}{z_m}; p \right) $$
    satisfies
    \begin{equation}
        \log \vert \Pi(z) \vert < k_p r^p \left( \int_0^r \frac{n_\Pi(t)}{t^{p+1}} \diff t + r \int_r^\infty \frac{n_\Pi(t)}{t^{p+2}} \diff t \right),
    \end{equation}
    for all $z \in \C^\times$ and $r = \vert z \vert$, where
    $$ k_0 \coloneqq 1, \quad \textrm{and} \quad k_p \coloneqq 3e(p+1)(2 + \log p), \quad p \in \N. $$
\end{lemma}

\begin{proof}
    Assuming $p > 0$, combining \Cref{lem:elementary-factor-estimate,lem:zeros-measure} gives
    \begin{align*}
        \log \vert \Pi(z) \vert &= \prod_{m \in M} W \left( \frac{z}{z_m}; p \right) = \sum_{m \in M} \log \left\vert W \left( \frac{z}{z_m}; p \right) \right\vert < \\
        &< A_p \sum_{m \in M} \frac{\left\vert \frac{z}{z_m} \right\vert^{p+1}}{1 + \left\vert \frac{z}{z_m} \right\vert} = A_p r^{p+1} \sum_{m \in M} \frac{1}{\vert z_m \vert^p (\vert z_m \vert + r)} = \\
        &= A_p r^{p+1} \int_{(0, \infty)} \frac{1}{t^p (t+r)} \diff \omega_\Pi(t),
    \end{align*}
    \Cref{lem:zeros-measure} also yields, for $s > 0$,
    \begin{equation*}
        \int_{(0, s)} \frac{1}{t^p (t+r)} \diff \omega_\Pi(t) = \frac{n_\Pi(s)}{s^p (s + r)} + \int_{(0, s)} n_\Pi(t) \frac{(p+1)t + pr}{t^{p+1} (t+r)^2} \diff t.
    \end{equation*}
    As seen in the proof of \Cref{prop:zeros-order-equals-zeros-exponent}, the left term on the right-hand side tends to $0$ as $s \to \infty$. Therefore we obtain
    \begin{align*}
        \log \vert \Pi(z) \vert &< A_p r^{p+1} \int_{(0, \infty)} \frac{1}{t^p (t+r)} \diff \omega_\Pi(t) = A_p r^{p+1} \int_0^\infty n_\Pi(t) \frac{(p+1)t + pr}{t^{p+1} (t+r)^2} \diff t < \\
        &< A_p (p+1) r^{p+1} \int_0^\infty \frac{n_\Pi(t)}{t^{p+1}(t+r)} \diff t < \\
        &< A_p (p+1) r^{p+1} \left( \frac{1}{r} \int_0^r \frac{n_\Pi(t)}{t^{p+1}} \diff t + \int_r^\infty \frac{n_\Pi(t)}{t^{p+2}} \diff t \right) = \\
        &= k_p r^{p} \left( \int_0^r \frac{n_\Pi(t)}{t^{p+1}} \diff t + r \int_r^\infty \frac{n_\Pi(t)}{t^{p+2}} \diff t \right).
    \end{align*}
    The case $p=0$ is verified in the same fashion:
    \begin{align*}
        \log \vert \Pi(z) \vert &= \sum_{m \in M} \log \left\vert W \left( \frac{z}{z_m}; 0 \right) \right\vert \leq \sum_{m \in M} \log \left( 1 + \left\vert \frac{z}{z_m} \right\vert \right) \leq \\
        &\leq \sum_{m \in M} \left\vert \frac{z}{z_m} \right\vert = r \sum_{m \in M} \frac{1}{\vert z_m \vert} = r \int_{(0, \infty)} \frac{1}{t} \diff \omega(t) = -r \int_0^\infty -\frac{n_\Pi(t)}{t^2} \diff t < \\
        &< r \left( \int_0^r \frac{n_\Pi(t)}{tr} \diff t + \int_r^\infty \frac{n_\Pi(t)}{t^2} \diff t \right) = \int_0^r \frac{n_\Pi(t)}{t} \diff t + r \int_r^\infty \frac{n_\Pi(t)}{t^2} \diff t. \qedhere
    \end{align*}
\end{proof}

\begin{theorem} \label{thm:exponent-of-convergence-Weierstraß-product}
    Let $\Pi \in H(\C)$ be a canonical product of finite order, then $\lambda_\Pi = \rho_\Pi$.
\end{theorem}

\begin{proof}
    By \Cref{thm:inequality-order-exponent-of-convergence} we already have $\lambda_\Pi \leq \rho_\Pi$, it remains to show that $\rho_\Pi \leq \lambda_\Pi$. We first recall that the convergence exponent always satisfies $\mu \leq \lambda_\Pi \leq \mu + 1$, where $\mu \coloneqq \mu_\Pi$.
    
    Suppose $\lambda_\Pi < \mu + 1$ and let $\lambda_\Pi < \lambda < \mu + 1$. By \Cref{prop:zeros-order-equals-zeros-exponent} we have
    $$ \lambda_\Pi = \nu_\Pi = \limsup_{r \to \infty} \frac{\log n_\Pi(r)}{\log r} < \lambda. $$
    Therefore there exists some $r_0 > 0$ such that for all $r \geq r_0$ we have $n_\Pi(r) \leq r^\lambda$. Choose $\varepsilon > 0$ small enough such that $n_\Pi(\varepsilon) = 0$, then
    \begin{equation*}
        n_\Pi(r) = \frac{n_\Pi(r)}{r^\lambda} r^\lambda \leq \frac{n_\Pi(r_0)}{\varepsilon^\lambda} r^\lambda
    \end{equation*}
    and choosing $C_\lambda > \max \{ 1, n_\Pi(r_0) / \varepsilon^\lambda \}$ yields $n_\Pi(r) \leq C_\lambda r^\lambda$ for all $r > 0$. Applying this inequality to \Cref{lem:canonical-product-estimate} we get, for $r = \vert z \vert$,
    \begin{align*}
        \log \vert \Pi(z) \vert &< k_\mu r^\mu \left( \int_0^r \frac{n_\Pi(t)}{t^{\mu + 1}} \diff t + r \int_r^\infty \frac{n_\Pi(t)}{t^{\mu + 2}} \diff t \right) \leq \\
        &\leq k_\mu r^\mu C_\lambda \left( \int_0^r t^{\lambda - \mu - 1} \diff t + r \int_r^\infty t^{\lambda - \mu - 2} \diff t \right) = \\
        &= k_\mu r^\mu C_\lambda \left( \frac{r^{\lambda - \mu}}{\lambda - \mu} - \frac{r^{\lambda - \mu}}{\lambda - \mu - 1} \right) = k_\mu C_\lambda \left( \frac{1}{\lambda - \mu} - \frac{1}{\lambda - \mu - 1} \right) r^\lambda.
    \end{align*}
    From this it follows $M_\Pi(r) = \O(\exp r^\lambda)$ and thus that $\rho_\Pi \leq \lambda$. Letting $\lambda \searrow \lambda_\Pi$ yields $\rho_\Pi \leq \lambda_\Pi$.

    If we otherwise assume $\lambda_\Pi = \mu + 1$, then by \Cref{lem:zeros-measure}
    $$ \int_{(0, r)} \frac{1}{t^{\mu + 1}} \diff \omega_\Pi(t) = \frac{n_\Pi(r)}{r^{\mu + 1}} + (\mu + 1) \int_0^r \frac{n_\Pi(t)}{t^{\mu + 2}} \diff t, $$
    wherein the leftmost integral converges as $r \to \infty$. Imitating the proof of \Cref{prop:zeros-order-equals-zeros-exponent} we see that the left term on the right-hand side converges to $0$ as $r \to \infty$, thus the rightmost integral converges as well. Let $\varepsilon > 0$. These two convergences imply that there is an $r_0 > 0$ such that for all $r \geq r_0$ we have
    $$ \int_{r_0}^\infty \frac{n_\Pi(t)}{t^{\mu + 2}} \diff t < \varepsilon, \quad \textrm{and} \quad \frac{n_\Pi(r)}{r^{\mu + 1}} < \varepsilon. $$
    Inserting these results into \Cref{lem:canonical-product-estimate} yields, for $r = \vert z \vert \geq r_0$,
    \begin{align*}
        \log \vert \Pi(z) \vert &< k_\mu r^\mu \left( \int_0^r \frac{n_\Pi(t)}{t^{\mu + 1}} \diff t + r \int_r^\infty \frac{n_\Pi(t)}{t^{\mu + 2}} \diff t \right) = \\
        &= k_\mu r^\mu \left( \int_0^{r_1} \frac{n_\Pi(t)}{t^{\mu + 1}} \diff t + \int_{r_1}^r \frac{n_\Pi(t)}{t^{\mu + 1}} \diff t + r \int_r^\infty \frac{n_\Pi(t)}{t^{\mu + 2}} \diff t \right) \leq \\
        &\leq k_\mu r^\mu \left(C + (r - r_1) \varepsilon + r \varepsilon \right) \leq k_\mu (C r^\mu + 2 \varepsilon r^{\mu + 1}).
    \end{align*}
    The right-most term asymptotically dominates, therefore $M_\Pi(r) = \O(\exp(\varepsilon r^{\mu + 1}))$ and thus $\Pi$ is at most of order $\mu + 1 = \lambda_\Pi$ and, since $\varepsilon$ was arbitrary, minimal type.
\end{proof}

Just as the order of a canonical product is determined by the order of growth of its zeros, its type is (to some extent) determined by the upper density of its zeros. A peculiarity arises when considering canonical products of integral order, in which case the type may also depend on the argument of the zeros \cite{levin-distribution-of-zeros}. However, for non-integral order we can obtain the following result:

\begin{theorem} \label{thm:canonical-product-type-density}
    Let $\Pi \in H(\C)$ be a canonical product of finite order. If $\lambda_\Pi$ is not an integer, then $\Pi$ is of maximal, minimal or normal type according to whether $\Delta_\Pi$ is equal to infinity, zero, or a positive, real number.
\end{theorem}

\begin{proof}
    by \Cref{lem:zeros-bounded-by-modulus} there is a constant $K > 0$ such that for all sufficiently large $r$ we have $n_\Pi(r) \leq K \log M_\Pi(er)$ and by \Cref{prop:zeros-order-equals-zeros-exponent} and \Cref{thm:exponent-of-convergence-Weierstraß-product} we have $\nu_\Pi = \lambda_\Pi = \rho_\Pi$. Therefore we have, for large $r$,
    \begin{equation} \label{eq:density-smaller-constant-type}
        \frac{n_\Pi(r)}{r^{\nu_\Pi}} = \frac{n_\Pi(r)}{r^{\rho_\Pi}} \leq K e^{\rho_\Pi} \frac{\log M_\Pi(er)}{(e r)^{\rho_\Pi}}.
    \end{equation}
    We now consider two cases. If $\Delta_\Pi$ is infinite then taking limits superior as $r \to \infty$ in the above shows than $\Pi$ must be of maximal type.

    If $\Delta_\Pi$ is finite then let $\Delta > \Delta_\Pi$. By definition of the limit superior there is an $r_0 > 0$ such that, for all $r \geq r_0$, we have
    \begin{equation*}
        n_\Pi(r) \leq \Delta r^{\nu_\Pi} = \Delta r^{\rho_\Pi}. 
    \end{equation*}
    As demonstrated in the proof of \Cref{thm:exponent-of-convergence-Weierstraß-product}, we can find a constant $C_{\lambda_\Pi} > 0$ such that for all $r > 0$ we have
    \begin{equation*}
        n_\Pi(r) \leq C_{\lambda_\Pi} \Delta r^{\rho_\Pi}
    \end{equation*}
    The genus $\mu$ of the canonical product satisfies $\mu < \lambda_\Pi < \mu + 1$, since $\lambda_\Pi$ is not an integer. Therefore, by \Cref{lem:canonical-product-estimate}, for all $r > 0$ and $\vert z \vert = r$ we have
    \begin{align*}
        \log \vert \Pi(z) \vert &< k_\mu r^\mu \left( \int_0^r \frac{n_\Pi(t)}{t^{\mu + 1}} \diff t + r \int_r^\infty \frac{n_\Pi(t)}{t^{\mu + 2}} \diff t \right) \leq \\
        &\leq k_\mu r^\mu C_{\lambda_\Pi} \Delta \left( \int_0^r t^{\rho_\Pi - \mu - 1} \diff t + r \int_r^\infty t^{\rho_\Pi - \mu - 2} \diff t \right) = \\
        &= k_\mu r^\mu C_{\lambda_\Pi} \Delta \left( \frac{r^{\rho_\Pi - \mu}}{\rho_\Pi - \mu} - \frac{r^{\rho_\Pi - \mu}}{\rho_\Pi - \mu - 1} \right) = k_\mu C_{\lambda_\Pi} \Delta \left( \frac{1}{\rho_\Pi - \mu} - \frac{1}{\rho_\Pi - \mu - 1} \right) r^{\rho_\Pi}.
    \end{align*}
    Dividing by $r^{\rho_\Pi}$, taking limits superior as $r \to \infty$ and letting $\Delta \searrow \Delta_\Pi$ therefore yields $\tau_\Pi \leq C \Delta_\Pi$ for a constant $C > 0$. Combining with \eqref{eq:density-smaller-constant-type}, we get
    \begin{equation*}
        C_1 \tau_\Pi \leq \Delta_\Pi \leq C_2 \tau_\Pi
    \end{equation*}
    for constants $C_1, C_2 > 0$. This shows that, if $\Delta_\Pi \in (0, \infty)$, then $\Pi$ is of normal type, and if $\Delta_\Pi = 0$, then $\Pi$ is of minimal type.
\end{proof}
\section{Hadamard's Theorem} \label{sec:hadamards-theorem}

Our main goal in this section will be to refine the factorization given by the Weierstraß Factorization Theorem for entire functions of finite order. Such a refinement is given by Hadamard's Theorem, the proof of which relies on the following lemma, which can be considered as a version of the maximum modulus principle applied to the real part of a holomorphic function.

The contents of this section closely follow the results found in Segal \cite{segal-complex-analysis}.

\begin{lemma}[Borel--Carathéodory] \label{lem:borel-caratheodory}
    Let $G \subseteq \C$ be a domain, $R > 0$ and suppose $\cl{B_R(0)} \subseteq G$ and $f \in H(G)$. Define $ A_f(r) \coloneqq \max_{\vert z \vert = r} \Re f(z) $, then, for $0 < r < R$,
    \begin{equation} \label{eq:borel-caratheodory-1}
        M_f(r) \leq \frac{2r}{R - r} A_f(R) + \frac{R + r}{R - r} \vert f(0) \vert
    \end{equation}
    and, if additionally $A_f(R) \geq 0$, then for any $n \in \N$
    \begin{equation} \label{eq:borel-caratheodory-2}
        M_{f^{(n)}}(r) \leq \frac{2^{n+2} n! R}{(R - r)^{n+1}} (A_f(R) + \vert f(0) \vert).
    \end{equation}
\end{lemma}

\begin{proof}
    If $f$ is constant, there is nothing to show.

    We assume $f$ being non-constant. First, for $r > 0$ we have
    \begin{align*}
        A_f(r) = \max_{\vert z \vert = r} \Re f(z) = \log \max_{\vert z \vert = r} \exp {\Re f(z)} = \log \max_{\vert z \vert = r} \vert \exp f(z) \vert = \log M_{\exp f}(r)
    \end{align*}
    and since $M_{\exp f}$ is strictly increasing and continuous therefore so is $A_f$.

    We will show (\ref{eq:borel-caratheodory-1}) by considering two cases. First assume $f$ is non-constant and $f(0) = 0$. Then by the above we have $A_f(R) > A_f(0) = 0$. Define
    \begin{equation*}
        \phi(z) \coloneqq \frac{f(z)}{2 A_f(R) - f(z)} \in H(B_R(0)).
    \end{equation*}
    Note that we have $\phi(0) = 0$ and
    \begin{align*}
        \vert \phi(z) \vert^2 &= \phi(z) \overline{\phi(z)} = \frac{\vert f(z) \vert^2}{(2 A_f(R) - f(z))(2 A_f(R) - \overline{f(z)})} = \\
        &= \frac{\vert f(z) \vert^2}{(2 A_f(R) - \Re f(z))^2 + \vert f(z) \vert^2 - (\Re f(z))^2} \leq 1,
    \end{align*}
    since clearly $2 A_f(R) - \Re f(z) \geq \Re f(z)$. Now, since $\phi(zR) \in H(\D)$ and $\vert \phi(zR) \vert \leq 1$ for $z \in \D$, Schwartz' Lemma implies $\vert \phi(zR) \vert \leq \vert z \vert$ for $z \in \D$. Therefore, for $z \in B_R(0)$ we have $\vert \phi (z) \vert \leq \frac{\vert z \vert}{R}$, and for $z \in B_r(0)$ we have $\vert \phi (z) \vert < \frac{r}{R} < 1$. Since
    \begin{equation*}
        f(z) = 2 A_f(R) \frac{\phi(z)}{1 + \phi(z)}
    \end{equation*}
    we obtain
    \begin{align*}
        \vert f(z) \vert &= 2 A_f(R) \left\vert \sum_{n=1}^\infty (-1)^n \phi(z)^n \right\vert \leq 2 A_f(R) \sum_{n=1}^\infty \left( \frac{r}{R} \right)^n = \frac{2 r}{R - r} A_f(R).
    \end{align*}

    Now assume that $f$ is non-constant and $f(0) \neq 0$. Set $g(z) \coloneqq f(z) - f(0)$, then by the above we have, for $w \in B_r(0)$,
    \begin{align*}
        \vert f(w) \vert - \vert f(0) \vert \leq \vert g(w) \vert \leq M_g(r) \leq \frac{2 r}{R - r} A_g(R) \leq \frac{2r}{R - r} A_f(R) - \frac{2r}{R - r} \Re f(0).
    \end{align*}
    Since $-\Re f(0) \leq \vert f(0) \vert$, this implies
    \begin{align*}
        \vert f(w) \vert \leq \frac{2r}{R - r}A_f(R) + \frac{2r}{R - r} \vert f(0) \vert + \vert f(0) \vert \leq \frac{2r}{R - r} A_f(R) + \frac{R + r}{R - r} \vert f(0) \vert,
    \end{align*}
    thus proving (\ref{eq:borel-caratheodory-1}).

    To show (\ref{eq:borel-caratheodory-2}), let $z \in \partial B_r(0)$ and set $s \coloneqq \frac{R - r}{2}$, then by Cauchy's integral formula
    \begin{equation*}
        f^{(n)}(z) = \frac{n!}{2 \pi i} \oint_{\partial B_s(z)} \frac{f(\zeta)}{(\zeta - z)^{n+1}} \diff \zeta.
    \end{equation*}
    Replacing $r$ with $\frac{R + r}{2} < R$ in (\ref{eq:borel-caratheodory-1}) we get
    \begin{align*}
        \vert f^{(n)}(z) \vert &\leq \frac{n!}{2 \pi} 2 \pi \frac{R - r}{2} \left( \frac{2}{R - r} \right)^{n+1} \left( \frac{2 (R + r)/2}{R - (R+r)/2} A_f(R) + \frac{R + (R + r)/2}{R - (R + r)/2} \vert f(0) \vert \right) \leq \\
        &\leq \frac{2^{n+1} n!}{(R - r)^{n+1}} \frac{R - r}{2} \left( 2 \frac{R+r}{R-r} A_f(R) + \frac{3R + r}{R - r} \vert f(0) \vert \right) \leq \\
        &\leq \frac{2^{n+1} n!}{(R - r)^{n+1}} \left( (R+r) A_f(R) + \frac{3R + r}{2} \vert f(0) \vert \right) \leq \frac{2^{n+2} n! R}{(R - r)^{n+1}} (A_f(R) + \vert f(0) \vert).
    \end{align*}
\end{proof}

\begin{theorem}[Hadamard]
    Let $f \in H(\C)$ be not identically zero, of finite order and let $m \in \N_0$ be the order of the zero of $f$ at the origin. Then there exists a polynomial $Q$ with $\deg Q \leq \rho_f$ such that
    \begin{equation}
        f(z) = z^m e^{Q(z)} \Pi(z),
    \end{equation}
    where $\Pi$ denotes the canonical product formed from the zeros of $f$.
\end{theorem}

Hadamard's Theorem can also be formulated more concisely: Let $f \in H(\C)$ be of finite order, then $\mu_f \leq \rho_f$, that is the genus of $f$ does not exceed its order.

\begin{proof}
    We may assume $f(0) \neq 0$. By the Weierstraß Factorization Theorem we have $f(z) = e^{Q(z)} \Pi(z)$ for some $Q \in H(\C)$ and a Weierstraß product $\Pi$. Since
    $$ \mu_\Pi \leq \lambda_\Pi = \lambda_f \leq \rho_f $$
    we may take $\Pi$ to be a canonical product. Set $\nu \coloneqq \lfloor \rho_f \rfloor$, then taking logarithms in the above factorization and differentiating $\nu + 1$ times we get
    \begin{align*}
        \deriv{z}{\nu} \left( \frac{f'(z)}{f(z)} \right)
        &= Q^{(\nu+1)}(z) + \deriv{z}{\nu + 1} \log \prod_{m \in M} W \left( \frac{z}{z_m}; \mu_\Pi \right) = \\
        &= Q^{(\nu+1)}(z) + \deriv{z}{\nu + 1} \sum_{m \in M} \left( \log \left( 1 - \frac{z}{z_m} \right) + \sum_{k=1}^{\mu_\Pi} \frac{z^k}{k} \right) = \\
        &= Q^{(\nu+1)}(z) - \sum_{m \in M} \frac{1}{z_m} \deriv{z}{\nu} \left( 1 - \frac{z}{z_m} \right)^{-1} = \\
        &= Q^{(\nu+1)}(z) - \nu! \sum_{m \in M} \frac{1}{(z_m - z)^{\nu + 1}}.
    \end{align*}
    We now aim to show that $Q^{(\nu + 1)}$ is identically zero. For $R > 0$ we set
    $$ g_R(z) \coloneqq \frac{f(z)}{f(0)} \prod_{\substack{m \in M \\ \vert z_m \vert \leq R}} \left(1 - \frac{z}{z_m} \right)^{-1} \in H(\C). $$
    Note that $g_R(0) = 1$. For $\vert z \vert = 2R$ and $\vert z_m \vert \leq R$ we have $\vert 1 - z / z_m \vert \geq \vert z \vert / \vert z_m \vert - 1 \geq 2R/R - 1 = 1$, therefore
    $$ M_{g_R}(2R) \leq \max_{\vert z \vert = 2R} \left\vert \frac{f(z)}{f(0)} \right\vert = \O(\exp (2R)^\alpha), \quad \textrm{as } R \to \infty $$
    for all $\alpha > \rho_f$. By the maximum modulus theorem this also holds for $\vert z \vert \leq 2R$, and in particular we have
    $$ \log M_{g_R}(R) = \O(R^\alpha). $$
    Note that $g_R$ has no zeros in $\cl{B_R(0)}$, therefore we can define $h_R(z) \coloneqq \log g_R(z)$, choosing the branch of the logarithm for which $h_R(0) = 0$, then $h_R$ is holomorphic on some domain containing $\cl{B_R(0)}$. We have
    $$ A_{h_R}(R) = \max_{\vert z \vert = R} \Re h_R(z) = \max_{\vert z \vert = R} \log \vert g_R(z) \vert = \log M_{g_R}(R) = \O(R^\alpha). $$
    Since by the maximum modulus theorem $\vert g_R(z) \vert \geq \vert g_R(0) \vert = 1$ we have $\Re h_R(z) \geq 0$. Invoking the Borel-Carathéodory lemma we therefore get, for $0 < r < R$ and sufficiently large $R$,
    $$ M_{h_R^{(\nu + 1)}}(R) \leq \frac{2^{\nu + 3} (\nu + 1)! R}{(R - r)^{\nu + 2}} A_{h_R}(R) \leq \frac{K_0 2^{\nu + 3} (\nu + 1)! R^{\alpha + 1}}{(R - r)^{\nu + 2}}, $$
    where $K_0 > 0$ is some constant depending only on $\alpha$. With $r \coloneqq R / 2$ it follows that
    $$ M_{h_R^{(\nu + 1)}}(R) \leq K_1 R^{\alpha - (\nu + 1)}, $$
    where $K_1 > 0$ is some constant depending only on $\alpha$ and $\nu$. Note that we have
    $$ h_R^{(\nu + 1)}(z) = \deriv{z}{\nu} \left( \frac{f'(z)}{f(z)} \right) + \nu! \sum_{\substack{m \in M \\ \vert z_m \vert \leq R}} \frac{1}{(z_m - z)^{\nu + 1}} = Q^{(\nu + 1)}(z) - \nu! \sum_{\substack{m \in M \\ \vert z_m \vert > R}} \frac{1}{(z_m - z)^{\nu + 1}}. $$
    For $\vert z \vert = R / 2$ and $z_m > R$ we have $\vert z_m - z \vert \geq \vert z_m \vert - R / 2 \geq \vert z_m \vert / 2$, therefore
    $$ \vert Q^{(\nu+1)}(z) \vert \leq \nu! \sum_{\substack{m \in M \\ \vert z_m \vert > R}} \frac{1}{\vert z_m - z \vert^{\nu + 1}} + \vert h_R^{(\nu + 1)}(z) \vert \leq K_2 \sum_{\substack{m \in M \\ \vert z_m \vert > R}} \frac{1}{\vert z_m \vert^{\nu + 1}} + K_1 R^{\alpha - (\nu + 1)}, $$
    where $K_2$ is some constant depending only on $\nu$. By the maximum modulus principle this holds for all $\vert z \vert \leq R / 2$. Since $\nu + 1 > \rho_f \geq \lambda_f$ both terms on the right side tend to zero as $R \to \infty$, if $\rho_f < \alpha < \nu + 1$. Therefore $Q^{(\nu + 1)}$ vanishes identically, thus $Q$ is a polynomial of degree at most $\nu \leq \rho_f$.
\end{proof}

\begin{remark} \label{rem:consequences-hadamard}
    An immediate consequence of Hadamard's Theorem is that entire functions of order $\rho < 1$ are of genus $0$, and are therefore -- like polynomials -- solely determined by their roots and some scaling factor. Indeed, in this case the polynomial $Q$ in the factorization is constant and the canonical product $\Pi$ is of genus $0$ as well. In particular, any non-vanishing function of order zero is constant.

    Furthermore, we can deduce that non-polynomial functions of order zero have no Picard exceptional value: Let $f$ denote such a function and assume towards a contradiction that $f$ assumes some value $a \in \C$ only finitely often. Then, by Hadamard's Theorem, we can write
    $$ f(z) - a = C z^m \Pi(z) $$
    for some constant $C > 0, m \in \N_0$ and a canonical product $\Pi$. Since $a$ is assumed only finitely often, the canonical product $\Pi$ must be finite and therefore $f$ a polynomial, a contradiction.
    
    As shown in \Cref{rem:order-zero} the entire function
    $$ f(z) \coloneqq \sum_{k=0}^\infty \frac{z^k}{(k^2)!} $$
    is an example of a non-polynomial, entire function of order zero and must therefore, by the above reasoning, assume all values in $\C$ infinitely often.
\end{remark}

Hadamard's Theorem has a few consequences for entire functions of finite order, which we will explore further. Combining it with \Cref{thm:exponent-of-convergence-Weierstraß-product} allows us to prove two results regarding functions of finite, non-integer order.

\begin{theorem} \label{thm:finite-non-integer-order-equals-exponent-of-convergence}
    Let $f \in H(\C)$ be of finite, non-integer order. Then $\rho_f = \lambda_f$.
\end{theorem}

\begin{proof}
    By \Cref{thm:inequality-order-exponent-of-convergence} we have $\lambda_f \leq \rho_f$. Invoking Hadamard's Theorem we can write
    $$ f(z) = z^m e^{Q(z)} \Pi(z) $$
    for some $m \in \N_0$, a polynomial $Q$ with $\deg Q \leq \rho_f$ and a canonical product $\Pi$. Since $\rho_f$ is not an integer we have $\deg Q \leq \lfloor \rho_f \rfloor < \rho_f$. By \Cref{prop:order-exponential-polynomial} $e^Q$ has order at most $\deg Q$ and by \Cref{thm:exponent-of-convergence-Weierstraß-product} $\Pi$ has order $\lambda_f$, therefore using \Cref{prop:order-sum-product-estimate} we obtain
    $$ \rho_f \leq \max \{ 0, \deg Q, \lambda_f \} = \lambda_f \leq \rho_f, $$
    since $\rho_f \leq \max \{ \deg Q, \lambda_f \} = \deg Q < \rho_f$ would be a contradiction, and we get $\rho_f = \lambda_f$.
\end{proof}

\begin{theorem} \label{thm:finite-non-integer-order-infinite-zeros}
    Let $f \in H(\C)$ be of finite, non-integer order. Then $f$ has infinitely many zeros.
\end{theorem}

\begin{proof}
    By \Cref{thm:finite-non-integer-order-equals-exponent-of-convergence} we have $\rho_f = \lambda_f$. Since $\rho_f$ is not an integer it follows that $\lambda_f > 0$, which implies that $f$ has infinitely many zeros.
\end{proof}

\begin{theorem}[Borel] \label{thm:existence-borel-exceptional-values}
    Let $f \in H(\C)$ be of integer order $\rho_f > 0$. Then for any $a \in \C$ we have $\lambda_f^{(a)} = \rho_f$, except possibly for one value of $a$.
\end{theorem}

\begin{proof}
    Note that for any $w \in \C$ we have
    $$ \lambda_f^{(w)} = \lambda_{f-w} \leq \rho_{f-w} = \rho_f. $$
    Assume towards a contradiction that there exist $a, b \in \C$ such that $\lambda_f^{(a)} < \rho_f$ and $\lambda_f^{(b)} < \rho_f$. By Hadamard's Theorem we can write
    $$ \alpha(z) \coloneqq f(z) - a = z^{m_1} e^{Q_1(z)} \Pi_1(z), \quad \beta(z) \coloneqq f(z) - b = z^{m_2} e^{Q_2(z)} \Pi_2(z), $$
    where $m_1, m_2 \in \N_0$, $Q_1, Q_2$ are polynomials of degree at most $\rho_f$ and $\Pi_1, \Pi_2$ denote appropriate canonical products. We have
    $$ \deg Q_1 \leq \rho_f = \rho_\alpha \leq \max \{ 0, \deg Q_1, \rho_{\Pi_1} \}, $$
    and since by \Cref{thm:exponent-of-convergence-Weierstraß-product} $\rho_{\Pi_1} = \lambda_f^{(a)} < \rho_f$, it follows that $\deg Q_1 = \rho_f$. The analogous argument for $Q_2$ yields $\deg Q_2 = \rho_f$. Subtracting $\beta$ from $\alpha$ yields
    \begin{equation} \label{thm:existence-borel-exceptional-values:difference-of-functions}
    \begin{aligned}
                        && b - a &= z^{m_1} e^{Q_1(z)} \Pi_1(z) - z^{m_2} e^{Q_2(z)} \Pi_2(z) \\
        \Leftrightarrow && (b - a) e^{-Q_2(z)} &= z^{m_1} e^{Q_1(z) - Q_2(z)} \Pi_1(z) - z^{m_2} \Pi_2(z) \\
        \Leftrightarrow && (b - a) e^{-Q_2(z)} + z^{m_2} \Pi_2(z) &= z^{m_1} e^{Q_1(z) - Q_2(z)} \Pi_1(z)
    \end{aligned}
    \end{equation}
    Since $\rho_{\Pi_2} = \lambda_f^{(b)} < \rho_f$ and $-Q_2$ is of degree $\rho_f$, the left-hand side is of order $\rho_f$ and thus so is the right-hand side. Now similarly, since $\rho_{\Pi_1} = \lambda_f^{(b)} < \rho_f$ and the right-hand side has order $\rho_f$, we have that $Q_1 - Q_2$ is of degree $\rho_f$.
    
    Differentiating the first equation in \eqref{thm:existence-borel-exceptional-values:difference-of-functions} gives
    \begin{equation*}
    \begin{aligned}
        && 0 &= m_1 z^{m_1 - 1} e^{Q_1(z)} \Pi_1(z) + z^{m_1} Q_1'(z) e^{Q_1(z)} \Pi_1(z) + z^{m_1} e^{Q_1}(z) \Pi_1'(z) \\
        && &\quad\quad - m_2 z^{m_2 - 1} e^{Q_2(z)} \Pi_2(z) - z^{m_2} Q_2'(z) e^{Q_2(z)} \Pi_2(z) - z^{m_2} e^{Q_2(z)} \Pi_2'(z) \\
        \Leftrightarrow && 0 &= e^{Q_1(z)} (m_1 z^{m_1 - 1} \Pi_1(z) + z^{m_1} Q_1'(z) \Pi_1(z) + z^{m_1} \Pi_1'(z)) \\
        && &\quad\quad - e^{Q_2(z)} (m_2 z^{m_2 - 1} \Pi_2(z) + z^{m_2} Q_2'(z) \Pi_2(z) + z^{m_2} \Pi_2'(z))
    \end{aligned}
    \end{equation*}
    By \Cref{prop:order-derivative} we have $\rho_{\Pi_1'} = \rho_{\Pi_1}$ and $\rho_{\Pi_2'} = \rho_{\Pi_2}$, therefore the coefficients of $e^{Q_1}$ and $e^{Q_2}$ have order less than $\rho_f$. By Hadamard's Theorem we therefore get
    \begin{equation*}
        e^{Q_1(z)} z^{m_3} e^{Q_3(z)} \Pi_3(z) = e^{Q_2(z)} z^{m_4} e^{Q_4(z)} \Pi_4(z)
    \end{equation*}
    where $m_3, m_4 \in \N_0, Q_3, Q_4$ are polynomials of degree less than or equal to $\rho_f - 1$ (since $\rho_f$ is an integer) and $P_3, P_4$ denote canonical products. Rewriting the above we get
    \begin{equation*}
        z^{m_3} e^{Q_1(z) + Q_3(z)} \Pi_3(z) = z^{m_4} e^{Q_2(z) + Q_4(z)} \Pi_4(z)
    \end{equation*}
    Since both sides are equal, in particular they must share the same zeros and multiplicities, therefore we have $m_3 = m_4$ and $\Pi_3 = \Pi_4$. But this implies
    \begin{equation*}
    \begin{aligned}
        && Q_1(z) + Q_3(z) &= Q_2(z) + Q_4(z) \\
        \Leftrightarrow && Q_2(z) - Q_1(z) &= Q_3(z) - Q_4(z)
    \end{aligned}
    \end{equation*}
    and by the above the left side is of degree $\rho_f$, whereas the right side is of degree less than $\rho_f$, a contradiction.
\end{proof}

In the context of Borel's Theorem, we refer to such an exceptional value as a \emph{Borel exceptional value}. The theorem thus shows that entire functions of integer order have at most one Borel exceptional value.

Borel's Theorem together with \Cref{thm:finite-non-integer-order-infinite-zeros} allow us to obtain a stronger version of Picard's Little Theorem for functions of finite, positive order:

\begin{corollary}
    Let $f \in H(\C)$ with $\rho_f \in (0, \infty)$. Then $f$ assumes at most one value only finitely often.
\end{corollary}

\begin{proof}
    If $\rho_f$ is an integer then and $f$ takes on some value $a \in \C$ only on finitely often, then $\lambda_f^{(a)} = 0 < \rho_f$ and by Borel's Theorem there is at most one value of $a$ for which this holds.

    If $\rho_f$ is not an integer, then for any $w \in \C$ the function $f(z) - w$ is also of finite, non-integer order and therefore has infinitely many zeros by \Cref{thm:finite-non-integer-order-infinite-zeros}. Thus $f$ assumes all values infinitely often.
\end{proof}

Hadamard's Theorem can also be used to obtain a generalization of \Cref{thm:canonical-product-type-density} which holds for general entire functions of non-integer order, not just canonical products. This requires the following result, the proof of which relies on finding a lower bound on the modulus of arbitrary entire functions and can be found in Levin \cite{levin-distribution-of-zeros} as Theorem 12(a).

\begin{proposition} \label{prop:order-type-product-equality}
    Let $f, g \in H(\C)$ be of finite order. If $\rho_f < \rho_g$, then it holds that $ \rho_{fg} = \rho_g $ and\footnote{This holds in the sense that if $g$ is of maximal or minimal type, then $fg$ is of maximal or minimal type respectively, and, if $g$ is of normal type, then $\tau_{fg} = \tau_g$.} $\tau_{fg} = \tau_g$.
\end{proposition}

\begin{theorem} \label{thm:entire-function-type-density}
    Let $f \in H(\C)$ be of finite, non-integer order. Then $f$ is of maximal, minimal or normal type according to whether $\Delta_f$ is equal to infinity, zero, or a positive, real number.
\end{theorem}

\begin{proof}
    By Hadamard's Theorem we can write
    $$ f(z) = z^m e^{Q(z)} \Pi(z). $$
    The first factor is of order zero, the second is at most of order $\lfloor \rho_f \rfloor < \rho_f$ and the last is of order $\rho_f$. Therefore, by \Cref{prop:order-type-product-equality} we have  $\tau_f = \tau_\Pi$. Since the order of $\Pi$ is not an integer and clearly $\Delta_f = \Delta_\Pi$, the assertion now follows from \Cref{thm:canonical-product-type-density}.
\end{proof}

In general, if $f \in H(\C)$ is of finite order with representation
$$ f(z) = e^{Q(z)} \Pi(z), $$
where $Q$ is a polynomial and $\Pi$ is a canonical product, the order and type of $f$ may be determined by either of the factors. However, \Cref{thm:finite-non-integer-order-equals-exponent-of-convergence,thm:entire-function-type-density} show that if $\rho_f$ is not an integer, then the growth of $f$ is determined solely by the canonical product $\Pi$, and therefore by the zeros of $f$.