\section{Hadamard's Theorem} \label{sec:hadamards-theorem}

Our main goal in this section will be to refine the factorization given by the Weierstraß Factorization Theorem for entire functions of finite order. Such a refinement is given by Hadamard's Theorem, the proof of which relies on the following lemma, which can be considered as a version of the maximum modulus principle applied to the real part of a holomorphic function.

The contents of this section closely follow the results found in Segal \cite{segal-complex-analysis}.

\begin{lemma}[Borel--Carathéodory] \label{lem:borel-caratheodory}
    Let $G \subseteq \C$ be a domain, $R > 0$ and suppose $\cl{B_R(0)} \subseteq G$ and $f \in H(G)$. Define $ A_f(r) \coloneqq \max_{\vert z \vert = r} \Re f(z) $, then, for $0 < r < R$,
    \begin{equation} \label{eq:borel-caratheodory-1}
        M_f(r) \leq \frac{2r}{R - r} A_f(R) + \frac{R + r}{R - r} \vert f(0) \vert
    \end{equation}
    and, if additionally $A_f(R) \geq 0$, then for any $n \in \N$
    \begin{equation} \label{eq:borel-caratheodory-2}
        M_{f^{(n)}}(r) \leq \frac{2^{n+2} n! R}{(R - r)^{n+1}} (A_f(R) + \vert f(0) \vert).
    \end{equation}
\end{lemma}

\begin{proof}
    If $f$ is constant, there is nothing to show.

    We assume $f$ being non-constant. First, for $r > 0$ we have
    \begin{align*}
        A_f(r) = \max_{\vert z \vert = r} \Re f(z) = \log \max_{\vert z \vert = r} \exp {\Re f(z)} = \log \max_{\vert z \vert = r} \vert \exp f(z) \vert = \log M_{\exp f}(r)
    \end{align*}
    and since $M_{\exp f}$ is strictly increasing and continuous therefore so is $A_f$.

    We will show (\ref{eq:borel-caratheodory-1}) by considering two cases. First assume $f$ is non-constant and $f(0) = 0$. Then by the above we have $A_f(R) > A_f(0) = 0$. Define
    \begin{equation*}
        \phi(z) \coloneqq \frac{f(z)}{2 A_f(R) - f(z)} \in H(B_R(0)).
    \end{equation*}
    Note that we have $\phi(0) = 0$ and
    \begin{align*}
        \vert \phi(z) \vert^2 &= \phi(z) \overline{\phi(z)} = \frac{\vert f(z) \vert^2}{(2 A_f(R) - f(z))(2 A_f(R) - \overline{f(z)})} = \\
        &= \frac{\vert f(z) \vert^2}{(2 A_f(R) - \Re f(z))^2 + \vert f(z) \vert^2 - (\Re f(z))^2} \leq 1,
    \end{align*}
    since clearly $2 A_f(R) - \Re f(z) \geq \Re f(z)$. Now, since $\phi(zR) \in H(\D)$ and $\vert \phi(zR) \vert \leq 1$ for $z \in \D$, Schwartz' Lemma implies $\vert \phi(zR) \vert \leq \vert z \vert$ for $z \in \D$. Therefore, for $z \in B_R(0)$ we have $\vert \phi (z) \vert \leq \frac{\vert z \vert}{R}$, and for $z \in B_r(0)$ we have $\vert \phi (z) \vert < \frac{r}{R} < 1$. Since
    \begin{equation*}
        f(z) = 2 A_f(R) \frac{\phi(z)}{1 + \phi(z)}
    \end{equation*}
    we obtain
    \begin{align*}
        \vert f(z) \vert &= 2 A_f(R) \left\vert \sum_{n=1}^\infty (-1)^n \phi(z)^n \right\vert \leq 2 A_f(R) \sum_{n=1}^\infty \left( \frac{r}{R} \right)^n = \frac{2 r}{R - r} A_f(R).
    \end{align*}

    Now assume that $f$ is non-constant and $f(0) \neq 0$. Set $g(z) \coloneqq f(z) - f(0)$, then by the above we have, for $w \in B_r(0)$,
    \begin{align*}
        \vert f(w) \vert - \vert f(0) \vert \leq \vert g(w) \vert \leq M_g(r) \leq \frac{2 r}{R - r} A_g(R) \leq \frac{2r}{R - r} A_f(R) - \frac{2r}{R - r} \Re f(0).
    \end{align*}
    Since $-\Re f(0) \leq \vert f(0) \vert$, this implies
    \begin{align*}
        \vert f(w) \vert \leq \frac{2r}{R - r}A_f(R) + \frac{2r}{R - r} \vert f(0) \vert + \vert f(0) \vert \leq \frac{2r}{R - r} A_f(R) + \frac{R + r}{R - r} \vert f(0) \vert,
    \end{align*}
    thus proving (\ref{eq:borel-caratheodory-1}).

    To show (\ref{eq:borel-caratheodory-2}), let $z \in \partial B_r(0)$ and set $s \coloneqq \frac{R - r}{2}$, then by Cauchy's integral formula
    \begin{equation*}
        f^{(n)}(z) = \frac{n!}{2 \pi i} \oint_{\partial B_s(z)} \frac{f(\zeta)}{(\zeta - z)^{n+1}} \diff \zeta.
    \end{equation*}
    Replacing $r$ with $\frac{R + r}{2} < R$ in (\ref{eq:borel-caratheodory-1}) we get
    \begin{align*}
        \vert f^{(n)}(z) \vert &\leq \frac{n!}{2 \pi} 2 \pi \frac{R - r}{2} \left( \frac{2}{R - r} \right)^{n+1} \left( \frac{2 (R + r)/2}{R - (R+r)/2} A_f(R) + \frac{R + (R + r)/2}{R - (R + r)/2} \vert f(0) \vert \right) \leq \\
        &\leq \frac{2^{n+1} n!}{(R - r)^{n+1}} \frac{R - r}{2} \left( 2 \frac{R+r}{R-r} A_f(R) + \frac{3R + r}{R - r} \vert f(0) \vert \right) \leq \\
        &\leq \frac{2^{n+1} n!}{(R - r)^{n+1}} \left( (R+r) A_f(R) + \frac{3R + r}{2} \vert f(0) \vert \right) \leq \frac{2^{n+2} n! R}{(R - r)^{n+1}} (A_f(R) + \vert f(0) \vert).
    \end{align*}
\end{proof}

\begin{theorem}[Hadamard]
    Let $f \in H(\C)$ be not identically zero, of finite order and let $m \in \N_0$ be the order of the zero of $f$ at the origin. Then there exists a polynomial $Q$ with $\deg Q \leq \rho_f$ such that
    \begin{equation}
        f(z) = z^m e^{Q(z)} \Pi(z),
    \end{equation}
    where $\Pi$ denotes the canonical product formed from the zeros of $f$.
\end{theorem}

Hadamard's Theorem can also be formulated more concisely: Let $f \in H(\C)$ be of finite order, then $\mu_f \leq \rho_f$, that is the genus of $f$ does not exceed its order.

\begin{proof}
    We may assume $f(0) \neq 0$. By the Weierstraß Factorization Theorem we have $f(z) = e^{Q(z)} \Pi(z)$ for some $Q \in H(\C)$ and a Weierstraß product $\Pi$. Since
    $$ \mu_\Pi \leq \lambda_\Pi = \lambda_f \leq \rho_f $$
    we may take $\Pi$ to be a canonical product. Set $\nu \coloneqq \lfloor \rho_f \rfloor$, then taking logarithms in the above factorization and differentiating $\nu + 1$ times we get
    \begin{align*}
        \deriv{z}{\nu} \left( \frac{f'(z)}{f(z)} \right)
        &= Q^{(\nu+1)}(z) + \deriv{z}{\nu + 1} \log \prod_{m \in M} W \left( \frac{z}{z_m}; \mu_\Pi \right) = \\
        &= Q^{(\nu+1)}(z) + \deriv{z}{\nu + 1} \sum_{m \in M} \left( \log \left( 1 - \frac{z}{z_m} \right) + \sum_{k=1}^{\mu_\Pi} \frac{z^k}{k} \right) = \\
        &= Q^{(\nu+1)}(z) - \sum_{m \in M} \frac{1}{z_m} \deriv{z}{\nu} \left( 1 - \frac{z}{z_m} \right)^{-1} = \\
        &= Q^{(\nu+1)}(z) - \nu! \sum_{m \in M} \frac{1}{(z_m - z)^{\nu + 1}}.
    \end{align*}
    We now aim to show that $Q^{(\nu + 1)}$ is identically zero. For $R > 0$ we set
    $$ g_R(z) \coloneqq \frac{f(z)}{f(0)} \prod_{\substack{m \in M \\ \vert z_m \vert \leq R}} \left(1 - \frac{z}{z_m} \right)^{-1} \in H(\C). $$
    Note that $g_R(0) = 1$. For $\vert z \vert = 2R$ and $\vert z_m \vert \leq R$ we have $\vert 1 - z / z_m \vert \geq \vert z \vert / \vert z_m \vert - 1 \geq 2R/R - 1 = 1$, therefore
    $$ M_{g_R}(2R) \leq \max_{\vert z \vert = 2R} \left\vert \frac{f(z)}{f(0)} \right\vert = \O(\exp (2R)^\alpha), \quad \textrm{as } R \to \infty $$
    for all $\alpha > \rho_f$. By the maximum modulus theorem this also holds for $\vert z \vert \leq 2R$, and in particular we have
    $$ \log M_{g_R}(R) = \O(R^\alpha). $$
    Note that $g_R$ has no zeros in $\cl{B_R(0)}$, therefore we can define $h_R(z) \coloneqq \log g_R(z)$, choosing the branch of the logarithm for which $h_R(0) = 0$, then $h_R$ is holomorphic on some domain containing $\cl{B_R(0)}$. We have
    $$ A_{h_R}(R) = \max_{\vert z \vert = R} \Re h_R(z) = \max_{\vert z \vert = R} \log \vert g_R(z) \vert = \log M_{g_R}(R) = \O(R^\alpha). $$
    Since by the maximum modulus theorem $\vert g_R(z) \vert \geq \vert g_R(0) \vert = 1$ we have $\Re h_R(z) \geq 0$. Invoking the Borel-Carathéodory lemma we therefore get, for $0 < r < R$ and sufficiently large $R$,
    $$ M_{h_R^{(\nu + 1)}}(R) \leq \frac{2^{\nu + 3} (\nu + 1)! R}{(R - r)^{\nu + 2}} A_{h_R}(R) \leq \frac{K_0 2^{\nu + 3} (\nu + 1)! R^{\alpha + 1}}{(R - r)^{\nu + 2}}, $$
    where $K_0 > 0$ is some constant depending only on $\alpha$. With $r \coloneqq R / 2$ it follows that
    $$ M_{h_R^{(\nu + 1)}}(R) \leq K_1 R^{\alpha - (\nu + 1)}, $$
    where $K_1 > 0$ is some constant depending only on $\alpha$ and $\nu$. Note that we have
    $$ h_R^{(\nu + 1)}(z) = \deriv{z}{\nu} \left( \frac{f'(z)}{f(z)} \right) + \nu! \sum_{\substack{m \in M \\ \vert z_m \vert \leq R}} \frac{1}{(z_m - z)^{\nu + 1}} = Q^{(\nu + 1)}(z) - \nu! \sum_{\substack{m \in M \\ \vert z_m \vert > R}} \frac{1}{(z_m - z)^{\nu + 1}}. $$
    For $\vert z \vert = R / 2$ and $z_m > R$ we have $\vert z_m - z \vert \geq \vert z_m \vert - R / 2 \geq \vert z_m \vert / 2$, therefore
    $$ \vert Q^{(\nu+1)}(z) \vert \leq \nu! \sum_{\substack{m \in M \\ \vert z_m \vert > R}} \frac{1}{\vert z_m - z \vert^{\nu + 1}} + \vert h_R^{(\nu + 1)}(z) \vert \leq K_2 \sum_{\substack{m \in M \\ \vert z_m \vert > R}} \frac{1}{\vert z_m \vert^{\nu + 1}} + K_1 R^{\alpha - (\nu + 1)}, $$
    where $K_2$ is some constant depending only on $\nu$. By the maximum modulus principle this holds for all $\vert z \vert \leq R / 2$. Since $\nu + 1 > \rho_f \geq \lambda_f$ both terms on the right side tend to zero as $R \to \infty$, if $\rho_f < \alpha < \nu + 1$. Therefore $Q^{(\nu + 1)}$ vanishes identically, thus $Q$ is a polynomial of degree at most $\nu \leq \rho_f$.
\end{proof}

\begin{remark} \label{rem:consequences-hadamard}
    An immediate consequence of Hadamard's Theorem is that entire functions of order $\rho < 1$ are of genus $0$, and are therefore -- like polynomials -- solely determined by their roots and some scaling factor. Indeed, in this case the polynomial $Q$ in the factorization is constant and the canonical product $\Pi$ is of genus $0$ as well. In particular, any non-vanishing function of order zero is constant.

    Furthermore, we can deduce that non-polynomial functions of order zero have no Picard exceptional value: Let $f$ denote such a function and assume towards a contradiction that $f$ assumes some value $a \in \C$ only finitely often. Then, by Hadamard's Theorem, we can write
    $$ f(z) - a = C z^m \Pi(z) $$
    for some constant $C > 0, m \in \N_0$ and a canonical product $\Pi$. Since $a$ is assumed only finitely often, the canonical product $\Pi$ must be finite and therefore $f$ a polynomial, a contradiction.
    
    As shown in \Cref{rem:order-zero} the entire function
    $$ f(z) \coloneqq \sum_{k=0}^\infty \frac{z^k}{(k^2)!} $$
    is an example of a non-polynomial, entire function of order zero and must therefore, by the above reasoning, assume all values in $\C$ infinitely often.
\end{remark}

Hadamard's Theorem has a few consequences for entire functions of finite order, which we will explore further. Combining it with \Cref{thm:exponent-of-convergence-Weierstraß-product} allows us to prove two results regarding functions of finite, non-integer order.

\begin{theorem} \label{thm:finite-non-integer-order-equals-exponent-of-convergence}
    Let $f \in H(\C)$ be of finite, non-integer order. Then $\rho_f = \lambda_f$.
\end{theorem}

\begin{proof}
    By \Cref{thm:inequality-order-exponent-of-convergence} we have $\lambda_f \leq \rho_f$. Invoking Hadamard's Theorem we can write
    $$ f(z) = z^m e^{Q(z)} \Pi(z) $$
    for some $m \in \N_0$, a polynomial $Q$ with $\deg Q \leq \rho_f$ and a canonical product $\Pi$. Since $\rho_f$ is not an integer we have $\deg Q \leq \lfloor \rho_f \rfloor < \rho_f$. By \Cref{prop:order-exponential-polynomial} $e^Q$ has order at most $\deg Q$ and by \Cref{thm:exponent-of-convergence-Weierstraß-product} $\Pi$ has order $\lambda_f$, therefore using \Cref{prop:order-sum-product-estimate} we obtain
    $$ \rho_f \leq \max \{ 0, \deg Q, \lambda_f \} = \lambda_f \leq \rho_f, $$
    since $\rho_f \leq \max \{ \deg Q, \lambda_f \} = \deg Q < \rho_f$ would be a contradiction, and we get $\rho_f = \lambda_f$.
\end{proof}

\begin{theorem} \label{thm:finite-non-integer-order-infinite-zeros}
    Let $f \in H(\C)$ be of finite, non-integer order. Then $f$ has infinitely many zeros.
\end{theorem}

\begin{proof}
    By \Cref{thm:finite-non-integer-order-equals-exponent-of-convergence} we have $\rho_f = \lambda_f$. Since $\rho_f$ is not an integer it follows that $\lambda_f > 0$, which implies that $f$ has infinitely many zeros.
\end{proof}

\begin{theorem}[Borel] \label{thm:existence-borel-exceptional-values}
    Let $f \in H(\C)$ be of integer order $\rho_f > 0$. Then for any $a \in \C$ we have $\lambda_f^{(a)} = \rho_f$, except possibly for one value of $a$.
\end{theorem}

\begin{proof}
    Note that for any $w \in \C$ we have
    $$ \lambda_f^{(w)} = \lambda_{f-w} \leq \rho_{f-w} = \rho_f. $$
    Assume towards a contradiction that there exist $a, b \in \C$ such that $\lambda_f^{(a)} < \rho_f$ and $\lambda_f^{(b)} < \rho_f$. By Hadamard's Theorem we can write
    $$ \alpha(z) \coloneqq f(z) - a = z^{m_1} e^{Q_1(z)} \Pi_1(z), \quad \beta(z) \coloneqq f(z) - b = z^{m_2} e^{Q_2(z)} \Pi_2(z), $$
    where $m_1, m_2 \in \N_0$, $Q_1, Q_2$ are polynomials of degree at most $\rho_f$ and $\Pi_1, \Pi_2$ denote appropriate canonical products. We have
    $$ \deg Q_1 \leq \rho_f = \rho_\alpha \leq \max \{ 0, \deg Q_1, \rho_{\Pi_1} \}, $$
    and since by \Cref{thm:exponent-of-convergence-Weierstraß-product} $\rho_{\Pi_1} = \lambda_f^{(a)} < \rho_f$, it follows that $\deg Q_1 = \rho_f$. The analogous argument for $Q_2$ yields $\deg Q_2 = \rho_f$. Subtracting $\beta$ from $\alpha$ yields
    \begin{equation} \label{thm:existence-borel-exceptional-values:difference-of-functions}
    \begin{aligned}
                        && b - a &= z^{m_1} e^{Q_1(z)} \Pi_1(z) - z^{m_2} e^{Q_2(z)} \Pi_2(z) \\
        \Leftrightarrow && (b - a) e^{-Q_2(z)} &= z^{m_1} e^{Q_1(z) - Q_2(z)} \Pi_1(z) - z^{m_2} \Pi_2(z) \\
        \Leftrightarrow && (b - a) e^{-Q_2(z)} + z^{m_2} \Pi_2(z) &= z^{m_1} e^{Q_1(z) - Q_2(z)} \Pi_1(z)
    \end{aligned}
    \end{equation}
    Since $\rho_{\Pi_2} = \lambda_f^{(b)} < \rho_f$ and $-Q_2$ is of degree $\rho_f$, the left-hand side is of order $\rho_f$ and thus so is the right-hand side. Now similarly, since $\rho_{\Pi_1} = \lambda_f^{(b)} < \rho_f$ and the right-hand side has order $\rho_f$, we have that $Q_1 - Q_2$ is of degree $\rho_f$.
    
    Differentiating the first equation in \eqref{thm:existence-borel-exceptional-values:difference-of-functions} gives
    \begin{equation*}
    \begin{aligned}
        && 0 &= m_1 z^{m_1 - 1} e^{Q_1(z)} \Pi_1(z) + z^{m_1} Q_1'(z) e^{Q_1(z)} \Pi_1(z) + z^{m_1} e^{Q_1}(z) \Pi_1'(z) \\
        && &\quad\quad - m_2 z^{m_2 - 1} e^{Q_2(z)} \Pi_2(z) - z^{m_2} Q_2'(z) e^{Q_2(z)} \Pi_2(z) - z^{m_2} e^{Q_2(z)} \Pi_2'(z) \\
        \Leftrightarrow && 0 &= e^{Q_1(z)} (m_1 z^{m_1 - 1} \Pi_1(z) + z^{m_1} Q_1'(z) \Pi_1(z) + z^{m_1} \Pi_1'(z)) \\
        && &\quad\quad - e^{Q_2(z)} (m_2 z^{m_2 - 1} \Pi_2(z) + z^{m_2} Q_2'(z) \Pi_2(z) + z^{m_2} \Pi_2'(z))
    \end{aligned}
    \end{equation*}
    By \Cref{prop:order-derivative} we have $\rho_{\Pi_1'} = \rho_{\Pi_1}$ and $\rho_{\Pi_2'} = \rho_{\Pi_2}$, therefore the coefficients of $e^{Q_1}$ and $e^{Q_2}$ have order less than $\rho_f$. By Hadamard's Theorem we therefore get
    \begin{equation*}
        e^{Q_1(z)} z^{m_3} e^{Q_3(z)} \Pi_3(z) = e^{Q_2(z)} z^{m_4} e^{Q_4(z)} \Pi_4(z)
    \end{equation*}
    where $m_3, m_4 \in \N_0, Q_3, Q_4$ are polynomials of degree less than or equal to $\rho_f - 1$ (since $\rho_f$ is an integer) and $P_3, P_4$ denote canonical products. Rewriting the above we get
    \begin{equation*}
        z^{m_3} e^{Q_1(z) + Q_3(z)} \Pi_3(z) = z^{m_4} e^{Q_2(z) + Q_4(z)} \Pi_4(z)
    \end{equation*}
    Since both sides are equal, in particular they must share the same zeros and multiplicities, therefore we have $m_3 = m_4$ and $\Pi_3 = \Pi_4$. But this implies
    \begin{equation*}
    \begin{aligned}
        && Q_1(z) + Q_3(z) &= Q_2(z) + Q_4(z) \\
        \Leftrightarrow && Q_2(z) - Q_1(z) &= Q_3(z) - Q_4(z)
    \end{aligned}
    \end{equation*}
    and by the above the left side is of degree $\rho_f$, whereas the right side is of degree less than $\rho_f$, a contradiction.
\end{proof}

In the context of Borel's Theorem, we refer to such an exceptional value as a \emph{Borel exceptional value}. The theorem thus shows that entire functions of integer order have at most one Borel exceptional value.

Borel's Theorem together with \Cref{thm:finite-non-integer-order-infinite-zeros} allow us to obtain a stronger version of Picard's Little Theorem for functions of finite, positive order:

\begin{corollary}
    Let $f \in H(\C)$ with $\rho_f \in (0, \infty)$. Then $f$ assumes at most one value only finitely often.
\end{corollary}

\begin{proof}
    If $\rho_f$ is an integer then and $f$ takes on some value $a \in \C$ only on finitely often, then $\lambda_f^{(a)} = 0 < \rho_f$ and by Borel's Theorem there is at most one value of $a$ for which this holds.

    If $\rho_f$ is not an integer, then for any $w \in \C$ the function $f(z) - w$ is also of finite, non-integer order and therefore has infinitely many zeros by \Cref{thm:finite-non-integer-order-infinite-zeros}. Thus $f$ assumes all values infinitely often.
\end{proof}

Hadamard's Theorem can also be used to obtain a generalization of \Cref{thm:canonical-product-type-density} which holds for entire functions other than canonical products. This requires the following result, the proof of which relies on finding a lower bound on the modulus of arbitrary entire functions and can be found in Levin \cite{levin-distribution-of-zeros} as Theorem 12(a).

\begin{proposition} \label{prop:order-type-product-equality}
    Let $f, g \in H(\C)$ be of finite order. If $\rho_f < \rho_g$, then it holds that $ \rho_{fg} = \rho_g $ and\footnote{This holds in the sense that if $g$ is of maximal or minimal type, then $fg$ is of maximal or minimal type respectively, and, if $g$ is of normal type, then $\tau_{fg} = \tau_g$.} $\tau_{fg} = \tau_g$.
\end{proposition}

\begin{theorem} \label{thm:entire-function-type-density}
    Let $f \in H(\C)$ be of finite, non-integer order. Then $f$ is of maximal, minimal or normal type according to whether $\Delta_f$ is equal to infinity, zero, or a positive, real number.
\end{theorem}

\begin{proof}
    By Hadamard's Theorem we can write
    $$ f(z) = z^m e^{Q(z)} \Pi(z). $$
    The first factor is of order zero, the second is at most of order $\lfloor \rho_f \rfloor < \rho_f$ and the last is of order $\rho_f$. Therefore, by \Cref{prop:order-type-product-equality} we have  $\tau_f = \tau_\Pi$. Since the order of $\Pi$ is not an integer and clearly $\Delta_f = \Delta_\Pi$, the assertion now follows from \Cref{thm:canonical-product-type-density}.
\end{proof}

\todo{Short recap what we achieved through this theorem.}