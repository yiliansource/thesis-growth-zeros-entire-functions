\section{Canonical Products} \label{sec:canonical-products}

By the Fundamental Theorem of Algebra every complex polynomial can be written as a product of linear factors involving only its zeros and some scaling factor. Conversely, for a finite sequence of points $z_1, \hdots, z_n$ the polynomial $p(z) \coloneqq \prod_{k=1}^n (z - z_k)$ vanishes only at said points.

The question arises if one can construct an entire function that vanishes precisely at the points of some infinite sequence $(z_k)_{k \in \N}$ and nowhere else. Clearly the product $\prod_{k=1}^\infty (z - z_k)$ need not be convergent in the entire complex plane. This issue is solved by introducing additional exponential factors, which improve convergence without introducing new zeros. We first study sufficient conditions for convergence of infinite products of numbers and holomorphic functions.

The early parts of the section will closely follow Stein and Shakarchi \cite{stein-shakarchi-princeton}, while the later estimates and results will follow Levin \cite{levin-distribution-of-zeros}.

\begin{lemma} \label{lem:infinite-product-criteria}
    Suppose $G \subseteq \C$ is open and let $(f_k)_{k \in \N}$ be a sequence of non-vanishing functions in $H(G)$. If $\sum_{k=1}^\infty (1 - f_k(z))$ converges absolutely compactly in $G$, then $\prod_{k=1}^\infty f_k(z)$ converges compactly in $G$ to a non-vanishing $f \in H(G)$.
\end{lemma}
    
\begin{proof}
    Let $K \subset G$ be compact, then the absolute, compact convergence of the sum implies that there is a $k_0$ such that $\vert 1 - f_k(z) \vert \leq 1/2$ for $z \in K, k > k_0$. For $\vert w \vert \leq 1/2$ we have
    \begin{equation*}
        \vert \log(1 - w) \vert \leq \sum_{k=1}^\infty \left\vert \frac{1}{k} w^k \right\vert \leq \vert w \vert \sum_{k=0}^\infty \vert w \vert^k = \frac{\vert w \vert}{1 - \vert w \vert} \leq 2 \vert w \vert.
    \end{equation*}
    Therefore
    \begin{equation*}
        \sum_{k=1}^\infty \vert \log f_k(z) \vert = \sum_{k=1}^\infty \vert \log (1 - (1 - f_k(z))) \vert \leq \sum_{k=1}^{k_0 - 1} \vert \log f_k(z) \vert + 2 \sum_{k=k_0}^\infty \vert 1 - f_k(z) \vert
    \end{equation*}
    and since the rightmost sum converges compactly it follows that $\sum_{k=1}^\infty \log f_k(z)$ converges absolutely compactly in $G$. Since the exponential function is uniformly continuous, the sequence
    $$ \exp \left( \sum_{k=k_0}^n \log f_k(z) \right) = \prod_{k=k_0}^n f_k(z), \quad n \in \N $$
    converges absolutely compactly to some $f \in H(G)$. Again by continuity of the exponential function we have
    $$ f(z) = \prod_{k=1}^\infty f_k(z) = \exp \left( \sum_{k=1}^\infty \log f_k(z) \right) \neq 0 $$
    and thus $f$ is non-vanishing.
\end{proof}

\begin{definition} \label{def:canonical-factors}
    The \emph{canonical factors} are defined at $z \in \C$ by
    $$ W(z; 0) \coloneqq 1-z, \quad \textrm{and} \quad W(z; n) \coloneqq (1-z) \exp \left( \sum_{k=1}^n \frac{z^k}{k} \right), \quad n \in \N. $$
    The number $n$ is called the \emph{degree} of the canonical factor.
\end{definition}

\begin{lemma} \label{lem:estimate-canonical-factors}
    Let $n \in \N_0$ and $\vert z \vert \leq 1/2$, then $ \vert 1 - W(z; n) \vert \leq 2e \vert z \vert^{n+1} $.
\end{lemma}

\begin{proof}
    For $n = 0$ there is nothing to show. Note that for such given $z$ we can expand $\log (1 - z) = - \sum_{k=1}^\infty \frac{z^k}{k}$, therefore
    \begin{equation*}
        W(z; n) = \exp \left( \log(1-z) + \sum_{k=1}^n \frac{z^k}{k} \right) = \exp \zeta,
    \end{equation*}
    where $\zeta \coloneqq - \sum_{k=n+1}^\infty z^k / k$. Furthermore, we have that
    \begin{equation*}
        \vert \zeta \vert \leq \vert z \vert^{n+1} \sum_{k=n+1}^\infty \vert z \vert^{k-n-1} / k \leq \vert z \vert^{n+1} \sum_{j=0}^\infty 2^{-j} \leq 2 \vert z \vert^{n+1}
    \end{equation*}
    and in particular $\vert \zeta \vert \leq 1$. We conclude
    \begin{equation*}
        \vert 1 - W(z; n) \vert = \vert \exp \zeta - 1 \vert \leq \sum_{k=1}^\infty \frac{\vert \zeta \vert^k}{k!} \leq \vert \zeta \vert \sum_{k=0}^\infty \frac{1}{k!} \leq 2 e \vert z \vert^{n+1}.
    \end{equation*}
\end{proof}

The next theorem -- which is also known as the \emph{Weierstraß Product Theorem} -- now constructively asserts the existence of entire function with prescribed zeros.

\begin{theorem} \label{thm:function-with-prescribed-zeros}
    Let $M \subseteq \N$ and $\mathbf{z} = (z_m)_{m \in M}$ be a family in $\C^\times$ without accumulation points. Then there is a family $(p_m)_{m \in M}$ in $\N_0$ such that
    \begin{equation}
        \Pi(z) \coloneqq \prod_{m \in M} W\left(\frac{z}{z_m}; p_m\right)
    \end{equation}
    is an entire function such that if $z_m, m \in M$ occurs in $\mathbf{z}$ exactly $n$ times, then $\Pi$ has a zero of order $n$ at $z_m$. If
    \begin{equation}
        \mu \coloneqq \min \left\{ p \in \N_0 : \sum_{m \in M} \frac{1}{\vert z_m \vert^{p+1}} < \infty \right\}
    \end{equation}
    is finite, then one can take $p_m = \mu, m \in M$.
\end{theorem}

\begin{proof}
    If $M$ is finite there is nothing to show. We may thus assume that $M$ is infinite and without loss of generality $M = \N$. Thus $\mathbf{z}$ is a sequence, which we can assume to be arranged in non-decreasing order according to the moduli its elements.

    We claim that there exists a sequence $(p_n)_{n \in \N}$ in $\N_0$ such that
    \begin{equation}
        \sum_{n=1}^\infty \left\vert \frac{z}{z_n} \right\vert^{p_n + 1}
    \end{equation}
    converges compactly in $\C$. Let $K \subset B_R(0) \subset \C$ be compact. Note that since $\mathbf{z}$ has no accumulation points we necessarily have $\lim_{n \to \infty} \vert z_n \vert = \infty$, thus there is some $n_0$ such that for $n > n_0$ we have $\vert z_n \vert \geq R + 1$. Therefore, for any $z \in K$,
    \begin{equation*}
        \sum_{n=1}^\infty \left\vert \frac{z}{z_n} \right\vert^{p_n + 1} \leq \sum_{n=1}^{n_0 - 1}  \left\vert \frac{z}{z_n} \right\vert^{p_n + 1} + \sum_{n=n_0}^\infty \left( \frac{R}{R+1} \right)^{p_n + 1},
    \end{equation*}
    and the rightmost series is finite if, for example, $p_n = n, n \in \N$. Therefore the series is uniformly bounded and thus compactly convergent, concluding our claim. Note that if $\mu$ is finite then we can take $p_n = \mu, n \in \N$, since for $z \in K$
    \begin{equation*}
        \sum_{n=1}^\infty \left\vert \frac{z}{z_n} \right\vert^{\mu + 1} \leq \vert z \vert^\mu \sum_{n=1}^\infty \frac{1}{\vert z_n \vert^{\mu+1}} \leq R^\mu \sum_{n=1}^\infty \frac{1}{\vert z_n \vert^{\mu+1}},
    \end{equation*}
    where the rightmost sum is a finite constant by the definition of $\mu$.

    It remains to show that is $\Pi$ is an entire function that satisfies the desired properties.Let $R > 0$, then it suffices to show that $\Pi$ has is holomorphic in $B_R(0)$ and vanishes only at points of $\mathbf{z}$ that are also in $B_R(0)$. We write
    $$ \Pi(z) = \left( \prod_{\substack{n=1 \\ \vert z_n \vert < 2R }}^\infty W\left(\frac{z}{z_n}; p_n\right) \right) \left( \prod_{\substack{n=1 \\ \vert z_n \vert \geq 2R }}^\infty W\left(\frac{z}{z_n}; p_n\right) \right) \eqqcolon \Pi_1(z) \Pi_2(z). $$
    Clearly $\Pi_1$ is a finite product which vanishes at points of $\mathbf{z}$ in $B_R(0)$ and nowhere else. Note that for $\vert z_m \vert \geq 2R$ we have $\vert z / z_m \vert \leq R / \vert z_m \vert \leq 1 / 2$, hence by \Cref{lem:estimate-canonical-factors} we have
    $$ \sum_{\substack{n=1 \\ \vert z_n \vert \geq 2R }}^\infty \left\vert 1 - W\left(\frac{z}{z_n}; p_n\right) \right\vert \leq 2e \sum_{\substack{n=1 \\ \vert z_n \vert \geq 2R }}^\infty \left\vert \frac{z}{z_n} \right\vert^{p_n + 1} $$
    and the compact convergence of the right sum, which stems from our choice of the $p_n, n \in \N$, implies compact convergence of the left sum. By \Cref{lem:infinite-product-criteria} we thereby have that $\Pi_2 \in H(B_R(0))$ and vanishes nowhere. Therefore $\Pi$ has the desired properties and, since $R$ was arbitrary, this concludes the proof.
\end{proof}

\begin{definition} \label{def:canonical-product}
    With the notation of \Cref{thm:function-with-prescribed-zeros}, $\Pi$ is called a \emph{Weierstraß product}\footnote{Note that this product is not uniquely determined, since the sequence $(p_m)_{m \in M}$ is not unique.} formed from the family $\mathbf{z}$. If $\mu < \infty$, then
    \begin{equation}
        \Pi(z) \coloneqq \prod_{m \in M} W\left(\frac{z}{z_m}; \mu \right)
    \end{equation}
    is called a \emph{(Weierstraß) canonical product} and $\mu$ is called the \emph{genus} of the canonical product, which will also be denoted as $\mu_\Pi$. Otherwise the genus of $\Pi$ is said to be infinite.
\end{definition}

\begin{remark}
    By comparing \Cref{def:zero-exponent} and \Cref{def:canonical-product} we immediately observe that if $\lambda_\Pi < \infty$, then $\mu_\Pi \leq \lambda_\Pi \leq \mu_\Pi + 1$. If therefore $\lambda_\Pi$ is an integer, and $\Pi$ has infinitely many zeros, the series
    $$ \sum_{m \in M} \frac{1}{\vert z_m \vert^{\lambda}}, $$
    will converge if $\lambda_\Pi = \mu_\Pi + 1$ and diverge if $\lambda_\Pi = \mu_\Pi$.
\end{remark}

Just as any polynomial decomposes into linear factors involving their zeros by the Fundamental Theorem of Algebra, Weierstraß products allow us to obtain a similar factorization result for arbitrary entire functions, known as the \emph{Weierstraß Factorization Theorem}.

\begin{theorem}[Weierstraß] \label{thm:Weierstraß}
    Let $f \in H(\C)$ be not identically zero and let $m \in \N_0$ be the order of the zero of $f$ at the origin. Then there exists $g \in H(\C)$ such that
    \begin{equation}
        f(z) = z^m e^{g(z)} \Pi(z),
    \end{equation}
    where $\Pi$ denotes a Weierstraß product formed from the zeros of $f$.
\end{theorem}

\begin{proof}
    Since $f(z)$ and $z^m \Pi(z)$ have the same zeros with identical multiplicities the function
    $$ \varphi(z) \coloneqq \frac{f(z)}{z^m \Pi(z)} $$
    is entire and vanishes nowhere. Therefore we can write
    $$ \varphi(z) = e^{g(z)} $$
    for some $g \in H(\C)$ and rearranging terms yields the desired representation.
\end{proof}

\begin{definition} \label{def:genus}
    In the context of Weierstraß' Theorem, if $\Pi$ is of finite genus and $g$ is a polynomial, then the \emph{genus of the entire function $f$} is defined as
    \begin{equation}
        \mu_f \coloneqq \max \{ \mu_\Pi, \deg g \}.
    \end{equation}

    If $g$ is not a polynomial or if $\Pi$ is of infinite genus, then $f$ is said to be of infinite genus.
\end{definition}

We wish to further estimate canonical products. To do this we first focus on the canonical factors.

\begin{lemma} \label{lem:elementary-factor-estimate}
    For $p \in \N$ and all $z \in \C$ we have
    \begin{equation}
        \log \vert W(z; p) \vert < A_p \frac{\vert z \vert^{p+1}}{1 + \vert z \vert}, \quad \textrm{where} \quad A_p \coloneqq 3e (2 + \log p).
    \end{equation}
    For $p = 0$ we have
    \begin{equation}
        \log \vert W(z; 0) \vert \leq \log (1 + \vert z \vert).
    \end{equation}
\end{lemma}

\begin{proof}
    The second assertion is clear since
    $$ \log \vert W(z; 0) \vert = \log \vert 1 - z \vert \leq \log (1 + \vert z \vert). $$
    If $\vert z \vert \leq \frac{p}{p+1}$ then in particular we have $\vert z \vert < 1$. Since furthermore $1 + \vert z \vert < 2 < A_p$ we get
    \begin{align*}
        \log \vert W(z; p) \vert &= \log \left\vert (1-u) \exp \left( \sum_{k=1}^p \frac{z^k}{k} \right) \right\vert = \Re \left( \log(1-u) + \sum_{k=1}^p \frac{z^k}{k} \right) = \\
        &= -\Re \sum_{k=p+1}^\infty \frac{z^k}{k} \leq \sum_{k=p+1}^\infty \frac{\vert z \vert^k}{k} < \frac{\vert z \vert^{p+1}}{(p+1)(1 - \vert z \vert)} \leq \vert z \vert^{p+1} \leq \\ &\leq A_p \frac{\vert z \vert^{p+1}}{1 + \vert z \vert}.
    \end{align*}
    If otherwise $\vert z \vert > \frac{p}{p+1}$ then we have $\frac{1}{\vert z \vert} < 1 + \frac{1}{p}$ and $2 \vert z \vert > 1$. From there we conclude
    \begin{align*}
        \log \vert W(z; p) \vert &= \log \left\vert (1-z) \exp \left( \sum_{k=1}^p \frac{z^k}{k} \right) \right\vert \leq \log(1 + \vert z \vert) + \sum_{k=1}^p \frac{\vert z \vert^k}{k} \leq \\
        &\leq 2 \vert z \vert + \sum_{k=2}^p \frac{\vert z \vert^k}{k} = \vert z \vert^p \left( 2 \left\vert \frac{1}{z} \right\vert^{p-1} + \sum_{k=2}^p \frac{1}{k} \left\vert \frac{1}{z} \right\vert^{p-k} \right) < \\
        &< \vert z \vert^p \left( 2 \left(1 + \frac{1}{p} \right)^{p-1} + \sum_{k=2}^p \frac{1}{k} \left( 1 + \frac{1}{p} \right)^{p-k} \right) < \\
        &< \vert z \vert^p \left( 1 + \frac{1}{p} \right)^p \left( 2 + \sum_{k=1}^p \frac{1}{k} \right) < \vert z \vert^p e \left( 2 + \int_1^p \frac{1}{t} \diff t \right) = \\
        &= \vert z \vert^p e \left( 2 + \log p \right) = \frac{1 + \vert z \vert}{1 + \vert z \vert} \vert z \vert^p e \left( 2 + \log p \right) < \\
        &< \frac{3 \vert z \vert}{1 + \vert z \vert} \vert z \vert^p e \left( 2 + \log p \right) = A_p \frac{\vert z \vert^{p+1}}{1 + \vert z \vert}. \qedhere
    \end{align*}
\end{proof}

\begin{lemma} \label{lem:canonical-product-estimate}
    Let $M \subseteq \N$ and $(z_m)_{m \in M}$ be a family in $\C^\times$ without accumulation points. If for given $p \in \N_0$ it holds that
    $$ \sum_{m \in M} \frac{1}{\vert z_m \vert^{p+1}} < \infty, $$
    then the entire function
    $$ \Pi(z) \coloneqq \prod_{m \in M} W \left( \frac{z}{z_m}; p \right) $$
    satisfies
    \begin{equation}
        \log \vert \Pi(z) \vert < k_p r^p \left( \int_0^r \frac{n_\Pi(t)}{t^{p+1}} \diff t + r \int_r^\infty \frac{n_\Pi(t)}{t^{p+2}} \diff t \right),
    \end{equation}
    for all $z \in \C^\times$ and $r = \vert z \vert$, where
    $$ k_0 \coloneqq 1, \quad \textrm{and} \quad k_p \coloneqq 3e(p+1)(2 + \log p), \quad p \in \N. $$
\end{lemma}

\begin{proof}
    Assuming $p > 0$, combining \Cref{lem:elementary-factor-estimate,lem:zeros-measure} gives
    \begin{align*}
        \log \vert \Pi(z) \vert &= \prod_{m \in M} W \left( \frac{z}{z_m}; p \right) = \sum_{m \in M} \log \left\vert W \left( \frac{z}{z_m}; p \right) \right\vert < \\
        &< A_p \sum_{m \in M} \frac{\left\vert \frac{z}{z_m} \right\vert^{p+1}}{1 + \left\vert \frac{z}{z_m} \right\vert} = A_p r^{p+1} \sum_{m \in M} \frac{1}{\vert z_m \vert^p (\vert z_m \vert + r)} = \\
        &= A_p r^{p+1} \int_{(0, \infty)} \frac{1}{t^p (t+r)} \diff \omega_\Pi(t),
    \end{align*}
    \Cref{lem:zeros-measure} also yields, for $s > 0$,
    \begin{equation*}
        \int_{(0, s)} \frac{1}{t^p (t+r)} \diff \omega_\Pi(t) = \frac{n_\Pi(s)}{s^p (s + r)} + \int_{(0, s)} n_\Pi(t) \frac{(p+1)t + pr}{t^{p+1} (t+r)^2} \diff t.
    \end{equation*}
    As seen in the proof of \Cref{prop:zeros-order-equals-zeros-exponent}, the left term on the right-hand side tends to $0$ as $s \to \infty$. Therefore we obtain
    \begin{align*}
        \log \vert \Pi(z) \vert &< A_p r^{p+1} \int_{(0, \infty)} \frac{1}{t^p (t+r)} \diff \omega_\Pi(t) = A_p r^{p+1} \int_0^\infty n_\Pi(t) \frac{(p+1)t + pr}{t^{p+1} (t+r)^2} \diff t < \\
        &< A_p (p+1) r^{p+1} \int_0^\infty \frac{n_\Pi(t)}{t^{p+1}(t+r)} \diff t < \\
        &< A_p (p+1) r^{p+1} \left( \frac{1}{r} \int_0^r \frac{n_\Pi(t)}{t^{p+1}} \diff t + \int_r^\infty \frac{n_\Pi(t)}{t^{p+2}} \diff t \right) = \\
        &= k_p r^{p} \left( \int_0^r \frac{n_\Pi(t)}{t^{p+1}} \diff t + r \int_r^\infty \frac{n_\Pi(t)}{t^{p+2}} \diff t \right).
    \end{align*}
    The case $p=0$ is verified in the same fashion:
    \begin{align*}
        \log \vert \Pi(z) \vert &= \sum_{m \in M} \log \left\vert W \left( \frac{z}{z_m}; 0 \right) \right\vert \leq \sum_{m \in M} \log \left( 1 + \left\vert \frac{z}{z_m} \right\vert \right) \leq \\
        &\leq \sum_{m \in M} \left\vert \frac{z}{z_m} \right\vert = r \sum_{m \in M} \frac{1}{\vert z_m \vert} = r \int_{(0, \infty)} \frac{1}{t} \diff \omega(t) = -r \int_0^\infty -\frac{n_\Pi(t)}{t^2} \diff t < \\
        &< r \left( \int_0^r \frac{n_\Pi(t)}{tr} \diff t + \int_r^\infty \frac{n_\Pi(t)}{t^2} \diff t \right) = \int_0^r \frac{n_\Pi(t)}{t} \diff t + r \int_r^\infty \frac{n_\Pi(t)}{t^2} \diff t. \qedhere
    \end{align*}
\end{proof}

\begin{theorem} \label{thm:exponent-of-convergence-Weierstraß-product}
    Let $\Pi \in H(\C)$ be a canonical product, then $\lambda_\Pi = \rho_\Pi$.
\end{theorem}

\begin{proof}
    By \Cref{thm:inequality-order-exponent-of-convergence} we already have $\lambda_\Pi \leq \rho_\Pi$, it remains to show that $\rho_\Pi \leq \lambda_\Pi$. We first recall that the convergence exponent always satisfies $\mu \leq \lambda_\Pi \leq \mu + 1$, where $\mu \coloneqq \mu_\Pi$.
    
    Suppose $\lambda_\Pi < \mu + 1$ and let $\lambda_\Pi < \lambda < \mu + 1$. By \Cref{prop:zeros-order-equals-zeros-exponent} we have
    $$ \lambda_\Pi = \nu_\Pi = \limsup_{r \to \infty} \frac{\log n_\Pi(r)}{\log r} < \lambda. $$
    Therefore there exists some $r_0 > 0$ such that for all $r \geq r_0$ we have $n_\Pi(r) \leq r^\lambda$. Choose $\varepsilon > 0$ small enough such that $n_\Pi(\varepsilon) = 0$, then
    \begin{equation*}
        n_\Pi(r) = \frac{n_\Pi(r)}{r^\lambda} r^\lambda \leq \frac{n_\Pi(r_0)}{\varepsilon^\lambda} r^\lambda
    \end{equation*}
    and choosing $C_\lambda > \max \{ 1, n_\Pi(r_0) / \varepsilon^\lambda \}$ yields $n_\Pi(r) \leq C_\lambda r^\lambda$ for all $r > 0$. Applying this inequality to \Cref{lem:canonical-product-estimate} we get, for $r = \vert z \vert$,
    \begin{align*}
        \log \vert \Pi(z) \vert &< k_\mu r^\mu \left( \int_0^r \frac{n_\Pi(t)}{t^{\mu + 1}} \diff t + r \int_r^\infty \frac{n_\Pi(t)}{t^{\mu + 2}} \diff t \right) \leq \\
        &\leq k_\mu r^\mu C_\lambda \left( \int_0^r t^{\lambda - \mu - 1} \diff t + r \int_r^\infty t^{\lambda - \mu - 2} \diff t \right) = \\
        &= k_\mu r^\mu C_\lambda \left( \frac{r^{\lambda - \mu}}{\lambda - \mu} - \frac{r^{\lambda - \mu}}{\lambda - \mu - 1} \right) = k_\mu C_\lambda \left( \frac{1}{\lambda - \mu} - \frac{1}{\lambda - \mu - 1} \right) r^\lambda.
    \end{align*}
    From this it follows $M_\Pi(r) = \O(\exp r^\lambda)$ and thus that $\rho_\Pi \leq \lambda$. Letting $\lambda \searrow \lambda_\Pi$ yields $\rho_\Pi \leq \lambda_\Pi$.

    If we otherwise assume $\lambda_\Pi = \mu + 1$, then by \Cref{lem:zeros-measure}
    $$ \int_{(0, r)} \frac{1}{t^{\mu + 1}} \diff \omega_\Pi(t) = \frac{n_\Pi(r)}{r^{\mu + 1}} + (\mu + 1) \int_0^r \frac{n_\Pi(t)}{t^{\mu + 2}} \diff t, $$
    wherein the leftmost integral converges as $r \to \infty$. Imitating the proof of \Cref{prop:zeros-order-equals-zeros-exponent} we see that the left term on the right-hand side converges to $0$ as $r \to \infty$, thus the rightmost integral converges as well. Let $\varepsilon > 0$. These two convergences imply that there is an $r_0 > 0$ such that for all $r \geq r_0$ we have
    $$ \int_{r_0}^\infty \frac{n_\Pi(t)}{t^{\mu + 2}} \diff t < \varepsilon, \quad \textrm{and} \quad \frac{n_\Pi(r)}{r^{\mu + 1}} < \varepsilon. $$
    Inserting these results into \Cref{lem:canonical-product-estimate} yields, for $r = \vert z \vert \geq r_0$,
    \begin{align*}
        \log \vert \Pi(z) \vert &< k_\mu r^\mu \left( \int_0^r \frac{n_\Pi(t)}{t^{\mu + 1}} \diff t + r \int_r^\infty \frac{n_\Pi(t)}{t^{\mu + 2}} \diff t \right) = \\
        &= k_\mu r^\mu \left( \int_0^{r_1} \frac{n_\Pi(t)}{t^{\mu + 1}} \diff t + \int_{r_1}^r \frac{n_\Pi(t)}{t^{\mu + 1}} \diff t + r \int_r^\infty \frac{n_\Pi(t)}{t^{\mu + 2}} \diff t \right) \leq \\
        &\leq k_\mu r^\mu \left(C + (r - r_1) \varepsilon + r \varepsilon \right) \leq k_\mu (C r^\mu + 2 \varepsilon r^{\mu + 1}).
    \end{align*}
    The right-most term asymptotically dominates, therefore $M_\Pi(r) = \O(\exp(\varepsilon r^{\mu + 1}))$ and thus $\Pi$ is at most of order $\mu + 1 = \lambda_\Pi$ and, since $\varepsilon$ was arbitrary, minimal type.
\end{proof}

Just as the order of a canonical product is determined by the order of growth of its zeros, its type is (to some extent) determined by the upper density of its zeros. A peculiarity arises when considering canonical products of integral order, in which case the type may also depend on the argument of the zeros \cite{levin-distribution-of-zeros}. However, for non-integral order we can obtain the following result:

\begin{theorem} \label{thm:canonical-product-type-density}
    Let $\Pi \in H(\C)$ be a canonical product. If $\lambda_\Pi$ is not an integer, then $\Pi$ is of maximal, minimal or normal type according to whether $\Delta_\Pi$ is equal to infinity, zero, or a positive, real number.
\end{theorem}

\begin{proof}
    by \Cref{lem:zeros-bounded-by-modulus} there is a constant $K > 0$ such that for all sufficiently large $r$ we have $n_\Pi(r) \leq K \log M_\Pi(er)$ and by \Cref{prop:zeros-order-equals-zeros-exponent} and \Cref{thm:exponent-of-convergence-Weierstraß-product} we have $\nu_\Pi = \lambda_\Pi = \rho_\Pi$. Therefore we have, for large $r$,
    \begin{equation} \label{eq:density-smaller-constant-type}
        \frac{n_\Pi(r)}{r^{\nu_\Pi}} = \frac{n_\Pi(r)}{r^{\rho_\Pi}} \leq K e^{\rho_\Pi} \frac{\log M_\Pi(er)}{(e r)^{\rho_\Pi}}.
    \end{equation}
    We now consider two cases. If $\Delta_\Pi$ is infinite then taking limits superior as $r \to \infty$ in the above shows than $\Pi$ must be of maximal type.

    If $\Delta_\Pi$ is finite then let $\Delta > \Delta_\Pi$. By definition of the limit superior there is an $r_0 > 0$ such that, for all $r \geq r_0$, we have
    \begin{equation*}
        n_\Pi(r) \leq \Delta r^{\nu_\Pi} = \Delta r^{\rho_\Pi}. 
    \end{equation*}
    As demonstrated in the proof of \Cref{thm:exponent-of-convergence-Weierstraß-product}, we can find a constant $C_{\lambda_\Pi} > 0$ such that for all $r > 0$ we have
    \begin{equation*}
        n_\Pi(r) \leq C_{\lambda_\Pi} \Delta r^{\rho_\Pi}
    \end{equation*}
    The genus $\mu$ of the canonical product satisfies $\mu < \lambda_\Pi < \mu + 1$, since $\lambda_\Pi$ is not an integer. Therefore, by \Cref{lem:canonical-product-estimate}, for all $r > 0$ and $\vert z \vert = r$ we have
    \begin{align*}
        \log \vert \Pi(z) \vert &< k_\mu r^\mu \left( \int_0^r \frac{n_\Pi(t)}{t^{\mu + 1}} \diff t + r \int_r^\infty \frac{n_\Pi(t)}{t^{\mu + 2}} \diff t \right) \leq \\
        &\leq k_\mu r^\mu C_{\lambda_\Pi} \Delta \left( \int_0^r t^{\rho_\Pi - \mu - 1} \diff t + r \int_r^\infty t^{\rho_\Pi - \mu - 2} \diff t \right) = \\
        &= k_\mu r^\mu C_{\lambda_\Pi} \Delta \left( \frac{r^{\rho_\Pi - \mu}}{\rho_\Pi - \mu} - \frac{r^{\rho_\Pi - \mu}}{\rho_\Pi - \mu - 1} \right) = k_\mu C_{\lambda_\Pi} \Delta \left( \frac{1}{\rho_\Pi - \mu} - \frac{1}{\rho_\Pi - \mu - 1} \right) r^{\rho_\Pi}.
    \end{align*}
    Dividing by $r^{\rho_\Pi}$, taking limits superior as $r \to \infty$ and letting $\Delta \searrow \Delta_\Pi$ therefore yields $\tau_\Pi \leq C \Delta_\Pi$ for a constant $C > 0$. Combining with \eqref{eq:density-smaller-constant-type}, we get
    \begin{equation*}
        C_1 \tau_\Pi \leq \Delta_\Pi \leq C_2 \tau_\Pi
    \end{equation*}
    for constants $C_1, C_2 > 0$. This shows that, if $\Delta_\Pi \in (0, \infty)$, then $\Pi$ is of normal type, and if $\Delta_\Pi = 0$, then $\Pi$ is of minimal type.
\end{proof}

\begin{samepage}    
\begin{example}
    Consider the canonical product $\Pi$ with zeros at $ n^2 (\log n)^2, n \in \N $. One immediately shows $n_\Pi(r) = \O(\varepsilon r^\frac{1}{2})$ as $r \to \infty$ for all $\varepsilon > 0$. Therefore $\rho_\Pi = \frac{1}{2}$ and $\Delta_\Pi = 0$, and \Cref{thm:canonical-product-type-density} shows that $\Pi$ is of minimal type.

    Following a similar reasoning, the canonical product with zeros at $ n^2 (\log n)^{-2} $ can be shown to be of order $\frac{1}{2}$ and maximal type.
\end{example}
\end{samepage}