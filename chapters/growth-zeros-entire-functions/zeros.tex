\section{Order, Density and Exponent of Convergence of Zeros} \label{sec:zeros}

% We will mostly follow the contents of chapter 1 in Levin \cite{levin-distribution-of-zeros}.

We first recall a rather explicit connection between the moduli of the zeros of a holomorphic function and the modulus of the function itself:

\begin{theorem}[Jensen] \label{thm:jensen}
    Let $R > 0$, $f \in H(B_R(0))$ with $f(0) \neq 0$ and let $r_1, r_2, \hdots$ denote the moduli of the zeros of $f$ in $B_{R}(0)$, repeated by multiplicity, arranged in a non-decreasing sequence. Then, for $r_n < r < r_{n+1}$, we have
    \begin{equation}
        \frac{1}{2 \pi} \int_0^{2\pi} \log \vert f(r e^{i \theta}) \vert \diff \theta = \log \vert f(0) \vert + \log \frac{r^n}{r_1 \hdots r_n}.
    \end{equation}
\end{theorem}

\begin{definition}
    Let $f \in H(B_R(0))$ be not identically zero. Then, for $0 < r < R$, we define
    \begin{equation}
        n_f(r) \coloneqq \vert \{ z \in \cl{B_r(0)} : f(z) = 0 \} \vert,
    \end{equation}
    that is, the number of zeros of $f$ in $\cl{B_{r}(0)}$.
\end{definition}

This zero-counting function is, in some scenarios, simpler to work with than the sequence of zeros. For instance, it can be used to obtain an equivalent version of Jensen's Theorem:

\begin{corollary} \label{cor:jensen-nf}
    Let $f \in H(B_R(0))$ with $f(0) \neq 0$. Then, for $0 < r < R$, we have
    \begin{equation}
        \frac{1}{2 \pi} \int_0^{2\pi} \log \vert f(re^{i \theta}) \vert \diff \theta = \log \vert f(0) \vert + \int_0^r \frac{n_f(s)}{s} \diff s
    \end{equation}
\end{corollary}

\begin{proof}
    Let $r_1, r_2, \hdots$ denote the moduli of the zeros of $f$ in $B_{R}(0)$, repeated by multiplicity, arranged in a non-decreasing sequence. Then, for any $r_n < r < r_{n+1}$, we obtain
    \begin{align*}
        \log \frac{r^n}{r_1 \hdots r_n} &= \sum_{k=1}^n \log \frac{r}{r_k} = \sum_{k=1}^n \int_{r_k}^r \frac{1}{s} \diff s = \\
        &= \sum_{k=1}^n \int_0^r \1_{(r_k, \infty)}(s) \frac{1}{s} \diff s = \int_0^r \left( \sum_{k=1}^n \1_{(r_k, \infty)}(s) \right) \frac{1}{s} \diff s = \\
        &= \int_0^r \frac{n_f(s)}{s} \diff s
    \end{align*}
    and \Cref{thm:jensen} establishes the assertion.
\end{proof}

This gives an immediate connection between the zeros and the growth of the modulus of an entire function:

\begin{lemma} \label{lem:zeros-bounded-by-modulus}
    If $f \in H(\C)$, then
    $$ n_f(r) = \O( \log M_f(er) ). $$
\end{lemma}

\begin{proof}
    We may assume $\vert f(0) \vert = 1$. Let $r > 0$, then since $n_f$ is non-negative and non-decreasing we have
    \begin{equation*}
        n_f(r) \leq \int_r^{er} \frac{n_f(t)}{t} \diff t \leq \frac{1}{2 \pi} \log \vert f(er e^{i \theta}) \vert \diff \theta \leq \log M_f(er),
    \end{equation*}
    establishing the assertion.
\end{proof}

\Cref{lem:zeros-bounded-by-modulus} shows that the more zeros a function $f$ has, the faster $M_f$ must grow as $r \to \infty$. The converse is naturally false; by composing exponentials one can obtain an entire function $f$ for which $M_f$ grow arbitrarily fast, yet $f$ has no zeros.

\begin{definition} \label{def:zero-exponent}
    Let $M \subseteq \N$ and $\mathbf{z} = (z_m)_{m \in M}$ be a family of points\footnote{Note that the exponent of convergence is only affected by their moduli, not by their arguments.} in $\C^\times$. Then
    $$ \lambda_{\mathbf{z}} \coloneqq \inf \left\{ \lambda > 0 : \sum_{m \in M} \frac{1}{\vert z_m \vert^\lambda} < \infty \right\} $$
    is called the \emph{exponent of convergence} of the family $\mathbf{z}$.

    Let $f \in H(\C)$, then the \emph{exponent of convergence of the zeros of $f$} is defined as the exponent of convergence of the non-zero roots of $f$, repeated by multiplicity, and is denoted $\lambda_f$. If $f$ is constant, we set $\lambda_f = 0$ by convention.

    Furthermore, for any $a \in \C$, the \emph{exponent of convergence of the $a$-points of $f$} is defined as exponent of convergence of the zeros of $f(z) - a$ and will be denoted $\lambda_f^{(a)}$.
\end{definition}

If $f \in H(\C)$ assumes some value $a \in \C$ only finitely often, then clearly $\lambda_f^{(a)} = 0$.

\begin{theorem} \label{thm:inequality-order-exponent-of-convergence}
    If $f \in H(\C)$ is of finite order, then $\lambda_f \leq \rho_f$.
\end{theorem}

\begin{proof}
    We may assume $f(0) \neq 0$. If $\lambda_f = 0$ there is nothing to show. Let $\rho > \rho_f$ then
    by \Cref{lem:zeros-bounded-by-modulus} and \Cref{prop:order-infimum} we have $n_f(r) = \O(\log M_f(er))$ and $M_f(r) = \O(\exp r^\rho)$, respectively. Therefore $n_f(r) = \O(r^\rho)$, thus there exists a constant $K_1 > 0$ such that for sufficiently large $r > 0$ we have $n_f(r) \leq K_1 r^\rho$.
    
    If $(r_j)_{j \in \N}$ denote the non-zero moduli of the zeros of $f$, repeated by multiplicity, arranged in non-decreasing order, then for any $m \in \N$ we have $m \leq n_f(r_m) \leq K_1 r_m^{\rho}$. Let $0 < \lambda < \lambda_f$ then this implies
    \begin{align*}
        \left( \frac{1}{r_m} \right)^{\lambda} \leq K_2 \left( \frac{1}{m} \right)^{\lambda/\rho}.
    \end{align*}
    for some constant $K_2 > 0$. Therefore
    \begin{equation*}
        \sum_{m=1}^\infty \left( \frac{1}{r_m} \right)^{\lambda} \leq K_2 \sum_{m=1}^\infty \left( \frac{1}{m} \right)^{\lambda/\rho}.
    \end{equation*}
    The left-hand side diverges by \Cref{def:zero-exponent}, thus so does the right-hand side. But the latter diverges if and only if $\frac{\lambda}{\rho} \leq 1$, therefore letting $\lambda \nearrow \lambda_f$ and then $\rho \searrow \rho_f$ yields $\lambda_f \leq \rho_f$.
\end{proof}

\begin{remark}
    The function $f(z) \coloneqq \exp z$ is of order one and has no zeros -- thus we observe that we may have $\lambda_f < \rho_f$ in some cases.
\end{remark}

A more precise analysis of the density of the zeros of an entire function $f$ is contained in the growth of the zero-counting function $n_f$. We therefore introduce the following definitions:

\begin{definition} \label{def:order-density-zero-counting}
    Let $f \in H(\C)$. As the \emph{order} of the zero-counting function $n_f$ we define
    \begin{equation}
        \nu_f \coloneqq \limsup_{r \to \infty} \frac{\log n_f(r)}{\log r}.
    \end{equation}
    If $\nu_f$ is finite, we additionally define the \emph{upper density} of the zeros of $f$ as
    \begin{equation}
        \Delta_f \coloneqq \limsup_{r \to \infty} \frac{n_f(r)}{r^{\nu_f}}.
    \end{equation}
    If the limit exists, then $\Delta_f$ is simply called the \emph{density}.
\end{definition}

The following lemma provides great utility in further analyzing the function $n_f$.

\begin{lemma} \label{lem:zeros-measure}
    Let $f \in H(\C)$ and denote by $(r_m)_{m \in M}$ the moduli of the zeros of $f$, repeated by multiplicity, arranged in non-decreasing order. Then $f$ induces a unique Lebesgue-Stieltjes measure $\omega_f$ on $(0, \infty)$, such that, for any non-negative, measureable function $\varphi$ on $(0, \infty)$, it holds that
    \begin{equation*}
        \sum_{m \in M} \varphi(r_m) = \int_{(0, \infty)} \varphi \diff \omega_f.
    \end{equation*}
    If in addition $\varphi$ is monotone and continuously differentiable on $(0, \infty)$, then for all $r > 0$ it holds that
    \begin{equation*}
        \int_{(0, r)} \varphi \diff \omega_f = n_f(r) \varphi(r) - \int_0^r n_f(t) \varphi'(t) \diff t.
    \end{equation*}
\end{lemma}

\begin{proof}
    From the representation
    $$ n_f(r) = \sum_{m \in M} \1_{[r_m, \infty)}(r) $$
    we observe that $n_f$ is right-continuous, non-negative and non-decreasing. Therefore $n_f$ induces a unique Lebesgue--Stieltjes measure $\omega_f$. Since an indicator function of the form $\1_{[a, \infty)}$, for $a \in \R$, corresponds to a Dirac measure $\delta_{\{ a \}}$, and the correspondence between Lebesgue-Stieltjes measures and their distribution functions is linear in nature, we have
    $$ \omega = \sum_{m \in M} \delta_{\{r_m\}}. $$
    From this it follows that
    \begin{equation*}
        \sum_{m \in M} \varphi(r_m) = \sum_{m \in M} \int_{(0, \infty)} \varphi \diff \delta_{\{r_m\}} = \int_{(0, \infty)} \varphi \diff \left( \sum_{m \in M} \delta_{\{r_m\}} \right) = \int_{(0, \infty)} \varphi \diff \omega_f.
    \end{equation*}
    Assume $\varphi$ is monotone and continuously differentiable on $(0, \infty)$. Then $\varphi$ also induces a Lebesgue--Stieltjes measure $\eta_\varphi$, with density $\varphi'$ with respect to the Lebesgue measure. Thus, for any $0 < \varepsilon < r < \infty$, integration by parts gives
    \begin{equation*}
        \int_{(\varepsilon, r)} \varphi \diff \omega_f + \int_{(\varepsilon, r)} n_f \diff \eta_\varphi = n_f(r) \varphi(r) - n_f(\varepsilon) \varphi(\varepsilon).
    \end{equation*}
    Since $\kappa \coloneqq \min_{m \in M} r_m > 0$ we have $n_f(s) = 0$ for $0 < s < \kappa$. Letting $\varepsilon \to 0$ and using the density of $\eta_\varphi$ therefore yields
    \begin{equation*}
        \int_{(0, r)} \varphi \diff \omega_f + \int_0^r n_f(t) \varphi'(t) \diff t = n_f(r) \varphi(r)
    \end{equation*}
    and rearranging terms provides the desired result.
\end{proof}

\begin{proposition} \label{prop:zeros-order-equals-zeros-exponent}
    Let $f \in H(\C)$, then $\lambda_f$ is finite if and only if $\nu_f$ is finite and in either case we have $\lambda_f = \nu_f$.
\end{proposition}

\begin{proof}
    If $f$ only has finitely many zeros the assertion is clear. Let $\lambda > \lambda_f$ and $r > 0$. Invoking \Cref{lem:zeros-measure} with $\varphi(t) \coloneqq t^{-\lambda}$ we get
    \begin{equation*}
        \int_{(0, r)} \frac{1}{t^\lambda} \diff \omega_f(t) = \frac{n_f(r)}{r^\lambda} + \lambda \int_0^r \frac{n_f(t)}{t^{\lambda + 1}} \diff t.
    \end{equation*}
    Since $\lambda > \lambda_f$, the same lemma implies that the left integral converges as $r \to \infty$. In particular, the right-hand side must be bounded. Since the right integrand is non-negative, the right-most term is increasing in $r$ and must therefore converge as $r \to \infty$. Therefore the right-most term in
    $$ 0 \leq \frac{n_f(r)}{r^\lambda} = \lambda n_f(r) \int_r^\infty \frac{1}{t^{\lambda + 1}} \diff t \leq \lambda \int_r^\infty \frac{n_f(t)}{t^{\lambda + 1}} \diff t $$
    converges to $0$ as $r \to \infty$, from which we get
    \begin{equation} \label{eq:zeros-order-equals-zeros-exponent:density-zero}
        \lim_{r \to \infty} \frac{n_f(r)}{r^\lambda} = 0.
    \end{equation}
    Taking limits superior as $r \to \infty$ in
    $$ \frac{\log n_f(r)}{\log r} = \left( \log \frac{n_f(r)}{r^\lambda} + \log r^\lambda\right)/{\log r} = \log \frac{n_f(r)}{r^\lambda} / \log r + \lambda $$
    and then letting $\lambda \searrow \lambda_f$ yields $\nu_f \leq \lambda_f$.

    Conversely suppose $\nu_f < \infty$ and let $\varepsilon > 0$. By the definition of $\nu_f$ and the nature of the limit superior there is an $r_0 > 0$ such that
    $$ n_f(r) \leq r^{\nu_f + \varepsilon} $$
    for all $r \geq r_0$. Set $\lambda \coloneqq \nu_f + 2 \varepsilon$, then
    \begin{equation*}
        \int_{r_0}^r \frac{n_f(t)}{t^{\lambda + 1}} \diff t \leq
        \int_{r_0}^r \frac{t^{\nu_f + \varepsilon}}{t^{\nu_f + 2\varepsilon + 1}} \diff t =
        \int_{r_0}^r t^{-\varepsilon - 1} \diff t = \frac{1}{\varepsilon r_0^{\varepsilon}} - \frac{1}{\varepsilon r^{\varepsilon}}
    \end{equation*}
    and since the right-hand side remains finite as $r \to \infty$ therefore the leftmost integral converges as $r \to \infty$. Clearly \eqref{eq:zeros-order-equals-zeros-exponent:density-zero} holds for our $\lambda$, therefore \Cref{lem:zeros-measure} implies the convergence of
    \begin{equation*}
        \int_{(0, \infty)} \frac{1}{t^\lambda} \diff \omega_f(t) = \sum_{m \in M} \frac{1}{r_m^\lambda},
    \end{equation*}
    yielding $\lambda_f \leq \lambda$, and letting $\varepsilon \to 0$ results in $\lambda_f \leq \nu_f$.
\end{proof}

\iffalse
\begin{example} \label{exm:exponent-of-convergence}
    Consider $f(z) \coloneqq \sin(z) \in H(\C)$, we want to calculate $\lambda_f$ and $\rho_f$. First, let $\lambda > 0$ and recall that $f$ has zeros at $(n \pi)_{n \in \Z}$. Since
    $$ \sum_{n \in \Z \setminus \{0\}} \frac{1}{\vert n \pi \vert^\lambda} = \frac{2}{\pi^\lambda} \sum_{n=1}^\infty \frac{1}{n^\lambda} $$
    is finite if and only if $\lambda > 1$, we obtain $\lambda_f = 1$. Furthermore, we have $\vert \sin(z) \vert \leq e^{\vert z \vert}$
    and therefore $M_f(r) \leq e^r$, whereby $\rho_f \leq 1$. Finally, \Cref{thm:inequality-order-exponent-of-convergence} concludes $\rho_f = 1$.
\end{example}
\fi